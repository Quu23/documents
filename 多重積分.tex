\documentclass[a4j,dvipdfmx]{jsarticle}
\usepackage{amsmath,amssymb}
\usepackage{siunitx}
\usepackage{ascmac}
\usepackage[dvips]{graphicx}
\usepackage{listings}

\renewcommand{\thesection}{\Roman{section}}
\renewcommand{\thesubsection}{\roman{subsection}}
\begin{document}
\part*{多重積分}
\section{多重積分}
\subsection{二重積分}
一変数の関数$y=f(x)$の定積分$\displaystyle \int_a^b f(x)dx$は去年学んだ。
少しだけ復習すると、積和の極限、すなわち$\displaystyle \lim_{n\to\infty}\sum_{k=1}^{n}f(\zeta_k)\Delta x_k$が定積分であった。これから、定積分を二変数の場合に拡張する。いままでは、積分領域は線分であったが、
今度は積分領域が面になる。

$xy$平面の領域$R$で定義された連続な関数を$f(x,y)$とする。領域$R$を、各々の面積が$\Delta A_1,\Delta A_2,\cdots,\Delta A_n$の
$n$個の章領域$R_1,R_2,\cdots,R_n$に分割する。小領域$R_1$内に点$P_1(\zeta_1,\eta_1)$、小領域$R_2$内に
$P_2(\zeta_2,\eta_2),\cdots,R_n$内に点$P_n(\zeta_n,\eta_n)$を選び、積和の極限
\begin{equation}
    \sum_{k=1}^{n}f(P_k)\Delta A_k=\sum_{k=1}^{n}f(\zeta_k,\eta_k)\Delta A_k\label{6.2}
\end{equation}
を作る。各小領域の直径(領域内のに転換の距離の最大値)が0に近づくように分割を細かくしていく。この時の極限値を
\begin{equation}
    \iint_R f(x,y)dA=\lim_{n\to\infty}\sum_{k=1}^{n}f(\zeta_k.\eta_k)\Delta A_k\label{6.3}
\end{equation}
とかき、関数$f(x,y)$の領域$R$における\textbf{二重積分}という。積分記号が二つあるのは、二変数関数の定積分であることを
示し、積分記号の添え字$R$は$x,y$の値の領域を表している。

特に、$f(x,y)=1$と置けば、その二重積分は領域$R$の面積$A$を与える。すなわち、
\begin{equation}
    A=\iint_RdA\label{6.4}
\end{equation}
\subsection{二重積分と体積}
定積分が面積と関係していたように、二重積分\eqref{6.3}式は体積と関係している。関数$z=f(x,y)$は領域
$R$で正とする。積和\eqref{6.2}の各項$f(\zeta_k,\eta_k)\Delta A_k$は、高さが$z_k=f(\zeta_k,\eta_k)$で、上下の平行面の面積が$\Delta A_k$の
垂直な``柱"の体積を与える。これは、底面積が$\Delta A_k$で、上面が曲線$z=f(x,y)$で与えられる垂直な柱の体積
を近似したものである。すなわち、\eqref{6.2}の積和は、局面下の体積を近似している。こうして、この極限値である二重積分\eqref{6.3}は、
局面$z=f(x,y)$、底面$R$、$R$の周上に建てた垂直面、がつくる領域の体積に等しいことがわかる。
\subsection{三重積分}
二重積分を導入したのと同じようにして、三重積分が定義される。三次元の領域$R$で連続な関数$f(x,y,z)$を考える。
領域$R$を、各体積が$\Delta V_1,\Delta V_2,\cdots,\Delta V_n$である$n$個の小区間$R_1,R_2,\cdots,R_n$に分割する。
小領域$R_k$内に点$P_k(\zeta_k,\eta_k,\xi_k)$をとり、積和
\begin{equation}
    \sum_{k=1}^{n}f(P_k)\Delta V_k=\sum_{k=1}^{n}f(\zeta_k,\eta_k,\xi_k)\Delta V_k\label{6.5}
\end{equation}
を作る。各小領域$R_k$の直径を0にするように、分割の数$n$を大きくする。その極限値を
\begin{equation}
    \iiint_R f(x,y,z)dV=\lim_{n\to\infty}\sum_{k=1}^{n}f(\zeta_k,\eta_k,\xi_k)\Delta V_k\label{6.6}
\end{equation}
と書き、領域$R$での関数$f(x,y,z)$の\textbf{三重積分}という。特に$f(x,y,z)=1$ならば、
三重積分は領域$R$の体積$V$を与える。すなわち、
\begin{equation}
    V=\iiint_R dV\label{6.7}
\end{equation}

同様にして、$n$次元での領域$R$で連続な関数$f(x_1,x_2,\cdots,x_n)$に対して、$n$重積分が定義される。

二重積分や三重積分を、\eqref{6.3}や\eqref{6.6}式のように積和の極限として計算するのはすごく面倒だ。
そこで実際には定積分と同様に、もっと楽に計算をする。そのための方法を次節で紹介する。

\section{二重積分は積分を2度行う}
\subsection{累次積分}
まず二重積分について考える。積分領域$R$の境界は、$x$軸または$y$軸に平行な線と3回以上交わらないとしよう。(下図参照)
\begin{figure}[h]
    \centering
    \includegraphics[width=10cm,scale=5]{img/累次積分.png}
    \caption{累次積分}
\end{figure}

より複雑な形の領域は、このような領域に分けられる。曲線ACBを$y=g(x)$,曲線BDAを$y=h(x)$
で表す。

区間$a\leq x\leq b$を、長さが$\Delta x_1,\Delta x_2,\cdots,\Delta x_m$の$m$個の小区間
$h_1,h_2.\cdots,h_m$に分割する。また、区間$c\leq y\leq d$を、長さが$\Delta y_1,\Delta y_2,\cdots,\Delta y_n$の$n$個の小区間
$k_1,k_2,\cdots,k_n$に分割する。そして、直線$x=x_1,x=x_2,\cdots,x=x_{m+1},y=y_1,y=y_2,\cdots,y=y_{n+1}$を引くと、領域$R$は
各面積が$\Delta x_i\Delta y_j$の長方形の領域$R_{ij}$に分割される。境界の近くでは長方形にはならないが、$m,n$が大きい限り、それらの寄与は無視できる。
領域$R_{ij}(i=1,2,\cdots,m,j=1,2,\cdots,n)$は全部で$mn$個ある。領域$R_{ij}$内に点$(\xi_i,\eta_j)$を選び、積和
\begin{equation}
    \sum_{i=1}^{m}\sum_{j=1}^{n}f(\xi_i,\eta_j)\Delta x_i\Delta y_j\label{6.8}
\end{equation}
を作る。

これは、\eqref{6.2}式の特別な場合であることに注意しよう。各区間の長さが0になるように、$m,n$を大きくすれば、
二重積分\eqref{6.3}式に等しい。

いま、最初に区間$h_i$を固定し、その上に並んだ長方形をおいて、積和
\begin{equation}
    \left\{\sum_{j=1}^{n}f(\xi_i,\eta_j)\Delta y_j\right\}\Delta x_i\quad(i\text{は固定})\label{6.9}
\end{equation}
を考える。そして、$y$軸方向の区間$k_j$の長さが0になるように$n\to\infty$の極限を取ると、
\begin{equation}
    \lim_{n\to\infty} \left\{\sum_{j=1}^{n}f(\xi_i,\eta_j)\Delta y_j\right\}\Delta x_i=\left\{\int_{g(\xi_i)}^{h(\xi_i)}f(\xi_i,\eta_j)dy\right\}\Delta x_i\label{6.10}
\end{equation}
を得る。次に、x軸方向の区間に対して和を取り、$m\to\infty$の極限を取ると
\begin{equation}
    \lim_{m\to\infty}\sum_{i=1}^{m}\left\{\int_{g(\xi_i)}^{h(\xi_i)}f(\xi_i,\eta_j)dy\right\}\Delta x_i=\int_a^b\left[\int_{g(\xi_i)}^{h(\xi_i)}f(\xi_i,\eta_j)dy\right]dx\label{6.11}
\end{equation}
が得られる。こうして、二重積分は、$f(x,y)$の$y$についての積分を行い、次に$x$についての積分を行うことで計算できることがわかる。
また、これは$x$積分と$y$積分の順序を入れ替えても同じである。

積分\eqref{6.11}式を\textbf{累次積分}という。

以上まとめるなら、$f(x,y)$が領域$R$で連続ならば、
\begin{align}
    \iint_R f(x,y)dxdy&=\int_a^b\int_{g(x)}^{h(x)}f(x,y)dydx\notag\\
    &=\int_c^d\int_{k(y)}^{l(y)}f(x,y)dxdy\label{6.13}
\end{align}

\subsubsection*{例題}
次の積分を求めよ。
\begin{equation*}
    (1)\int_0^q\int_0^xdydx\qquad(2)\int_0^1\int_1^2(x+y)dxdy
\end{equation*}

\subsubsection*{解答}
\begin{align*}
    &(1)\int_0^1\int_0^xdydx=\int_0^1[y]_0^xdx=\int_0^1xdx=\frac{1}{2}\\
    &(2)\int_0^1\int_1^2(x+y)dxdy=\int_0^1\left[\frac{x^2}{2}+xy\right]_1^2dy=
    \int_0^1(\frac{3}{2}+y)dy=\frac{3}{2}+\frac{1}{2}=2
\end{align*}

\subsubsection*{例題}
直線$y=x,x=1,y=0$で囲まれた領域を$R$とするとき、次の二重積分を求めよ。
\begin{equation*}
    I=\iint_R(x^2+y^2)dxdy
\end{equation*}

\subsubsection*{解答}
最初に、$y$積分、$x$積分の順に計算する。
\begin{equation*}
    I=\int_0^1\left\{\int_0^x (x^2+y^2)dy\right\}dx=\frac{1}{3}
\end{equation*}
次に、$x$積分、$y$積分の順に計算する。
\begin{equation*}
    I=\int_0^1\left\{\int_y^1(x^2+y^2)dx\right\}dy=\frac{1}{3}
\end{equation*}


三重積分の場合にも、同様に累次積分が定義される。
\begin{equation}
    \iiint_R f(x,y,z)dxdydz=\int_a^b\left[\int_{y_1(x)}^{y_2(x)}\left\{\int_{z_1(x,y)}^{z_2(x,y)}f(x,y,z)dz\right\}dy\right]dx\label{6.14}
\end{equation}
上の式では、積分はかっこの内側から順に行う。すなわち$x,y$を固定して、$z$積分を行い、次に$x$を固定して
$y$積分を行い、最後に$x$積分を行う。二重積分と同様に、$f(x,y,z)$が連続ならば、個の積分は存在して、積分の
順序を変えることができる。

\subsubsection*{例題}
\begin{equation*}
    I=\int_0^1\int_0^{1-x}\int_0^{2-y}xyzdzdydx
\end{equation*}
を求めよ。

\subsubsection*{解答}
\begin{align*}
    I&=\int_0^1\int_0^{1-x}\int_0^{2-y}xyzdzdydx\\
     &=\int_0^1\int_0^{1-x}\frac{xy}{2}(2-y)^2dydx\\
     &=\int_0^1\frac{1}{2}x\left\{\frac{1}{4}(1-x)^4-\frac{4}{3}(1-x)^3+2(1-x)^2\right\}dx\\
     &=\int_0^1\frac{1}{2}\left(\frac{1}{4}x^5+\frac{1}{3}x^4-\frac{1}{2}x^3-x^2+\frac{11}{12}x\right)dx=\frac{13}{240}
\end{align*}

\section{積分変数の変換}
\subsection{座標系}
\end{document}