\documentclass[a4j,dvipdfmx]{jsarticle}
\usepackage{amsmath,amssymb}
\usepackage{siunitx}
\begin{document}
\section*{微積分テスト2解答}
\subsection*{はじめに}
微積分テスト2の解答を以下に乗せる。途中式等もできるだけ丁寧に書く。参考にしてほしい。

なお、解答に誤りがあればすぐに教えてください。計算しなおします。
\newpage
\subsection*{大問1}
\begin{align*}
    &(1)\int \frac{dx}{\sin\theta\cos\theta}=\int \frac{dx}{\tan x}+\int \tan xdx=\log|\tan x|+C\\
    &(2) \int_{-1}^{1}\frac{dx}{x^2-4}=\frac{1}{4}(\int_{-1}^1\frac{dx}{x-2}-\int_{-1}^1\frac{dx}{x+2})=\frac{1}{4}[\log|\frac{x-2}{x+2}|]_{-1}^1=\frac{1}{4}(\log|\frac{-1}{3}|-\log|-3|)=\frac{1}{4}\log\frac{1}{9}=-\frac{\log 3}{2}\\
    &(3) \int_1^{\infty}\frac{dx}{x^2}=[-\frac{1}{x}]_1^{\infty}=[0-(-1)]=1\\
    &(4)\int_{-\infty}^{\infty}\frac{dx}{e^x+e^{-x}}=\int_{-\infty}^{\infty}\frac{e^x dx}{e^{2x}+1}=\int_{0}^{\infty}\frac{dt}{1+t^2}=[\arctan t]_{0}^{\infty}=(\frac{\pi}{2}-0)=\frac{\pi}{2}\\
    &(5) \int_a^b \frac{dx}{\sqrt{(x-a)(b-x)}}=\lim_{\epsilon_1 \to +0}\lim_{\epsilon_2 \to +0}\int_{a+\epsilon_1}^{b-\epsilon_2}\frac{dx}{\sqrt{(x-a)(b-x)}}\\
    &\text{ここからは極限の操作を省略する。}\\
    &=\int_a^b\frac{dx}{\sqrt{(\frac{b-a}{2})^2-(x-\frac{a+b}{2})^2}}\quad\text{ここで、$A=\frac{b-a}{2},t=x-\frac{a+b}{2}$とすると}\\
    &=\int_{-A}^A\frac{dt}{\sqrt{A^2-t^2}}=[\arcsin \frac{t}{A}]_{-A}^{A}=(\frac{\pi}{2}-(-\frac{\pi}{2}))=\pi
\end{align*}
式を変形するだけのものが多く、そこまでの難易度ではないだろう。ここくらいまでは解けてほしいところ。

\subsection*{大問2}
すべて$(0<\theta<1)$とする
\begin{align*}
    &(1)e^x=1+x+\frac{x^2}{2!}+\frac{x^3}{3!}+\cdots+\frac{x^n}{n!}+\frac{e^{\theta x}}{(n+1)!}x^{n+1}\quad\\
    &(2)(1+x)^{\alpha}=1+\alpha x+\frac{\alpha(\alpha-1)}{2!}x^2+\cdots+\frac{\alpha(\alpha-1)\cdots(\alpha-n+1)}{n!}x^n+\frac{\alpha(\alpha-1)\cdots(\alpha-n)}{(n+1)!}(1+\theta x)^{\alpha-n-1}x^{n+1}
\end{align*}
最後の一項はどちらも剰余の和である。\\
(2)は$\alpha$が正の整数ならば右辺が有限の値になることに気づこう。よく見てみると二項定理の式である。
\newpage
\subsection*{大問3}
(1) $\displaystyle t=\sqrt{ax^2+bx+c}+\sqrt{a}x$と置換して、$x=$の式で表してみよう。
\begin{align*}
    t&=\sqrt{ax^2+bx+c}+\sqrt{a}x\\
    t^2-2\sqrt{a}tx+ax^2&=ax^2+bx+c\\
    -2\sqrt{a}t\cdot x-bx&=c-t^2\\
    x&=\frac{t^2-c}{2t\sqrt{a}+b}\\
    dx&=\frac{2(\sqrt{a}t^2+bt+\sqrt{a}c)}{(2\sqrt{a}t+b)^2}dt\\
    \sqrt{ax^2+bx+c}&=t-\sqrt{a}\cdot\frac{t^2-c}{2t\sqrt{a}+b}\quad\text{※ここは面倒なのでこれ以上計算をしない}\\
    \therefore\int f(x,\sqrt{ax^2+bx+c})dx&=2\cdot\int f(\frac{t^2-c}{2t\sqrt{a}+b},t-\sqrt{a}\cdot\frac{t^2-c}{2t\sqrt{a}+b})\frac{(\sqrt{a}t^2+bt+\sqrt{a}c)}{(2\sqrt{a}t+b)^2}dt
\end{align*}
よって、$t$についての有理関数に帰着できるので、必ず積分できる。\\

(2)まず、楕円の半円部分(今回はx軸より上)の面積を求める。楕円の面積を$S$とすると、
\begin{equation*}
    \frac{S}{2}=\int_{-a}^a\frac{b}{a}\sqrt{a^2-x^2}dx
\end{equation*}
となる。ここで、$\displaystyle \int_{-a}^{a}\sqrt{a^2-x^2}dx$は半径$a$の半円の面積を表すので、$\frac{1}{2}a^2\pi$となる。よって、
\begin{equation*}
    \frac{S}{2}=\frac{b}{a}\cdot \frac{a^2}{2}\pi\to S=ab\pi
\end{equation*}
よって示された。\\

(3)増加が早いことを示すために、次の極限を取る。
\begin{equation*}
    \lim_{x\to\infty}\frac{e^x}{x^n}
\end{equation*}
ここで、ド・ロピタルの定理を適用して、
\begin{equation*}
    \lim_{x\to\infty}\frac{e^x}{x^n}=\lim_{x\to\infty}\frac{e^x}{nx^{n-1}}\lim_{x\to\infty}=\frac{e^x}{n(n-1)x^{n-2}}=\cdots=\lim_{x\to\infty}\frac{e^x}{n!}=\infty
\end{equation*}
となるため、分子のほうが分母より増加が早い。よって示された。\footnote{同様にして対数関数$\log x$はどんな正のべきよりも増加が遅い。これらは一生覚えておいて損はない。}
\newpage
\subsection*{大問4}
(1)ただ計算すればよいから、
\begin{equation*}
    \int_2^6\frac{dx}{x}dx=[\log(x)]_2^6=\log\frac{6}{2}=\log 3
\end{equation*}
(2)$\displaystyle \frac{67}{60}=1.1167...$\\
(3)$\displaystyle \frac{1.1}{10}=1.1$\\
(4)自然対数の値を覚えてないと答えられないので、ここは全員点が与えられる。\\
一応解答としては、$\log 3=1.0986...$より、シンプソンの公式のほうが近似値として有効である。
\subsection*{大問5}
(1)広義積分であるが、$x=a\sin^2t$と置換して
\begin{align*}
    \int_0^a\frac{x}{\sqrt{ax-x^2}}dx&=\int_0^{\frac{\pi}{2}}\frac{a\sin^2t\cdot 2a\sin t\cos t}{a\sin t \sqrt{1-\sin^2t}}dt\\
    &=2a\int_0^{\frac{\pi}{2}}\sin^2tdt=2a[\frac{1}{2}t-\frac{\sin 2t}{4}]_0^\frac{\pi}{2}\\
    &=2a(\frac{\pi}{4})=\frac{a\pi}{2}
\end{align*}
(2)$f(x)=\sqrt[5]{x}$とすると、$f(33)=f'(32+\theta)+f(32)\quad(0<\theta<1)$である。近似値を求めるので
\begin{equation*}
    \sqrt[5]{33}\thickapprox \frac{1}{5\sqrt[5]{32^4}}+2=\frac{161}{80}=2.0125
\end{equation*}
よって$2.0125$が答えとなる。気になるなら電卓で計算してみよう。
\subsection*{大問6}
よくよく読めば意外と簡単であることがわかる。$\delta$は問題文につられて積分定数を忘れないようにしよう。
\begin{align*}
    (\alpha)&=\cos(bx)+i\sin(bx)\\
    (\beta) &=\frac{e^{ax+ibx}}{a+ib}\\
    (\gamma)&=\frac{a-ib}{a^2+b^2}\\
    (\delta)&=\frac{e^{ax}}{a^2+b^2}(a\cos(bx)+b\sin(bx))+C
\end{align*}
以上
\end{document}