\documentclass[a4paper,10pt]{article}
\usepackage{color}
\usepackage{longtable}
\usepackage[margin=12mm,nohead]{geometry}
\usepackage{hyperref}
\renewcommand{\familydefault}{\sfdefault}
\title{国語試験範囲:問題と答え一覧}
\date{}
\begin{document}
\maketitle
\vspace{-2cm}
\begin{longtable}{|p{0.45\textwidth}|p{0.45\textwidth}|}
\hline
\textbf{問題} & \textbf{答え} \\ \hline
一般的な儒教の認識とは? & \textcolor{red}{孔子 が唱えた道徳・教理を体系化したもの。倫理道徳としてしか理解されず、しかも古い封健的\footnote{一般に、上下関係を重視し、個人の自由や権利を認めないさま}なものという、否定のおまけまでついている...} \\ \hline
加地氏によると、儒教とは何か? & \textcolor{red}{死ならびに死後の説明者} \\ \hline
仏教では本尊を拝むべきだとされる。その理由は? & \textcolor{red}{仏数では、仏を、ひいては法をこそめ拝むのがいちばん大切なことであり、
崇め拝み暮った本尊の広大な恵みや余光を得て、導師に導かれて来世の幸福が得られることを、あるいは成仏することを死者に期待するから。} \\ \hline
儒教(および日本)の葬儀では何を拝む? & \textcolor{red}{死者の柩(ひつぎ)や写真、位牌} \\ \hline
殯とは何か?読み方は? & \textcolor{red}{遺体を死から葬るまでの間自宅などに安置しておくこと。読みは「もがり」} \\ \hline
死から殯の儀式を経て、遺体を地中に葬る。この一連の儀式全体を何と呼ぶ? & \textcolor{red}{喪(そう)} \\ \hline
中国人に納得のゆく死の説明をした集団は? & \textcolor{red}{儒} \\ \hline
人間を精神と肉体に分ける儒教の考え方では、両者が一致している(一致しない)ときはどういう状態? & \textcolor{red}{生きている(死んでいる)状態} \\ \hline
霊を現世に再生させる仕事をする宗教者を何と呼ぶ? & \textcolor{red}{シャーマン} \\ \hline
儒が招魂儀礼を基礎にして築いた一大理論体系とは? & \textcolor{red}{〈儒〉は祖先祭祀・父母敬愛・子孫産出の三事をまとめて〈孝〉と呼称し、それを行うことによって自己を含めた一族が永遠の生命(現世との永久的な関係)を獲得できると説明した} \\ \hline
ドグマとは何? & \textcolor{red}{非科学的な独断と偏見} \\ \hline
加地氏の宗教の定義とは? & \textcolor{red}{死と死後の説明を行うもの} \\ \hline
君子儒と小人儒の違いは? & \textcolor{red}{君子需:礼教性を中心。王朝の祭祀儀礼を担当。(知識人上層) 主知的。小人需は宗教性を中心。祈祷や葬祭を担当。(シャーマン系下層) 主情的。\footnote{一般の理解 君子儒:志をもって知を社会に役立てる者、小人儒:志を持たず知を現世利益に用いる者}} \\ \hline
一般的な〈仁〉と原儒における〈仁〉の違いは? & \textcolor{red}{一般的な仁:積極的な愛、原儒の仁:顔色を伺う/媚びを売る} \\ \hline
ルサンチマンとは? & \textcolor{red}{弱者が強者に抱く恨み・嫉妬心} \\ \hline
経世の才とは? & \textcolor{red}{世の中を治める才能} \\ \hline
子曰「苟有用我者、期月而已可也。三年有成。」の現代語訳は? & \textcolor{red}{もし私を用いて政治をやらせてくれる国があったら、一年で一通りのことはできます。三年あれば完璧に仕上げられます。} \\ \hline
儒教にとって最も厄介な書物は? & \textcolor{red}{『論語』} \\ \hline
儒教にとって最も困った人物は? & \textcolor{red}{孔子} \\ \hline
上二つについて それはなぜ? & \textcolor{red}{孔子の偉大さを宣伝したいが、論語の中の孔子は王者や天子との接点が見いだせず悲哀に満ちており、一方で孔子を知るうえでその言行録である論語を見捨てるわけにはいかないから。}\\ \hline
人を見て法を説くとは? & \textcolor{red}{ある目的を達成するために、相手の能力や性格に応じて説明の仕方を変えること} \\ \hline
素王とは? & \textcolor{red}{王の位は得てないが, 王の素質・素養を備えた人} \\ \hline
「身體髮膚、受之父母、不敢毀傷、孝之始也」の現代語訳は? & \textcolor{red}{父母から授かった身体・皮膚・髪の毛を大事にするのが、「孝」の第一歩である} \\ \hline
仏教では修行者に何を求める? & \textcolor{red}{剃髪(ていはつ)} \\ \hline
仏教者は儒教の「\textcolor{red}{不孝}」批判をどう切り抜けた? & \textcolor{red}{\underline{剃髪しても最終的には「孝」を実現できる(方便)}//悟りを開く修行の一つ/「剃髪や妻子を捨てるのは確かに孝に反するが、この孝は『孝の始め』であって、権道\footnote{手段 は正しくないが、目的は正しい道にかなうこと。目的をとげるために執る便宜の(不正な)手段 。}である。
/その後、出家者として名を高めれば、父母を顕彰することになり、『孝の終り』を全うすることができる} \\ \hline
\end{longtable}

\end{document}
