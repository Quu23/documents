\documentclass[a4paper,10pt]{article}
\usepackage{color}
\usepackage{longtable}
\usepackage[margin=1in]{geometry}
\usepackage{hyperref}
\renewcommand{\familydefault}{\sfdefault}
\title{国語試験範囲:問題と答え一覧}
\date{}
\begin{document}
\maketitle

\begin{longtable}{|p{0.45\textwidth}|p{0.45\textwidth}|}
\hline
\textbf{問題} & \textbf{答え} \\ \hline
一般的な儒教の認識とは? & \textcolor{red}{道徳的規範(道徳の教え、礼節の教え)} \\ \hline
加地氏によると、儒教とは何か? & \textcolor{red}{宗教(死と死後の説明を行う)} \\ \hline
仏教では本尊を拝むべきだとされる。その理由は? & \textcolor{red}{仏数では、仏を、ひいては法をこそめ拝むのがいちばん大切なことであり、
崇め拝み暮った本尊の広大な恵みや余光を得て、導師に導かれて来世の幸福が得られることを、あるいは成仏することを死者に期待するから。} \\ \hline
儒教(および日本)の葬儀では何を拝む? & \textcolor{red}{死者の柩(ひつぎ)や写真、位牌} \\ \hline
殯とは何か?読み方は? & \textcolor{red}{遺体を一定期間自宅などに安置しておくこと。読みは「もがり」} \\ \hline
死から殯の儀式を経て、遺体を地中に葬る。この一連の儀式全体を何と呼ぶ? & \textcolor{red}{喪(そう)} \\ \hline
中国人に納得のゆく死の説明をした集団は? & \textcolor{red}{儒} \\ \hline
人間を精神と肉体に分ける儒教の考え方では、両者が一致しているときはどういう状態? & \textcolor{red}{生きている状態} \\ \hline
霊を現世に再生させる仕事をする宗教者を何と呼ぶ? & \textcolor{red}{シャーマン} \\ \hline
儒が招魂儀礼を基礎にして築いた一大理論体系とは? & \textcolor{red}{儒教} \\ \hline
ドグマとは何? & \textcolor{red}{科学的根拠をもたない教義} \\ \hline
加地氏の宗教の定義とは? & \textcolor{red}{死と死後の説明を行うもの} \\ \hline
君子儒と小人儒の違いは? & \textcolor{red}{君子儒:志をもって知を社会に役立てる者、小人儒:志を持たず知を現世利益に用いる者} \\ \hline
一般的な〈仁〉と原儒における〈仁〉の違いは? & \textcolor{red}{一般的な仁:積極的な愛、原儒の仁:顔色をうかがう媚び} \\ \hline
ルサンチマンとは? & \textcolor{red}{弱者が強者に抱く恨み・嫉妬} \\ \hline
経世の才とは? & \textcolor{red}{国を治める能力} \\ \hline
子曰「苟有用我者、期月而已可也。三年有成。」の現代語訳は? & \textcolor{red}{もし私を登用してくれる者がいれば、一年で成果を出し、三年で完璧に仕上げることができる} \\ \hline
儒教にとって最も厄介な書物は? & \textcolor{red}{『論語』} \\ \hline
儒教にとって最も困った人物は? & \textcolor{red}{孔子} \\ \hline
人を見て法を説くとは? & \textcolor{red}{相手に応じて言うことを変えること} \\ \hline
素王とは? & \textcolor{red}{王にはなっていないが、王の器を備えた人物} \\ \hline
「身體髮膚、受之父母、不敢毀傷、孝之始也」の現代語訳は? & \textcolor{red}{体や髪の毛は親から授かったものであり、それを傷つけないことが孝の始まりである} \\ \hline
仏教では修行者に何を求める? & \textcolor{red}{剃髪} \\ \hline
仏教者は儒教の「不孝」批判をどう切り抜けた? & \textcolor{red}{「剃髪も最終的には孝に通じる」と主張した(方便)} \\ \hline
\end{longtable}

\end{document}
