\documentclass[a4j,dvipdfmx]{jsarticle}
\usepackage{amsmath,amssymb}
\usepackage{siunitx}
\usepackage{ascmac}
\usepackage{bm}
\usepackage{url}
\usepackage[dvipdfmx]{graphicx}

\begin{document}
    解答するときは, 数式と下線部の内容程度のことがかけてればよい. そのため, \textbf{定理の細かい条件などは書く必要はない.}
    あくまで自分の言葉でどんな定理かを答える(何が便利かに重点を置いている).
    \begin{itembox}{グリーンの定理}
        \color{red}
        単純閉曲線$C$によって囲まれた範囲を$D$とする. このとき, 関数$F(x,y),G(x,y)$が$C$と$D$
        を含む領域で連続な偏導関数を持つならば
        \begin{equation}
            \int_C (Fdx+Gdy)=\iint_D \left(\frac{\partial G}{\partial x}-\frac{\partial F}{\partial y}\right)dxdy
        \end{equation}
        \underline{線積分を重積分に直せる. これは, 例えば被積分関数を微分することで簡単になる場合に有効.}
    \end{itembox}
    \begin{itembox}{発散定理}
        \color{red}
        閉曲面$S$で囲まれた立体$V$があり, $S$の単位法線ベクトル$\bm{n}$は$S$の外側を向くとする. $V$を
        含むある範囲でベクトル場$\bm{a}$とその偏導関数が連続であるとき, 次の等式が成り立つ.
        \begin{equation}
            \int_S \bm{a}\cdot\bm{n}dS=\int_V \nabla\cdot\bm{a}dV
        \end{equation}
        \underline{面積分を発散の体積分に直せる.}
    \end{itembox}
    \begin{itembox}{ストークスの定理}
        \color{red}
        曲面$S$, 単純閉曲線$C$,  $S$の単位法線ベクトル$\bm{n}$の向きを次のように定める. 
        $C$を曲面$S$の境界とし, $S$の単位法線ベクトル$\bm{n}$の向きは, $S$上で$n$が連続的に変わるようにとる.
        $\bm{n}$の向く側を$S$の正の側ということにし, $C$の向きは, $S$の正の側に立ち$C$に沿って進むとき$S$が
        常に左側にあるようにする.
        このとき, $S$を含むある範囲でベクトル場$\bm{a}$とその偏導関数が連続であるとき, 次の等式が成り立つ.
        \begin{equation}
            \int_C \bm{a}\cdot d\bm{r}=\int_S (\nabla\times\bm{a})\cdot \bm{n}dS
        \end{equation}
        {\footnotesize 図の参考: \url{https://hooktail.sub.jp/vectoranalysis/StokesTheorem/}}\\
        \underline{回転の面積分を体積分に直せる.}
    \end{itembox}
    補足. 定理内で, 連続な偏導関数をもつという条件は, 積分がうまく定義できるために必要. (不連続な
    (偏導)関数を積分するなんて難しい!) なお, 偏導関数が存在し, 連続であれば, 元の関数は全微分可能であるから
    当然連続である.
\end{document}