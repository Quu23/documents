\documentclass[a4j,dvipdfmx]{jsarticle}
\usepackage{amsmath,amssymb}
\usepackage{siunitx}
\usepackage{bm}

\usepackage[margin=15truemm,nohead]{geometry}
\usepackage{qexam}

\begin{document}
    \part*{理学同好会 ベクトル解析 テスト1}
    \question{問1}
        ベクトルの基礎的な計算について考える。以下の問いに答えよ。
        \begin{qparts}
            \qpart まずは平面ベクトルで計算してみよう。基本ベクトルを$\bm{i}=[1,0],\bm{j}=[0,1]$とする。
            \begin{qlist}
                \qitem ベクトル$\bm{A}=[2,5]$を基本ベクトルの線形結合で表せ。また、大きさを求めよ。
                \qitem ベクトル$\bm{B}=[-3,-8]$を正規化せよ。また、$\bm{A}+\bm{B}$を正規化せよ。
                \qitem 内積$\bm{A}\cdot \bm{B}$を求めよ。
                \qitem ベクトル$\bm{A},\bm{B}$のなす角度を求めよ。
            \end{qlist}
            \qpart 次に空間ベクトルでも計算してみよう。基本ベクトルを$\bm{i}=[1,0,0],\bm{j}=[0,1,0],\bm{k}=[0,0,1]$とする。
            \begin{qlist}
                \qitem ベクトル$\bm{A}=\bm{i}+2\bm{j}+3\bm{k}$と同じ向きの単位ベクトルの成分を求めよ。また、$\bm{A}$の方向余弦も求めよ。\label{q:1ii5}
                \qitem 二つのベクトル$\bm{B}=[B_x,B_y,B_z],\bm{C}=[C_x,C_y,C_z]$について、$|\bm{B}-\bm{C}|$を求めよ。
                \qitem ベクトル$\bm{D}=[3,1,2]$を正規化し、問\qref{q:1ii5}で求めた単位ベクトルとの内積を求めよ。
                \qitem 外積$\bm{A}\times \bm{D}$を求めよ。
            \end{qlist}
            \qpart 以下ではベクトルはすべて空間ベクトルを指すものとする。
            \begin{qlist}
                \qitem ベクトル$\bm{a}\times \bm{b}$が$\bm{a}$に直行することを計算によって確かめよ。
                \qitem ベクトル三重積の公式
                \begin{equation}
                    \bm{a}\times(\bm{b}\times\bm{c})=\bm{b}(\bm{c}\cdot\bm{a})-\bm{c}(\bm{a}\cdot\bm{b})
                \end{equation}
                を外積のEinsteinの規約(もどき)による表示から示せ。(すなわち$(\bm{A}\times\bm{B})_i=\varepsilon_{ijk}\bm{A}_j\bm{B}_k$\footnote{授業中は述べなかったが、本来Levi-Civita記号は下付きの添え字である。そうすると今回のように外積は$(\bm{a}\times\bm{b})_i=\varepsilon_{ijk}a_jb_k$となる。
                しかしこの場合、(本来の)Einsteinの規約に外れてしまう。そこで、見た目的にわかりやすくするため授業等では$\varepsilon^{ijk}$と書いた次第である。(やはり相対論屋さんからすると気持ちが悪いらしい。)}から示せ。)
                \qitem 公式
                \begin{equation}
                    (\bm{a}\times\bm{b})\cdot(\bm{c}\times\bm{d}) = (\bm{a}\cdot\bm{c})(\bm{b}\cdot\bm{d})-(\bm{a}\cdot\bm{d})(\bm{b}\cdot\bm{c})
                \end{equation}
                を証明せよ(Lagrange)。
            \end{qlist}
        \end{qparts}

    \question{問2}
        ベクトル値関数の微分演算について考える。以下の問いに答えよ。
        \begin{qparts}
            \qpart 単純な計算をしてみる。
            \begin{qlist}
                \qitem 楕円$\bm{r}=[a\cos t,b\sin t]$を$t$で微分せよ。
                \qitem 放物線$\bm{r}=[t,t^2]$を$t$で微分せよ。
                \qitem 螺旋$\bm{r}=[\cos t,\sin t,t]$を$t$で微分せよ。
            \end{qlist}
            \qpart 以下単位ベクトル$\bm{e}(t)$の$t$微分を$\bm{e}'$で表す。次の問いに答えよ。
            \begin{qlist}
                \qitem $\bm{e}\cdot\bm{e}'=0$を示せ。
                \qitem $|\bm{e}\times\bm{e}'|=|\bm{e}'|$を示せ。
                \qitem $\bm{e}\cdot(\bm{e}\times\bm{e}')=0$を示せ。
            \end{qlist}
            \qpart 物体が回転運動を行うとき、物体の位置を$\bm{r}(t)=[x(t),y(t),z(t)]$とすると、その速度は$\bm{\omega}\times\bm{r}$で与えられる。ここで$\bm{\omega}$は
            角速度ベクトルで、回転軸の方向を向き、大きさが回転の角速度の大きさと一致する。回転軸$\bm{\omega}$をz軸に取って、$|\bm{\omega}|=\omega=一定$であるとき、
            これが等速円運動であることを示せ。
        \end{qparts}
    \clearpage
    \question{問3}
        以下では曲線について曲率等のパラメータを求めてみる。
        \begin{qparts}
            \qpart 平面曲線について考えてみる。
            \begin{qlist}
                \qitem 放物線$\displaystyle y=\frac{1}{2}ax^2$の曲率を求めよ。\vspace{0.5mm}
                \qitem 楕円$\displaystyle \frac{x^2}{a^2}+\frac{y^2}{b^2}=1$の$x=a,y=0$における曲率半径を求めよ。また、$x=0,y=b$の曲率半径を求めよ。
            \end{qlist}
            \qpart 空間曲線についても計算してみよう。
            地球を完全な球体だと仮定したとき、北緯45度の緯線の曲率、捩率を求めよ。ただし、地球の半径は$R_E=6.4\times10^6[\si{\meter}]$とする。
        \end{qparts}
    \hrulefill  以上 \hrulefill
    \begin{table}[h]
        \centering
        \begin{tabular}{|c|c|c||c|}\hline
              /30 &   /30 &   /40 &   /100\\\hline
        \end{tabular}
    \end{table}
\end{document}