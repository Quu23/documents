\documentclass[a4j,dvipdfmx]{jsarticle}
\usepackage{amsmath,amssymb}
\usepackage{siunitx}
\usepackage{ascmac}
\usepackage{amsthm}

\usepackage[margin=15truemm,nohead]{geometry}

\newcommand{\R}{\mathbb{R}}
\newcommand{\M}{\mathbb{C}}

\newcommand{\Q}{\textbf{問}. }

\newcommand{\Ker}{{\rm{Ker}}\hspace{0.5mm}}
\renewcommand{\Im}{{\rm{Im}}\hspace{0.5mm}}

\theoremstyle{definition}
\newtheorem{definition}{定義}

\theoremstyle{definition}
\newtheorem{theorem}{定理}

\theoremstyle{definition}
\newtheorem{proposition}{命題}


\begin{document}
    \section*{代数}
        教科書 pp.63-86. 定理番号などは本の番号と\textbf{一致しない}ことに注意.
        \subsection*{基本事項の確認}
            \begin{definition}[全称記号と存在記号]
                $\forall x$は「任意の$x$」や「全ての$x$」の意. $\exists x$は「ある$x$が存在して」の意.
            \end{definition}
            \begin{definition}[基本的な集合の表記]集合とはものの集まり. なお以下, 有理数については略.
                \begin{align*}
                    \mathbb{N}&=\text{自然数全体の集合}\\
                    \mathbb{Z}&=\text{整数全体の集合}\\
                    \mathbb{R}&=\text{実数全体の集合}\\
                    \mathbb{C}&=\text{複素数全体の集合}
                \end{align*}
            \end{definition}

            \begin{definition}[群の定義]
                集合$G$が群というのは, $G$の任意の元に対して演算が定まり, 結合則が満たされ, 単位元と逆元が存在すること.
            \end{definition}

            \begin{definition}[部分群の定義]
                群$G$の部分集合$H\subset G$が$G$の部分群であるとは, $G$の演算で$H$も群となるということ. すなわち, $H$の元が演算に閉じていて, 逆元が$H$内に存在していればよい. 
            \end{definition}
        \subsection*{右剰余類分解}
            \begin{itembox}[l]{$Review$}
                集合$A$の同値関係$\sim$とは, $a,b,c\in A$に対して
                \begin{itemize}
                    \item $a\sim a$
                    \item $a\sim b \rightarrow b\sim a$
                    \item $a\sim b \land b\sim c \rightarrow a\sim c$
                \end{itemize}
                を満たす二項関係のこと. これによって, $A$を直和分割できるのだった.
            \end{itembox}
            $G$を群, $H$を部分群とする. $a,b\in G$に対し, $a\sim' b\Leftrightarrow b\cdot a^{-1}\in H$と定めると, $\sim'$は同値関係だから, 
            同値類が考えられる. この同値類を$H$に関する右剰余類という. したがって, $G$の$H$に関する右剰余類分解は
            \begin{equation}
                G=\coprod_{\lambda'\in\Lambda'} H_{a_{\lambda'}}
            \end{equation}
            となる.
            \begin{itembox}[l]{$Review$}
                $G$を群, $N$を$G$の部分群とする. このとき, $N$が正規部分群であるというのは, $\forall x\in G, xNx^{-1}=N$.\\
                また, $N$が正規部分群であることと``同値''な命題は
                \begin{itemize}
                    \item $xN=Nx$
                    \item $\forall x\in G, xNx^{-1}\subset N$
                \end{itemize}
                であった.
            \end{itembox}
            $N$が正規部分群であれば, その右剰余類と左剰余類は一致する. よって, $G$の分解も等しくなる.
        \clearpage
        \subsection*{剰余群}
            \begin{proposition}
                $G$を群, $N$を$G$の正規部分群とする. $A,B$が$N$に関する左剰余類なら, $AB$も$N$に関する左剰余類.
            \end{proposition}

            \begin{theorem}
                $G$を群, $N$を$G$の正規部分群とする. このとき$G/N$は$\cdot:G/N\times G/N\rightarrow G/N;(A,B)\mapsto AB$を演算とする群の構造を持つ.
            \end{theorem}

            \begin{definition}
                上のような群を剰余群(商群)という.
            \end{definition}

        \subsection*{群の準同型写像, 群の同型}
            \begin{itembox}[l]{$Review$}
                $f:A\to B$が$A$から$B$への写像というのは, $A$の任意の元に対して, $B$の元が``一意に''定まっている対応関係のことをいう.
            \end{itembox}

            \begin{definition}[準同型写像]
                $G,G'$を群とする. $f:G\to G'$が群の準同型写像であるとは, 
                \begin{equation}
                    \forall x,y\in G,f(x\cdot y)=f(x)\cdot f(y)
                \end{equation}
                が成り立つときにいう.
            \end{definition}

            \begin{proposition}
                $f:G\to G'$は群の準同型写像.
                \begin{itemize}
                    \item $f(e)=e'$
                    \item $\forall x\in G,f(x^{-1})=(f(x))^{-1}$
                \end{itemize}
            \end{proposition}

            \begin{itembox}[l]{$Review$}
                写像$f:A\to B$が全単射であるとは, 次の二つの条件を満たすこと.
                \begin{itemize}
                    \item $x\neq y\rightarrow f(x)\neq f(y)$ (\textbf{単射})
                    \item $f(A)=B$ (\textbf{全射})
                \end{itemize}
            \end{itembox}

            \begin{definition}[同型写像, 同型]
                $G,G'$を群とする.
                \begin{enumerate}
                    \item $f:G\to G'$が準同型写像で, 全単射であるとき, $f$を同型写像という.
                    \item $G,G'$の間に同型写像が存在するとき, $G$と$G'$は同型といい, $G\simeq G'$とかく.
                \end{enumerate}
            \end{definition}

            \begin{itembox}[l]{$Review$}
                群$G$がAbel群であるとは, その演算が可換(交換可能)であること.\\
                $G$の``元''$x$に対して, 位数とは$x^{k}=e$なる$k$のうち最小のもの.
            \end{itembox}

            \begin{proposition}
                $G\simeq G'$とし, 同型写像を$f:G\to G'$とする.
                \begin{itemize}
                    \item $|G|=|G'|$
                    \item $G$がAbel群$\Rightarrow$ $G'$もAbel群
                    \item $x\in G$が位数$n$の元 $\Rightarrow$ $f(x)$は$G'$の位数$n$の元
                \end{itemize}
            \end{proposition}
        \clearpage

        \subsection*{群の準同型定理}
            \begin{definition}[核と像]
                $f:G\to G'$を準同型写像とする.
                \begin{itemize}
                    \item $f$の核(Kernel) $\Ker f:=\{x\in G\mid f(x)=e'\}$
                    \item $f$の像(Image) $\Im f:=\{f(x)\mid x\in G\}$
                \end{itemize}
                と定める.\footnote{集合論の記号を借りれば, $\Ker f=f^{-1}(\{e'\}),\Im f=f(G)$に過ぎないとわかる.}
            \end{definition}

            \begin{proposition} $f:G\to G'$は準同型写像.
                \begin{itemize}
                    \item $\Ker f$は$G$の正規部分群
                    \item $\Im f$は$G'$の部分群
                    \item $f$が単射$\Leftrightarrow$ $\Ker f=\{e\}$
                \end{itemize}
            \end{proposition}

            \begin{theorem}[\textbf{準同型定理}]
                $f:G\to G'$を群の準同型写像とする. このとき, $\Ker f$の剰余類$x\Ker f \hspace{0.5mm}(x\in G)$に$G'$の元$f(x)$を
                対応させることで, 群の``同型''写像
                \begin{equation}
                    \overline{f}:G/\Ker f \to \Im f
                \end{equation}
                が得られる.
            \end{theorem}

            \begin{itembox}[l]{$Review$}
                $G$の位数とは$G$の要素数のこと.\\
                $G$が巡回群とは, $\exists a\in G,G=\langle a \rangle(=\{a^{k};k\in\mathbb{Z}\})$.\\
                $\mathbb{Z}/n\mathbb{Z}$は, 整数を$n$で割った余りで作った同値類全体.
            \end{itembox}
            
            \begin{proposition}
                $G$が位数$n$の巡回群とする. このとき, $G\simeq \mathbb{Z}/n\mathbb{Z}$.
            \end{proposition}
\end{document}