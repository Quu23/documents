\documentclass[a4j,dvipdfmx]{jsarticle}

\usepackage[dvipdfmx]{graphicx}
\usepackage{amsmath,amssymb}
\usepackage{siunitx}
\usepackage{ascmac}
\usepackage[subrefformat=parens]{subcaption}
\usepackage{fancyhdr}
\usepackage{otf}
\usepackage[dvipdfmx]{hyperref}
\usepackage{pxjahyper}
\usepackage{okumacro}
\usepackage{tikz}
\usepackage{bm}
\usepackage{ulem}

\usepackage{titlesec}
\usepackage{tocloft} % 目次の設定をカスタマイズするためのパッケージ

\pagestyle{headings}

% \renewcommand{\thesubsection}{\arabic{subsection}}
\renewcommand{\headrulewidth}{1pt}
\renewcommand{\Re}{\operatorname{Re}}
\renewcommand{\Im}{\operatorname{Im}}

% 目次にも「\S」を反映
\renewcommand{\thesection}{\S\arabic{section}}

% tocの番号幅を調整(「\S」で番号が長くなるため)
\setlength{\cftsecnumwidth}{3.5em} % セクション番号の幅を調整



\newcounter{basic_quastion}\setcounter{basic_quastion}{1}
\newcommand{\basicquestion}{\noindent{\large 基本問題\hspace{1mm}\huge\fbox{\textbf{\arabic{basic_quastion}}}\addtocounter{basic_quastion}{1}}\thispagestyle{fancy}\lhead{$\Sigma$基本問題}\rhead{\thepage}\cfoot{}\quad}
\newcommand{\sign}{\mathop{\mathrm{sign}}\nolimits}
\newcommand{\linktoMOKUZI}{\vspace{\stretch{1}}\fbox{\centerline{\hyperref[目次]{目次に戻る}}}}

\newcounter{basic_answer}\setcounter{basic_answer}{1}
\newcommand{\basicanswer}{\noindent{\large 基本問題\hspace{1mm}\huge\fbox{\textbf{\arabic{basic_answer}}}\addtocounter{basic_answer}{1}}\thispagestyle{fancy}\lhead{基本問題解答}\rhead{\thepage}\cfoot{}\quad}

\title{Quuノート ー微分積分\ajRoman{2}ー}
\date{最終更新 2025/01/07}
\author{責任者 Quu}

\begin{document}
    \maketitle
        \thispagestyle{empty}
    \begin{figure}[h]
        \centering
        \includegraphics[scale=0.5]{img/QuuNote/QuuNote2/icon.png}
    \end{figure}
    
    \vspace{\stretch{1}}
    \centerline{\textbf{概要}}
    \noindent
    微分積分学入門についてのノート。\\
    主に、多変数微分積分、ベクトル解析、複素解析について扱う。
    \clearpage
     
    \clearpage
    \part*{このノートを読む前に}
        この本の読み方とか, この本が扱う内容について書く予定. 今回から英語人名にすることとか, 「.」や「,」を使うことも触れる.
    \clearpage
    \label{目次}
    \tableofcontents
    \clearpage

    \part{基礎数学}
    \vspace{\stretch{1}}
    \begin{screen}
        ここでは, 数学をするうえで必要となる最低限の基礎知識を学ぶ. 主に, 集合論基礎, 線形代数のうちベクトル, 行列, 行列式の基礎が含まれる.
        また, 多変数関数について微分・積分を定義する. いわゆる偏微分, 重積分というもので, これらの概念を習得することは, 数学, 物理, 工学を学ぶ上で重要である.
    \end{screen}
    \clearpage
    \section{集合論基礎}
        集合とその演算, 写像, 濃度について軽く触れ, 実数論についても少し触れる.
    \clearpage
    \section{行列と行列式}
        ベクトルのイメージとその和・差について. 内積についても述べる. 外積はベクトル解析のときに述べる. 続いて, 行列についてその定義を述べ, 和・差・積について述べる.
        転置行列と逆行列についても述べる. 行列式は, 一般の定義を述べ, その後2x2と3x3について計算方法を述べる. 余因子展開についてその計算方法を述べる.
    \clearpage
    \section{偏微分}
        多変数関数をまず具体例で挙げ, その後偏微分について解説する.二変数テイラー展開及び積分記号下の微分まで述べる.
    \clearpage
    \section{多重積分}
        多重積分についてまず二重積分についてその定義を話す. かんたんな計算ののちに三重積分も述べる. その後, 変数返還について, 一次変換の場合について厳密な証明を行い, 
        それ以外は感覚的なものにとどめる.
    \clearpage
    
    \part{ベクトル解析}
    \vspace{\stretch{1}}
    \begin{screen}
        ベクトル解析は, 理工系の学生にとって馴染み深いものと聞く. たいていの物理科と電気科の学生は, 電磁気学にて顔を合わせることになるだろう.
        よく電磁気学は, ベクトル解析をふんだんに用いるから難しいと言われているが, 少なくともベクトル解析単体で見ればそこまで難しいものではない. 
        そして何よりベクトル解析は楽しいものである. ここでは, まずベクトルについての基礎知識について述べた後, ベクトル値関数についてその微分積分を定義する.
        その後, 空間上の曲線および曲面の解析についてすこし述べ, ベクトル解析の顔ともいえる微分演算子について述べる.
        電磁気学ではよく用いられるGaussの発散定理およびStokesの定理についても扱う. 
    \end{screen}
    \clearpage
    \section{ベクトルの性質と演算}
        数学基礎で述べたことと多少重複するが, 内積, 外積およびそれらの性質を述べる. スカラー三重積とベクトル三重積も述べる.
    \clearpage
    \section{ベクトル値関数とその微分}
        ベクトル値関数について述べたのち, それらの微分積分を定義する. ただし, 積分は定義のみで深く触れない.
    \clearpage
    \section{曲線の解析}
        曲線について解析する. 平面曲線について述べて, 空間曲線でも議論する. Frenet-Serretの公式まで.
    \clearpage
    \section{曲面論入門}
        曲面について解析する. 基本量を導出し, Gauss曲率と平均曲率を紹介する.
    \clearpage
    \section{微分演算子}
        ベクトル場とスカラー場について説明する. その後各微分演算子について述べる.
    \clearpage
    \section{線積分と面積分}
        線積分と面積分について定義を述べる. Gaussの定理とStokesの定理について述べ, 電磁気学への応用をしてみる.
    \clearpage

    \part{複素解析}
    \vspace{\stretch{1}}
    \begin{screen}
        複素解析は, 数学の中で最も美しい理論の一つと言われる. 複素関数(複素数から複素数へ対応させる関数)に対して, 正則という概念が定義できる.
        正則性とはかんたんに言えば微分可能性のことなのであるが, 驚くべきことにこの正則性を満たせば,それらを微分した関数も正則性を保つのである. 
        これらの性質を含め, 正則関数の解析の基本となるのはCauchyの積分定理である. ここでは, 複素数についてその基本的な性質を述べ, 複素関数および
        複素微分について定義し, 複素平面上での積分を述べる. その後, Cauchyの積分定理をはじめとする, 正則関数に関する多くの定理を証明する.
        その中には実積分の計算に対してたいへん有効な留数定理もある. この理論の美しさをじっくり味わってほしい.
    \end{screen}
    \clearpage
    \section{複素数}
        複素数についてその基礎知識を述べる. 複素平面上の領域も. De Moivreの定理まで述べる.
    \clearpage
    \section{複素関数とその微分}
        複素関数について定義し, その性質について簡単に述べる. Cauchy-Riemannの方程式も述べる. 初等関数についても述べる.
    \clearpage
    \section{複素線積分}
        かんたんな線積分の計算をして, Cauchyの積分定理を述べる. Cauchyの積分公式やGoursatの定理などの多数定理を述べる. 最大値の原理
        については証明させる?
    \clearpage
    \section{級数}
        複素数列について, その収束等を定義し, 複素級数についても定義する. ベキ級数やTaylor展開, Maclaurin展開についても述べる.
        Laurent展開を重点的に扱う.特異点とその分類も述べる.Picardの大定理は入れない. 無限遠点のLaurent展開も述べる.
    \clearpage
    \section{留数定理}
        留数について定義を述べて, 留数定理を示す. その後, 実積分の計算を行う. 無限遠での留数についても述べる.
    \clearpage
    \section{解析接続}
        解析接続について簡単な例を挙げ, 一致の定理を証明して, 解析接続の一意性について説明する.
    \clearpage
    \section{Riemann面}
        余裕があれば,多価関数とRiemann面についてすこしだけ述べる.
    \clearpage

    \part{相対性理論}
    \vspace{\stretch{1}}
    \begin{screen}
        相対性理論...それはかの天才物理学者Einsteinが作り上げた物理学の中で最も美しい理論である. 理科や宇宙が好きな
        小学生であればほとんどの人があこがれていたものであると思うし, それ以外の人でも, 相対性理論から導かれる不思議な
        世界(双子のパラドクスなど)について聞いたことがある人も多いと思う. 相対性理論が美しいといわれるその所以は, たった一つ
        の物理的要請に真摯に従って計算することで, 重力場の基礎方程式(いわゆる, Einstein方程式)までたどり着けるところであろう. 
        その要請とは, 「物理学の法則は, 座標系に依存しない形式に書かれなければならない」という, 実に自然な, 当然ともいえる要請である.
        この要請から, さまざまな相対性理論の世界が開けることには, ただただ驚嘆するばかりである.\\
        相対性理論は, 一般に非常に難しいといわれている. 確かに, 相対性理論, 特に一般相対性理論を真に理解するには, 数学のRiemann幾何学について熟知していないといけないだろう.
        しかし, 特殊相対性理論に限って言えば, かんたんな力学の知識さえあれば(一部を除いて)理解することができる. また, 一般相対性理論に関しても, 重力場の方程式を導くだけであれば
        必要なRiemann幾何学の知識も特別難しいものではないのである.\\
        ここでは, 特殊相対性理論について, よく子供向けの科学本などに載っている事象を中心に数学的に理解していく.
        また, 一般相対性理論についても軽く触れる.
    \end{screen}
    \clearpage
    \section{特殊相対論入門}
        Galilei変換についてと慣性系について述べる. 光速度不変の原理について, 実験結果から述べ, Lorentz変換を導出する. 
         Minkowski時空上の距離, 世界間隔について説明する. このときLorentz変換に対して不変であることを述べる.
        固有時間などについても述べる.
    \clearpage
    \section{パラドクスの解決}
        パラドクスをここで解決する. 双子のパラドクスと時計のパラドクス
    \clearpage
    \section{数学的準備}
        ベクトルやテンソルについて, かんたんに定義する.
    \clearpage
    \section{相対論的力学}
        相対論上で力学を展開する. $E=mc^2$の導出や変分原理についても扱う. 双子のパラドクスを変分原理を用いて解決する.
    \clearpage
    \section{一般相対論への展望}
        Riemann幾何学の$ds^2$を紹介して, 等価原理について$\Gamma=0$であることを紹介する.
    \clearpage

    \part{Lebesgue積分入門}
    \vspace{\stretch{1}}
    \begin{screen}
        積分好きなら一度は聞いたことがあるのが, このLebesgue積分である. Lebesgue積分は, 単なる数学の枠を飛び越えて, 
        物理学や工学で必須のFourier解析や, 偏微分方程式, また確率論などのいたるところに顔を出す非常に重要な概念である.
        それにもかかわらず, このLebesgue積分はなかなかに難しく, 習得にも時間がかかる. これはLebesgue積分が素朴なRiemann積分と違って
        内容がいささか抽象的であることが原因であるように思える. さらに, 集合論に関する知識も必要であり, 学ぶための敷居が高い.
        そこでここでは, Lebesgue積分論のうち, 特に重要であるものを選択して系統的に学べるよう, 構成を工夫した. 
        端的に言えば, 極限と積分の順序交換ができる単調収束定理へ最短経路で学べるようになっているのである.
        なおこのLebesgue積分は, Riemann積分と対照的に, しばしばグラフを横に切る積分であると説明される. 実際まちがってはないが, 
        実際にLebesgue積分を学んでいると, 横に切っているというイメージはあまりないので注意してほしい.
    \end{screen}
    \section{いろいろ}
    \clearpage
    
    \part{終わりに}

\end{document}