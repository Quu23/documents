\documentclass[a4j,dvipdfmx]{jsarticle}
\usepackage{amsmath,amssymb}

\usepackage{color}
\usepackage{ulem}
\usepackage{fancyhdr}
\usepackage{ascmac}
\usepackage{siunitx}

\renewcommand{\thesection}{\Roman{section}}
\renewcommand{\thesubsection}{\roman{subsection}}

\begin{document}
\section*{写像ってなんすか}
「写像」...某論破王の影響で言葉事態は知っている人が多いかもしれない。ただ、その意味について考えたことはあるだろうか?
おそらくほとんどの人は「写像」の意味を知らないと思う。そこで今回はこの「写像」について学んでみよう。\xout{そして某論破王を論破しよう!}

\subsection{対応ってなんすか}
数学では、集合と並んで基本的な概念として\textbf{対応}というものがある。その定義を述べると、
\begin{screen}
    $A,B$を二つの集合とし、ある規則$\Gamma$によって$A$の各元$a$にたいしてそれぞれ一つずつ$B$の部分集合$\Gamma(a)$が定められるとする。
    そのとき、その規則$\Gamma(a)$のことを$A$から$B$の\textbf{対応}という。
\end{screen}
さらに、$A$の元$a$にたいして定まる$B$の部分集合$\Gamma(a)$を、$\Gamma$による$a$の像という。また、$A,B$をそれぞれ対応$\Gamma$の
始域(定義域)、終域という。またこのとき、$B$の部分集合のうち同じものがあってもよいし、部分集合が空集合であるような元が存在してもよい。つまり、
$\Gamma(a)=\Gamma(a')\quad(a\neq a')$であってもよいし、$\Gamma(a)=\phi$であるような$a$があってもよい。
ちなみに、$\Gamma$が$A$から$B$の対応であることを$\Gamma:A\to B$と表す。\\

とはいえ、これだけ聞いてもイメージしにくいので現実世界に置き換えて考えてみよう。たとえば、トマラーに友達といった時を考えてみる。
このとき、自分と知り合いを含めた客(Customer)は、お店に対してメニュー(Menu)から`料理を選ぶ'はずだ。
この`料理を選ぶ'ことこそがまさに対応である。客の集合$C$の各元はメニューの集合$M$の元から料理を一つでも複数でも選ぶ。その選んだ品物の集合は$M$の部分集合になっているはずだ。
このとき、客ごとに選んだ料理の品はかぶってもいいし、かぶらなくてもいい。もちろん一つも料理を頼まず水だけ頼む客もいるだろう(ほんとはよくない)。\\

\subsection{写像ってなんすか}
さて、ここから本題の写像について考える。とはいえ写像は対応がとある性質を持つ場合のものであるため、ほとんど対応と同じみたいなものである。
その性質とは、
\begin{screen}
    $A$の任意の元$a$に対して、$f:A\to B$である対応$f$の$f(a)$が$B$のただ一つの元からなる集合。
\end{screen}
である。つまり\textbf{写像}とは\underline{$A$のどんな元に対しても$B$の元を一つに対応させる}規則のことである。\\

先ほどの対応との違いは対応先が一つでしかないところである。そのため対応を説明する際の例は写像でもある。
ほかにも我々になじみのある$f(x)=x^2$は、ある実数の集合$\mathbb{R}$の元$x$に対して実数の集合の元$x^2$が対応している
ので写像である。なお、$f(x)=x^2$のように終域が数字である場合は関数といい、それ以外の場合は写像ということが多い。

反対に、写像ではない例として逆三角関数がある。例えば、$\arcsin x$は$-1\leq x\leq 1$の実数$x$に対して無限個の値が存在する。
(ただ実際は主値を取っている。$\arcsin x$なら$-\frac{\pi}{2}\leq \arcsin x\leq \frac{\pi}{2}$)
\newpage
写像$f:A\to B$によって$A$の元$a$に$B$の元$b$が対応するとき、$b$を$f$による$a$の\textbf{像}といい、
$b=f(a)$と表す。このとき、$a$を$f$による$b$の\textbf{原像}という。

\uwave{中学生でもわかること}だが、関数$f(x)=4x^2$と関数$g(x)=(2x)^2$は$\mathbb{R}$のどんな元$x$についても$f(x)=g(x)$が成り立つ。
同じように、$f:A\to B$である写像$f$と$g:A\to B$である写像$g$が、$A$のどんな元$a$についても常に$f(a)=g(a)$となるとき、
写像として\textbf{等しい}といい$f=g$と表す。もちろん等しくないときは$f\neq g$で表す。

\subsubsection*{問題}
$A,B$がそれぞれ$m$個,$n$個の元からなる有限集合のとき、$A$から$B$への対応は全部でいくつあるか。また、写像はいくつあるか。
{\scriptsize ヒント:まずは$A$の一つの元について考えてみよう。}
\vspace{70mm}
\subsubsection*{解答}
\color{red}
まず$A$の集合の元の一つ$a_1$について考えてみよう。このとき$A$から$B$の対応は$2^n$個ある。
なぜなら、$a_1$に対して$B$の一つの元に対応する場合は$_nC_1$個、二つの元に対応する場合は$_nC_2$個、
三つの元に対応する場合は$_nC_3$個...$n$個の元に対応する場合は$_nC_n$個あり、一つの元も対応しない場合
も合わせると
\begin{equation*}
    1+_nC_1+_nC_2+\cdots+_nC_n=2^n
\end{equation*}
となるからである。(この計算は基礎数学1問題集のstep up 464等を参照するといい。)
同様に写像についても考える。写像の場合は、一つの元$a_1$に対して$B$の元が一つ対応するわけだから、その数は$B$の元の数と同じになる。
よって$n$個。

次に、$A$のすべての元について考える。対応は$a_1$のときは$2^n$個あり、元は$a_1,a_2,\cdots,a_m$とあるわけだから、
$2^n\times 2^n\times\cdots\times 2^n=(2^n)^m=2^{mn}$となり、答えは$2^{mn}$個となる。
同様に、写像も$n\times n\times\cdots\times n=n^m$となるため、$n^m$個となる。\\\\
\rightline{\underbar{対応は$2^{mn}$個,写像は$n^m$個}}
\color{black}
\newpage
\subsection{合成写像ってなんすか}


\end{document}