\documentclass[a4j,dvipdfmx]{jsarticle}
\usepackage{amsmath,amssymb}
\usepackage{ulem}
\usepackage{siunitx}

\renewcommand{\thesection}{\Roman{section}}
\renewcommand{\thesubsection}{\roman{subsection}}

\begin{document}
\section*{写像ってなんすか}
「写像」...某論破王の影響で言葉事態は知っている人が多いかもしれない。ただ、その意味について考えたことはあるだろうか?
おそらくほとんどの人は「写像」の意味を知らないと思う。そこで今回はこの「写像」について学んでみよう。\xout{そして某論破王を論破しよう!}

\subsection{写像}
実数について考えるときは文字を含めることがある。例えば、
\begin{equation}
    x+1\quad y^2-1\quad \sqrt{1-w^2}\quad \frac{z}{z+1}\quad 
\end{equation}
\end{document}