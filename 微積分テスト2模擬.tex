\documentclass[a4j,dvipdfmx]{jsarticle}
\usepackage{amsmath,amssymb}
\usepackage{siunitx}
\usepackage{ascmac}

\begin{document}
    \section*{愛好会模擬テスト 微積分2}
    \begin{itembox}[c]{テストを始める前に}
        テスト時間は30分であり、テスト開始の合図があるまで取り組まないこと。\\

        今回は模擬試験であるため、点数は気にしなくてもよい。\\

        テスト範囲は、「不定積分」「定積分」「広義積分」「数値積分」「関数の展開」「オイラーの等式」である。一部の公式をこの枠の下に書いておく。自由に参考にせよ。\\

        問題は大問1から大問6まで存在する。大問1は1問2点(5.だけ3点)、大問6は一問4点である。残りはすべて1問3点である。\\

        持ち物は、鉛筆またはシャーペン、消しゴム、定規、答えを書く用の紙、計算用紙のみとする。
    \end{itembox}
    \subsubsection*{公式集}
    \begin{alignat*}{3}
        & \int f(x)dx=\int f(g(t))g'(t)dt  &/& \int f'(x)g(x)dx = f(x)g(x)-\int f(x)g'(x)dx\\
        & f(x)=\frac{d}{dx}\int_a^x f(t)dt &/& \int_a^b f(x)=F(b)-F(a)\\
        & \int_a^b f(x)dx=\int_{g(a)}^{g(b)} f(g(t))g'(t)dt &/& \int_a^b f'(x)g(x)dx = [f(x)g(x)]_a^b-\int f(x)g'(x)dx\\
        & \int_{-\infty}^{\infty} f(x)dx=\lim_{a\to -\infty}\lim_{b\to\infty}\int_a^b f(x)dx &/& \int_a^b f(x)dx=\lim_{\epsilon_1\to +0}\int_a^{c-\epsilon_1}f(x)dx+\lim_{\epsilon_2\to+0}\int_{c+\epsilon_2}^bf(x)dx\\
        & \int_a^b f(x)dx\fallingdotseq\frac{h}{2}\{f(a)+2f(a+h)+2f(a+2h)&+\cdots&+2f(a+(n-1)h)+f(b)\}\\
        & \int_a^b f(x)dx\fallingdotseq\frac{h}{3}\{f(a)+4f(a+h)+2f(a+2h)&+4f(&a+3h)+\cdots+2f(a+(2m-2)h)+4f(a+(2m-1)h+f(b))\}\\
        &f'(c)=\frac{f(b)-f(a)}{b-a}\quad(a<c<b)&/&\lim_{x\to a}\frac{f(x)}{g(x)}=\frac{f'(a)}{g'(a)}\\
    \end{alignat*}
    マクローリン展開
    \begin{equation*}
        f(x)=f(0)+f'(0)x+\frac{1}{2!}f^{''}(0)x^2+\cdots+\frac{1}{n!}f^{(n)}(0)x^n+\frac{1}{(n+1)!}f^{(n+1)}(\theta x)x^{n+1}\quad(0<\theta<1)
    \end{equation*}
    オイラーの公式$\quad e^{i\theta}=\cos \theta+i\sin\theta$\\\\
    次のページから問題です。
    \newpage
    \subsubsection*{大問1}
    以下の不定積分、定積分を求めよ。ただし、$a,b$は定数とする。
    \begin{align*}
        &(1)\quad\int \frac{dx}{\tan\theta}&&(2) \int_{-2}^{2}\frac{dx}{x^2-9}&&(3) \int_1^{\infty}\frac{dx}{\sqrt{x}}\\
        &(4)\int_{0}^{\frac{\pi}{2}}\frac{\cos x}{\sqrt{1-\sin x}}dx&&(5) \int_0^{\pi} \frac{dx}{a-\cos x}\quad(a>1)
    \end{align*}
    \subsubsection*{大問2}
    以下の関数をマクローリン展開せよ。
    \begin{align*}
        &\quad (1)e^x\\
        &\quad (2)\sin x
    \end{align*}
    \subsubsection*{大問3}
    以下を証明せよ。\\
    
    (1) $\displaystyle \Gamma(s)=\int_0^{\infty}x^{s-1}e^{-x}dx$は$s>0$のとき確定した値を持つ。$\Gamma(s)$はガンマ関数と呼ばれる。このとき次のことを示せ。
    \begin{equation*}
        1.\Gamma(s+1)=s\Gamma(s)\quad2.\Gamma(n+1)=n!\quad(n\in\mathbb{N} )
    \end{equation*}

    (2) $\displaystyle y^2=4ax\text{と}x^2=4ay\quad(a>0)$の面積を$S$とすると、
    \begin{equation*}
        S=\frac{16}{3}a^2
    \end{equation*}
    となることを示せ。

    (3) $\displaystyle \lim_{x\to+0}x^n\log x$は(nが正の整数)であるとき極限値が0となる。このことを示せ。
    \subsubsection*{大問4}
    定積分$\displaystyle \int_4^8\frac{dx}{x}dx$について、以下の問いに答えよ。

    (1) 定積分の値をこたえよ。

    (2) $n=3$の台形公式を使い、定積分の値をこたえよ。

    (3) $n=3$のシンプソンの公式を使い、定積分の値をこたえよ。

    \subsubsection*{}
    (まだ続きます。)
    \newpage
    \subsubsection*{大問5}
    次の文章はのびたくんとドラえもンの会話である。これを読んで後の問に答えよ。
    \begin{itembox}[c]{のびたくんとドラえもンの会話1}
        のびた  :どらえも~ん、助けてよ~!\\
        ドラえもン:どうしたんだい、のびたくん?\\
        のびた  :実はね、昨日ジャいアンたちと遊んでいたら、ジャいアンを怒らせちゃって...。\\
        ドラえもン:それは大変だね...。でも明日になったら機嫌はもとに戻るんじゃない?\\
        のびた  :それがね、明日までにこの問題を解けって言われちゃったんだよ~。解けなきゃぶっ飛ばすっていってるんだよ~!\\
        ドラえもン:それはまずいね。そいで、どんな問題なんだい?\\
        のびた  :えっとね...。定積分の問題なんだけど、$a>0$だとして、
        \begin{equation*}
            \int_0^a\sqrt{a^2-x^2}dx
        \end{equation*}
              っていうのなんだけど...。\\
        ドラえもン:なるほどね、こういう時は...!(ポケットをまさぐる音)\\\\
        \centerline{\underbar{(効果音)$x=a\sin t$と置いて置換する~!(某青だぬき風に)}}\\

        のびた  :おお!さすがぁ、ドラえもン!こんどどらやき買って帰るよ!
        (つづく)
    \end{itembox}
    (1) 下線部のドラえもンのヒントをもとに
    \begin{equation*}
        \int_0^a\sqrt{a^2-x^2}dx
    \end{equation*}
    を求めよ。

    \begin{itembox}[c]{のびたくんとドラえもんの会話2}
        のびた  :どらえも~ん!今度はもっとむずかしいもんだいがだされたよぉ~。\\
        ドラえもン:...今度はなんだい?\\
        のびた  :今度はね、$\frac{1}{e}$の近似値を求めろっていうんだよ!\\
        ドラえもン:へぇ~、それじゃあ\underbar{マクローリン展開しちゃえばいいんじゃない?}\\
        のびた  :あ、確かに。ごめんもういいかも。\\
        ドラえもン:おい、どら焼きは?\\
        のびた  :もう用ないからいいよ。あっち行ってて。
    \end{itembox}
    (2) 下線部のドラえもンのヒントをもとに$e$の近似値を途中式を記述して求めよ。
    \newpage
    \subsubsection*{大問6}
    以下の不定積分
    \begin{equation*}
        I=\int \sin x\sin 2x\sin 3xdx
    \end{equation*}
    を次の``手順"で解いた。($\alpha$)~($\delta$)に当てはまる数式や数字をかけ。
    \begin{itembox}{手順}
        三角関数の加法定理から。$\sin x\sin 2x=$\fbox{$(\alpha)$}である。よって$\sin$の和で表すと
        \begin{equation*}
            \sin x\sin 2x\sin 3x=\frac{1}{4}(\fbox{\text{($\beta$)}})
        \end{equation*}
        であることを用いて、
        \begin{align*}
            I&=\frac{1}{4}\int (\fbox{$(\gamma)$})dx
        \end{align*}
        となる。よって、
        \begin{equation*}
            \int \sin x\sin 2x\sin 3xdx=\fbox{$\delta$}
        \end{equation*}
    \end{itembox}
    \rightline{問題はこれで終わりです。見直しをしておきましょう。}\\
    \rightline{点数$\quad \quad /60$}
\end{document}