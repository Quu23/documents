\documentclass[a4j,dvipdfmx]{jsarticle}

\usepackage[dvipdfmx]{graphicx}
\usepackage{amsmath,amssymb}
\usepackage{siunitx}
\usepackage{ascmac}
\usepackage[subrefformat=parens]{subcaption}
\usepackage{fancyhdr}
\usepackage{otf}
\usepackage[dvipdfmx]{hyperref}
\usepackage{pxjahyper}
\usepackage{okumacro}
\usepackage{tikz}
\usepackage{bm}

\pagestyle{headings}

% \renewcommand{\thesubsection}{\arabic{subsection}}
\renewcommand{\headrulewidth}{1pt}
\renewcommand{\Re}{\operatorname{Re}}
\renewcommand{\Im}{\operatorname{Im}}


\newcounter{basic_quastion}\setcounter{basic_quastion}{1}
\newcommand{\basicquestion}{\noindent{\large 基本問題\hspace{1mm}\huge\fbox{\textbf{\arabic{basic_quastion}}}\addtocounter{basic_quastion}{1}}\thispagestyle{fancy}\lhead{$\Sigma$基本問題}\rhead{\thepage}\cfoot{}\quad}
\newcommand{\sign}{\mathop{\mathrm{sign}}\nolimits}
\newcommand{\linktoMOKUZI}{\vspace{\stretch{1}}\fbox{\centerline{\hyperref[目次]{目次に戻る}}}}

\newcounter{basic_answer}\setcounter{basic_answer}{1}
\newcommand{\basicanswer}{\noindent{\large 基本問題\hspace{1mm}\huge\fbox{\textbf{\arabic{basic_answer}}}\addtocounter{basic_answer}{1}}\thispagestyle{fancy}\lhead{基本問題解答}\rhead{\thepage}\cfoot{}\quad}

\newcommand{\ner}{
\begin{tikzpicture}[scale=0.3,baseline=0.3]
\draw[->,>=stealth] (0,0) to[bend right=45] (1,1);
\end{tikzpicture}
}

\newcommand{\nel}{
\begin{tikzpicture}[scale=0.3,baseline=0.3]
\draw[->,>=stealth] (0,0) to[bend left=45] (1.2,1);
\end{tikzpicture}
}

\newcommand{\sel}{
\begin{tikzpicture}[scale=0.3,baseline=0.3]
\draw[->,>=stealth] (0,1) to[bend left=45] (1,0);
\end{tikzpicture}
}

\newcommand{\ser}{
\begin{tikzpicture}[scale=0.3,baseline=0.3]
\draw[->,>=stealth] (0,1) to[bend right=45] (1.2,0);
\end{tikzpicture}
}



\title{Quuノート ー微分積分\ajRoman{1}ー}
\date{最終更新 2023/12/01}
\author{責任者 Quu}

\begin{document}
    \maketitle
    \thispagestyle{empty}
    \begin{figure}[h]
        \centering
        \includegraphics[scale=0.5]{img/QuuNote/icon.png}
    \end{figure}
    
    \vspace{\stretch{1}}
    \centerline{\textbf{概要}}
    \noindent
    微分積分学入門についてのノート。\\
    主に、一変数の極限、一変数の微分・積分、実数の無限級数について扱う。
    
    \clearpage
    \label{目次}
    \tableofcontents
    \clearpage
    
    \part{前提知識}
    \vspace{\stretch{1}}
    \begin{screen}
        微分積分を学ぶうえで前提となる知識をまとめた。微分積分は主に関数の微分・積分について扱うわけだから、ある程度の関数の扱い方も知っておく必要がある。
        そのほか数の種類についてや閉区間・開区間、極限についてもまとめてある。極限は微分積分を学ぶ際にいたるところに出てきて、陰から支える縁の下の力持ち的な役割を持つ。極限は一見すると
        代入と同じように見えるが、実は違う。極限は代入だと都合が悪い時にありがたみが実感できる。極限に関連して、無限という概念も登場する。
    \end{screen}
    \clearpage
    \section{様々な`数'と数直線}

        \subsection{数の種類}
            数学を勉強するうえで、様々な数が登場する。まず一番初めに思いつくのが$1,2,3...$といった\textbf{自然数}である。
            次の自然数に$0$と負の符号をつけたものを加えた\textbf{整数}が考えられる。整数同士で足し算、引き算、掛け算を行っても
            その値は整数である。このことを\textbf{和、差、積について閉じている}という。これは$a,b\in \mathbb{Z}$となる任意の$a,b$
            について
            \begin{equation}
                a+b , a-b , a\times b \in \mathbb{Z}
            \end{equation}
            が成り立つことを意味している。

            一方割り算は整数の中に閉じていない。\footnote{例えば$1\div 2$など。}しかし、$0.5,3.14$などの\textbf{有理数}まで数を拡張すれば、その中に商は閉じている。
            つまり、$a,b \in \mathbb{Q}$となる任意の$a,b$について
            \begin{equation}
                \frac{a}{b} \in \mathbb{Q}
            \end{equation}
            となる。よって、数を有理数まで拡張すれば\textbf{四則について閉じている}ことがわかる。

            さらに数の拡張を考えよう。たとえば$x^2-2=0$を満たす$x$について考えてみるには、数を\textbf{無理数}まで拡張しなければならない。
            一般の二次方程式の解も
            \begin{equation}
                x = \frac{-b\pm \sqrt{b^2-4ac}}{2a}
            \end{equation}
            と有理数だけでは表現できないことがわかる。無理数には$\pi,e\footnote{自然対数の底またはネイピア数と呼ばれる。具体的な値は$e=2.71...$}$などの\textbf{超越数}もふくむ。

            私たちの生活の中では有理数と無理数をあわせた\textbf{実数}があれば十分事足りるが、数学の世界ではそうもいかない。先ほどの二次方程式についてより深く調べてみると、
            解を持たない条件(根号の中身が負)があることがすぐに分かる。例えば、$x^2+1 = 0$は$x^2 = -1$と変形できるが、二乗して負になるような数は実数のうちには存在しない。
            よってこの方程式は\textbf{解なし}となる。がしかし、ここで
            \begin{equation}
                i = \sqrt{-1}
            \end{equation}
            となる`数'を定義してあげると、方程式は$x=\pm i$となり、$(実数)+i$を含めた範囲に解をもつことがわかる。
            この数は、今までの実数とは異なる数であり、実数との和,差は直接計算できない。一般に実数$a,b$と$i$を用いて
            \begin{equation}
                z = a + bi
            \end{equation}
            として表した$z$を\textbf{複素数}といい、$i$を虚数単位という。さらに$a$を$z$の実部、$b$を$z$の虚部といい、それぞれ
            $a=\Re z,b=\Im z$と表す。

            先ほど、二次方程式の解を有理数だけでは表現できないといったが、実は無理数を含めてもできない。この複素数を含めることで初めてすべて表現できるようになるのだ。\footnote{もちろんこれで数の拡張が終わるわけではない。しかし微分積分を学ぶうち間は複素数まで拡張すれば事足りる。}
        \subsection{数直線}
            では、数の大小関係をわかりやすくするためにはどうすればよいだろうか。視覚的にわかりやすくするためには数直線を用いればよい。
            \begin{figure}[h]
                \centering
                \includegraphics[keepaspectratio,scale=0.5]{img/QuuNote/NumLine_1.png}
                \caption{数直線}
            \end{figure}
            
            上の例では、整数、有理数、無理数の一部を記載している。当然書いてある数以外も数直線の中には含まれている。
            むしろ、数の点の集まりとして数直線を捉える方がイメージがわきやすいかもしれない。
            ここで注意しなければならないのは、この数直線上に複素数$(\Im z\neq 0)$は含まれないといけないということである。
            数直線は\underline{実数を表す}直線なので、複素数のすべては含むことができないのである。

            では実数も含めた複素数はどう表せばよいのか。答えは単純で実数の軸\footnote{これを実軸と呼ぶ。}とは別の軸\footnote{虚軸という。}を
            加えればよい。つまり複素数は平`面'上で表せられるのである。\footnote{この平面を複素平面という。いつものy-xグラフとは見た目は同じだが感覚が違うので注意。}  
            \newpage
        \subsection{開区間・閉区間}
            実数が数直線上の一点で表せることはすでに前項で述べた。では点に続いて次は区間について考えていこう。

            区間は大きく二つある。それらはそれぞれ\textbf{開区間}、\textbf{閉区間}と呼ばれる。これらの違いは端点を含むかどうかで、
            逆に言えば端以外は同じである。例えば、$1<x<2,1\leq x\leq2$について前者は端点$x=1,2$を含まず、後者は端点を含むのである。
            端点を含まない場合が開区間、端点を含む場合が閉区間である。

            閉区間、開区間を数直線上で表すにはどうすればよいだろうか。これも数直線と同様に区間の端から端まで線を引けばよい。注意しないといけないのが
            端点で、区間が開区間か閉区間かによって端点を書き分けないといけない。開区間のときは$\circ$、閉区間のときは$\bullet$と書けばよい。

            \begin{figure}[h]
                \centering
                \includegraphics[keepaspectratio,scale=0.7]{img/QuuNote/IntervalLine.png}
                \caption{開区間と閉区間}
            \end{figure}
            例えば上図の例をみてみると、$A,B,C$の三つの区間がある。それぞれ$-4\leq x\leq -1,0<x<3,-1\leq x<1$となる。
            今までは不等号を用いて区間を表現してきたが\footnote{厳密に言えば区間は集合なので、不等号を用いて区間を表現するという言い方は適切ではない。}、もっと簡潔に$(\hspace{1mm}),[\hspace{1mm}]$を用いて表現する方法もある。
            この表現方法を使えば、$A,B,C$はそれぞれ$[-4,-1],(0,3),[-1,1)$と表せる。$(\hspace{1mm})$が等号を含まない、$[\hspace{1mm}]$が
            等号を含む、というわけである。

            では、値が無限に続く(例えば実数全体など)場合はどう表現すればよいのか。この場合は無限大の記号$\infty$を用いて$(-\infty,\infty)$などと表せばよい。\footnote{この方法を使えば、a以上の実数などの場合でも$[a,\infty)$と表せばよいことがわかる。}
            \clearpage
            
        \basicquestion 以下問に答えよ。

        \paragraph{問1}以下の主張のうち正しいものには〇を、間違っているものには×をつけよ。
            \begin{enumerate}
                \item $\sqrt{9}$は無理数である。
                \item 有理数は全て分子分母が整数である分数の形で表せる。
                \item $i$は複素数である。
                \item 有理数の集合は$\mathbb{Q}$として表し、無理数の集合は$\mathbb{N}$で表す。
                \item 自然数全体の集合(区間)は$(0,\infty]$である。
            \end{enumerate}
        \paragraph{問2}以下の区間について、数直線上に示せ。もし数直線上に記されていない数字が出てくる場合はそれも記載せよ。
            \begin{figure}[h]
                \centering
                \includegraphics[keepaspectratio,scale=0.6]{img/QuuNote/NumLine_1.png}
                \caption{数直線}
            \end{figure}

            $1.\quad [2,3]$\hspace{3mm}
            $2.\quad (3,5)$\hspace{3mm}
            $3.\quad [-5,\pi]$\hspace{3mm}
            $4.\quad (-2,0.5]$\hspace{3mm}
            $5.\quad [-1,0)$\hspace{3mm}
            $6.\quad (-\infty,0)$\hspace{3mm}
            $7.\quad [0,\infty)$\hspace{3mm}
        \clearpage
        
        \section{関数の性質}
            \subsection{偶関数・奇関数}
                一般の関数$f(x)$について、$f(-x)=f(x)$を満たすものを\textbf{偶関数}、$f(-x)=-f(x)$を満たすものを\textbf{奇関数}という。
                もちろん全ての関数が偶関数・奇関数のどちらかであるというわけではない。しかし、全ての関数は偶関数と奇関数の和で表せられること
                が知られている。\\

                関数が偶関数・奇関数である場合のグラフはどうなるだろうか。まずは偶関数から考えてみると、定義より$x>0$と$x<0$の点において$f$は
                同じ値を取るわけであるから、グラフはy軸に対して対象になるはずである。つぎに奇関数について考えてみよう。これも定義より$x>0$と$x<0$の
                点において、$f$はx軸に対してそれぞれ対象に点を取るはずである。つまり、グラフは原点に対して点対象になるはずである。\\

                偶関数の例となる関数は、$x^2,\cos x,a(\text{定数関数})$などがあげられる。奇関数の例となる関数は、$x,\sin x,\tan x$などがあげられる。
                各自でグラフソフトなどでグラフを見てみるとよい。
            \clearpage
            \subsection{べき関数}
                関数のなかでもっともなじみやすいのが、$f(x)=x^n$であろう。例えば$x$は一次関数、$x^2$は二次関数と呼ばれる。
                別に$n$は自然数に限らなくてもよい。$x^{\frac{1}{2}}=\sqrt{x}$は指数が自然数ではないが、これもべき関数の一つである。
                $n$が自然数のうちは、関数の定義域について特別意識をする必要はない。しかし$n=\frac{1}{2}$などのように指数が有理数であったり、
                $n=-1$のように指数が負の値を取る場合には定義域に十分注意する必要がある。このように、一般に$f(x)=x^n$で表される関数を\textbf{べき関数}という。

                次に関数のグラフについて、グラフ描画ソフトを用いて数式を入力すると以下のようになる。
                \begin{figure}[h]
                    \centering
                    \includegraphics[keepaspectratio,scale=0.5]{img/QuuNote/PowerFuncGraph.png}
                    \caption{べき関数グラフ}
                \end{figure}

                図からも$\sqrt{x}$が$x<0$で定義されないことがわかる。また、$x^2$と$\sqrt{x}$は$y=x$を軸にして線対象になっており、
                $x^2$と$\sqrt{x}$は互いに\textbf{逆関数}であることがわかる。
            \clearpage
            \subsection{三角関数}
                \textbf{三角関数}は三角比を一般角に拡張した関数である。直交座標から極座標に変換する際などにしばし用いられる。
                三角比の定義自体を忘れた人はいないだろうが一応説明しておく。
                \begin{figure}[h]
                    \centering
                    \includegraphics[keepaspectratio,scale=0.3]{img/QuuNote/triangleFunc.png}
                    \caption{三角比}
                \end{figure}

                上図\footnote{aの辺のことを対辺、bの辺のことを隣辺、cの辺のことを斜辺という。}において
                \begin{align}
                    \sin \theta &= \frac{a}{c}\\
                    \cos \theta &= \frac{b}{c}\\
                    \tan \theta &= \frac{a}{b}
                \end{align}
                また、定義より$\displaystyle\tan \theta = \frac{\sin \theta}{\cos \theta}$が成り立つ。

                三角関数には様々な公式がある。\footnote{公式集を眺めるとやたらと二乗がついていることがわかる。つまり三角関数は\underline{二乗に強い}のである。この性質は積分を解く際に重要である。}
                しかしそれらは単位円を書けばすぐに導けるので、一部を除いて割愛する。
                また、二倍角の公式などについても、全て\textbf{加法定理}より導けるのでここでは加法定理のみ紹介する。

                \begin{align}
                    \sin^2 x + \cos ^2 x &= 1 &\quad 1 + \tan^2 x &= \frac{1}{\cos^2 x}\\
                    \sin\left(\frac{\pi}{2}-x\right) &= \cos x &\quad \cos\left(\frac{\pi}{2}-x\right) &= \sin x\\
                    \tan\left(\frac{\pi}{2}-x\right) &= \frac{1}{\tan x} &&
                \end{align}
                \begin{align}
                    \sin(\alpha\pm\beta) &= \sin\alpha\cos\beta \pm \cos\alpha\sin\beta\\
                    \cos(\alpha\pm\beta) &= \cos\alpha\cos\beta \mp \sin\alpha\sin\beta\\
                    \tan(\alpha\pm\beta) &= \frac{\tan\alpha\pm\tan\beta}{1\mp\tan\alpha\tan\beta}
                \end{align}
                また、三角関数は\textbf{周期関数}である。$\sin x,\cos x$は周期$2\pi$、$\tan x$は周期$\pi$である。
            \clearpage
            \subsection{指数・対数関数}
                `指数関数的に増加する'という言葉をよく耳にする。これは、なにか爆発的な増加の様子を示している表現である。このように、\textbf{指数関数}は$x$の値が少し変わるだけで値が
                大きく増加・減少する関数である。その具体的な表式は$a^x$と表される。指数の部分が変数になっているのである。

                指数関数は、$a$の値によって性質が少し異なる。$0<a<1$の場合には単調減少関数となり、$1<a$の場合には単調増加関数になる。
                このとき$a$が負の値の場合は定義しない。\footnote{例として$(-2)^x$のグラフを書いてその理由を考えてみるといい。}\footnote{実際は定義することができるがその際には複素関数の知識が必要。なので今回は扱わない。}

                以下、指数法則について述べる。$a,b>0\quad x,y\in \mathbb{R}$とすると、
                \begin{align}
                    a^x\cdot a^y&=a^{x+y}\\
                    \frac{a^x}{a^y} &= a^{x-y}\\
                    (a^x)^y &= a^{xy}\\
                    (ab)^{x} &= a^x\cdot b^x
                \end{align}

                指数関数のグラフは、以下のようになる。
                \begin{figure}[h]
                    \centering
                    \includegraphics[keepaspectratio,scale=0.3]{img/QuuNote/ExpFuncGraph.png}
                    \caption{指数関数のグラフ}
                \end{figure}

                グラフから、$a$が1より大きくても小さくても$x=0$で$y=1$を取ることがわかる。

                なお、底が$e$の場合の指数関数は$e^x=\exp{x}$と書くこともある。\\

                では次に、指数関数の逆関数を考えてみよう。指数関数の逆関数は与えられた値に対して、
                底を何回掛けたらその値になるかの回数を表す関数である。つまり、指数関数の底ごとに
                逆関数が存在する。文章で見てもわかりずらいので数式で以下示す。

                \begin{equation}
                    f(x)=a^x \leftrightarrow x = f^{-1}(y) = \log_a{y}
                \end{equation}
                指数関数の逆関数は底が何かを示さないといけないので、逆関数$\log$に下付き文字で書く。
                しかし、底が$e$だった場合は省略して$\log y$と書いてもよい。この関数を\textbf{自然対数}という。\footnote{自然対数は$\ln x$と書くこともある。natural logarithmのことである。}
                底が$e$じゃない場合は単に対数と呼ぶ。\footnote{底が10の場合は常用対数という。}

                以下、対数の性質を述べる。必要ない限り底は省略して記載する。指数法則と見比べると理解が深まる。

                \begin{align}
                    \log(xy)&=\log x+\log y\\
                    \log\left(\frac{x}{y}\right)&=\log x - \log y\\
                    \log(a^b) &= b\log a \\
                    \frac{\log_c b}{\log_c a}&=\log_a b\qquad(\text{\textbf{底の変換公式}})
                \end{align}

                また、対数の定義より
                \begin{equation}
                    \log 1 = 0\quad \log_a a = 1
                \end{equation}
                が成り立つ。対数関数$\log x$の引数$x$のことを真数と呼び、これは$x>0$である。\footnote{真数が正であるという条件のことを真数条件という。}

                対数関数のグラフは以下のようになる。

                \begin{figure}[h]
                    \centering
                    \includegraphics[keepaspectratio,scale=0.5]{img/QuuNote/LogFuncGraph.png}
                    \caption{対数関数}
                \end{figure}
                
                グラフを見ればわかるように、底が1より大きいか小さいかで単調増加・減少かが変わる。

                対数関数は爆発的に増加・減少する指数関数とは対照的に、値の変化が($x<1$を除いて)緩やかである。
                そのため、値がとても大きい値でも対数を取ることで値のスケールを小さくすることができる。
                また、対数を取ることで\underline{掛け算を足し算にできる}。この性質は非常に重要である。
            \clearpage
            \subsection{逆三角関数}
                指数関数の逆関数である対数関数を考えたのと同じように、三角関数の逆関数も考えてみよう。
                三角関数は与えられた角度に対応するそれぞれの三角比を返す関数である。では三角関数の逆関数は
                与えられた三角比に対応する`角度'を返す関数であることがすぐに分かる。これらを次のように書くことにする。
                \begin{align}
                    \arcsin x &= \sin^{-1} x \\
                    \arccos x &= \cos^{-1} x \\
                    \arctan x &= \tan^{-1} x
                \end{align}
                左辺にちなんで左からそれぞれ「アークサイン」,「アークコサイン」,「アークタンジェント」と読む。これらをまとめて
                \textbf{逆三角関数}という。表記に左辺を用いるか右辺を用いるかは個人の好みによる。\footnote{だからといって、$\frac{1}{\sin x}$を$\sin^{-1}x$と書くことはまずない。}

                三角関数が周期関数であるため、逆三角関数は多価関数であることは容易に想像できる。
                逆三角関数を一価関数にするため、値域をそれぞれ$[-\frac{\pi}{2},\frac{\pi}{2}],[0,\pi],(-\frac{\pi}{2},\frac{\pi}{2})$
                に制限して用いることがある。このことを\ruby{主枝}{しゅし}を取るという。またこの制限した値域を主枝という。
                主枝以外の値域を分枝と呼ぶ。

                逆三角関数が主枝を取っていることを明示するために
                \begin{align}
                    {\rm Arcsin} x &= {\rm Sin^{-1}} x \\
                    {\rm Arccos} x &= {\rm Cos^{-1}} x \\
                    {\rm Arctan} x &= {\rm Tan^{-1}} x
                \end{align}
                のように、先頭を大文字で書くこともある。しかし今回はこの記法は採用しない。

                逆三角関数のグラフは以下のようになる。
                \begin{figure}[h]
                    \begin{minipage}{5cm}
                        \centering
                        \includegraphics[keepaspectratio,scale=0.3]{img/QuuNote/ArcsinFuncGraph.png}
                        \caption{$\arcsin x$のグラフ}
                    \end{minipage}
                    \begin{minipage}{5cm}
                        \centering
                        \includegraphics[keepaspectratio,scale=0.3]{img/QuuNote/ArccosFuncGraph_ver2.png}
                        \caption{$\arccos x$のグラフ}
                    \end{minipage}
                    \begin{minipage}{5cm}
                        \centering
                        \includegraphics[keepaspectratio,scale=0.3]{img/QuuNote/ArctanFuncGraph.png}
                        \caption{$\arctan x$のグラフ}
                    \end{minipage}
                \end{figure}
                
                もちろん主枝を取らない場合は、それぞれと同じグラフが上や下につながっていく。\\

                三角関数の公式から、逆三角関数についても公式が導ける。例えば、
                \begin{equation}
                    \sin^{-1} x+\cos^{-1} x =\frac{\pi}{2}
                \end{equation}
                ほかにも公式が導けるので、各自で考えてみるとよい。
            \clearpage
            \subsection{双曲線関数}
                いきなりだが、次のように関数を定義する。$e$はネイピア数である。
                \begin{align}
                    \sinh x &= \frac{e^x - e^{-x}}{2}\\
                    \cosh x &= \frac{e^x + e^{-x}}{2}\\
                    \tanh x &= \frac{\sinh x}{\cosh x} = \frac{e^x - e^{-x}}{e^x + e^{-x}}
                \end{align}
                これらは\textbf{双曲線関数}と呼ばれる。読み方はそれぞれ「ハイパボリックサイン」、「ハイパボリックコサイン」、
                「ハイパボリックタンジェント」である。ただこれだと長ったらしいので「シンチ」、「コッシュ」、「タンチ」と呼ぶ
                場合もある。

                見た目が三角関数と酷使しているが、実は性質も似たものを持つ。例えば、$\cosh^2 x - \sinh^2 x = 1$など。
                また、加法定理も符号は若干異なるがほとんど同じ形をしている。

                さらに、オイラーの公式$e^{ix}=\cos x + i\sin x$\footnote{この公式自体はだいぶ後になって解説する。}を用いれば、
                $\sin ix = i\sinh x,\cos ix=\cosh x$が導ける。\footnote{むしろこの性質が成り立つように双曲線関数を定義するといったほうが正しいかもしれない。}\\

                双曲線関数のグラフは以下のようになる。
                \begin{figure}[h]
                    \begin{minipage}{5cm}
                        \centering
                        \includegraphics[keepaspectratio,scale=0.3]{img/QuuNote/SinhFuncGraph.png}
                        \caption{$\sinh x$のグラフ}
                    \end{minipage}
                    \begin{minipage}{5cm}
                        \centering
                        \includegraphics[keepaspectratio,scale=0.3]{img/QuuNote/CoshFuncGraph.png}
                        \caption{$\cosh x$のグラフ}
                    \end{minipage}
                    \begin{minipage}{5cm}
                        \centering
                        \includegraphics[keepaspectratio,scale=0.3]{img/QuuNote/TanhFuncGraph.png}
                        \caption{$\tanh x$のグラフ}
                    \end{minipage}
                \end{figure}

                $\cosh x$のグラフはカテナリーと呼ばれる。電柱などの垂れた線はこれにあたる。
            \clearpage
            \subsection{初等関数}
                前節まで述べたべき関数、三角関数、指数・対数関数、逆三角関数は\textbf{初等関数}という。
                また、それらの関数からなる多項式の関数や初等関数の合成関数は初等関数である。例えば、双曲線関数は
                $e^x,e^{-x}$の和でなっているが、これらは初等関数なので双曲線関数も初等関数である。

                しかし初等関数の逆関数は必ずしも初等関数であるとは言えない。よく例に挙げられるのが$f(x)=xe^x$の逆関数である。
                この関数はランベルトのW関数とよばれ、$f^{-1}(x)=W(x)$と書かれる。

                初等関数は性質がよく知られているので、微分・積分するうえで比較的扱いやすい。初等関数を微分したもの
                も初等関数であるが、初等関数を積分したものが必ずしも初等関数である保証はない。とはいえ今は微分も積分も知らないわけだから、
                単に事実として受け入れるだけでよい。\\

                初等関数に対して高等関数というものもある。これは初等関数以外の関数のことで、初等関数よりも数は多い。
                先ほど紹介したランベルトのW関数以外には以下のようなものがある。

                \begin{align}
                    \Gamma(s) &= \int_0^\infty e^{-x}x^{s-1}dx\qquad&&(\text{\textbf{ガンマ関数}})\\ 
                    B(p,q) &= \int_{0}^{1} x^{p-1}(1-x)^{q-1}dx\qquad&&(\text{\textbf{ベータ関数}})\\
                    {\rm Li}(x) &= \int \frac{dx}{\log x}\qquad&&(\text{\textbf{対数積分}})\\
                    \zeta(s) &=  \sum_{n=1}^{\infty}\frac{1}{n^s}\qquad&&(\text{\textbf{ゼータ関数}})
                \end{align}

                もちろんこれ以外にも様々な関数がある。興味があったら調べてみるといい。
                \clearpage
                \basicquestion 以下の問いに答えよ。

                \paragraph{問1}次の関数が偶関数か奇関数かを判別せよ。

                \noindent
                (1)$x\sin x$\hspace{3mm}
                (2)$x^5$\hspace{3mm}
                (3)$\sinh x$\hspace{3mm}
                (4)$\log|x^2|$\hspace{3mm}
                (5)$x^3+x+\sin x$\hspace{3mm}
                (6)$e^{-x}$\hspace{3mm}
                (7)$f(\cos x)$\hspace{3mm}
                (8)$\arctan x$

                \paragraph{問2}以下の等式を証明せよ。

                \noindent
                $(1)\sin 2x=2\sin x\cos x/\cos 2x=\cos^2 x-\sin^2 x$(\textbf{倍角の公式})\\
                $(2)\displaystyle \sin^2\frac{x}{2}=\frac{1-\cos x}{2}/\cos^2\frac{x}{2}=\frac{1+\cos x}{2}$(\textbf{半角の公式})\\
                $(3)\sinh(x+y)=\sinh x\cosh y + \cosh x\sinh y/\cosh(x+y)=\cosh(x+y)=\cosh x\cosh y + \sinh x\sinh y$\\
                
                \paragraph{問3}以下の値を求めよ。

                \noindent
                $(1)\sin \pi+\cos \frac{3}{2}\pi + \tan(-\pi)$\hspace{3mm}
                $(2)\arcsin(\frac{1}{2})$\hspace{3mm}
                $(3)\arccos(1)$\hspace{3mm}
                $(4)\log_28$\hspace{3mm}
                $(5)\log_a(\tan(\frac{\pi}{4}))\quad(a>1)$\\
                $(6)\log_63+\log_62$\hspace{3mm}
                $(7)\arcsin(1-\log\pi)-\cos(\log2)\sin(\log 3)+\sin(\log3 -\log 2)-\log e^{\arccos(\log \pi-1)}$

                \paragraph{問4}対数を使えば桁数の多い数字同士の掛け算を足し算で計算することができる。簡単な例として$271\times 314$を
                対数を用いて計算せよ。ただし、常用対数$\log_{10} 2.71\simeq0.4346,\log_{10} 3.14\simeq0.4969,10^{4.9315}\simeq85408$は用いてよい。
                
                \paragraph{問5}$t=\tan\frac{x}{2}$とするとき、$\sin x,\cos x,\tan x$をそれぞれ$t$を用いた式で表せ。\footnote{ヒント:半角の公式を用いる。}
            \clearpage
            \section{極限}
            \subsection{数列の極限}
                \textbf{数列}とは、数字をある規則によって並べた列のことで、例えば$1,2,3,4,\cdots,100$などがある。
                この数字に左から順に番号付けすることを考える。そのときある数列$\{a_n\}$について、左から$n$番目の数値を$a_n$
                と表し、第$n$項とよぶ。特に$n=1$の一番初めの項$a_1$を初項という。
                数列には等差数列や階差数列があるがここでは詳しく述べない。

                項の数は有限でも無限でもよいので、項が無限にある数列$\{a_n\}$について考えてみることにする。$n$を限りなく大きくすると
                数列の値がある一定の値$a$に近づくときがある。この時数列$\{a_n\}$は\textbf{収束する}といい、近づく値$a$を\textbf{極限値}
                という。これを数式で表すと
                \begin{equation}
                    \lim_{n\to \infty}a_n=a
                \end{equation}
                新しく$\lim$という記号が出てきたが、これは$n$を限りなく大きくする(無限大に近づける)という操作を表す記号である。
                $a_n\to a \hspace{1mm}(n\to\infty)$と書いてもよい。このとき$a_n=a$となる$n$が存在する必要はない。
                
                \paragraph*{例}$\{a_n\}=1-\frac{1}{n}$について考える。$n$を限りなく大きくすると$\frac{1}{n}$は限りなく小さくなる($0$に近づく)ので
                $a_n$は$1-0=1$に近づく。よって$\{a_n\}$は収束し、極限値は$1$。\\

                始めのうちは値が近づく、と言われても何をどうすればよいかわからないであろうから、代入のような何かという風にとらえてもよい。実際極限のほとんどの操作は
                代入と結果的に等しくなる。気を付けないといけないは代入だと定義できない$\frac{1}{0}$などの場合である。\\

                極限の性質について以下にまとめる。$\displaystyle\lim_{n\to\infty}a_n=a,\lim_{n\to\infty}b_n=b,cは定数$とする。
                \begin{align}
                    \lim_{n\to\infty}(a_n\pm b_n)&=a\pm b\\
                    \lim_{n\to\infty}(c\cdot a_n)&=c\cdot a\\
                    \lim_{n\to\infty}(a_n\cdot b_n)&=a\cdot b\\
                    \lim_{n\to\infty}\frac{a_n}{b_n}&=\frac{a}{b}\quad(b_n\neq 0,b\neq 0)
                \end{align}

                収束するとは$n$を限りなく大きくしたときに数列が有限の値に近づくことだが、これはより厳密に言うことができる。
                \begin{itembox}{数列の収束の定義}
                    数列$\{a_n\}$が$a$に収束する$\Leftrightarrow $任意の$\varepsilon>0$が与えられたとき、それに対応してある$N$が\fbox{$n>N$のとき$|a-a_n|<\varepsilon$}となるように定められる。
                \end{itembox}
                正直一目見ただけでは全然意味がわからないはず。少しずつ理解していこう。この定義は二つに分けると見やすい。
                まず、任意の$\varepsilon$、つまりどんな$\varepsilon>0$に対しても、(収束するなら)対応する$N$\footnote{限定記号を使えば$\forall \varepsilon>0\exists N\in\mathbb{N}$となるので、$N=N(\varepsilon)$と書いてもよい。}が必ず見つかるということを言っている。
                ただ、このままだとなにをどう対応する$N$なのかがはっきりしない。そこで四角で囲った条件が必要になる。ざっくり言ってしまえば
                \underbar{正数$\varepsilon$がどんな値でも、四角で囲った条件を満たす$N$が見つかる}ということになる。

                次に四角で囲った条件について詳しく見ていこう。始めの条件$n>N$は一旦無視して、$|a-a_n|<\varepsilon$に注目する。
                この不等式の左辺が何を表すかを考えてみよう。数直線上に$a,a_n$をプロットすると、$|a-a_n|$はそのプロットした点と点
                との距離を表す。つまり不等式は、「近づく(であろう)値と$a_n$との距離をどんなに小さくとっても\footnote{$\varepsilon$は任意の数なのでどんなに小さくとってもよい。
                反対に大きくとることもできるが、$<\varepsilon$なので大きくとることに言及する意味はない。}」という意味になる。
                ここで飛ばした$n>N$についてみてみると、これは$n$が$\varepsilon$によって決まる$N$より大きい、という意味であるから、
                四角の条件をまとめると「近づく(であろう)値と$a_n$との距離をどんなに小さくとっても、その小さくとった幅に対応して$n$を大きくとれる」ということになる。\\

                とはいえ、これを文章で説明されても全然イメージがわかない。ということで実際に値をプロットしてみた。
                \begin{figure}[h]
                    \centering
                    \includegraphics[keepaspectratio,scale=0.45]{img/QuuNote/SequeanceLimitGraph.png}
                    \caption{$\{a_n\}=1-\frac{1}{n}$のグラフ}
                \end{figure}
                
                この数列で「$\varepsilon=0.7$を取るとき、$|a-a_n|=\frac{1}{n}<0.7\to n> 1=N$となるように$N$を取れば、$n>N$となる全ての$a_n$は
                図中の黒線と青線の間に入っている」が成り立っている。
                次の「$\varepsilon=0.3$を取るとき、$|a-a_n|=\frac{1}{n}<0.3\to n>3=N$となるように$N$を取れば、$n>N$となる全ての$a_n$は
                図中の黒線と青線の間に入っている」も成り立っている。

                このように「黒線との距離がどんなに小さな点線を考えても、ある番号$N$以上なら黒線とその小さい点線の間にプロットされる。そのような$N$が必ず見つかる。」というのが収束の定義の主張である。\\

                この収束の定義はとても難しい話なので、理解するのに時間がかかるかもしれない。(丁寧に説明したつもりけど、逆に回りくどくなってわかりにくいかも)そんな時は一旦飛ばすというのも手である。
                もちろんここで立ち止まって考えてもいいが一旦放置してあとから見直すとわかる、なんてこともざらにある。\\

                {\color{blue}$\diamondsuit$収束する数列はすべて有界\footnote{有界とは全ての$n$に対して$m\leq a_n \leq M$である定数$m,M$が存在すること。}である。\footnote{詳しい話は無限級数を扱うときに述べる。}}
            \clearpage
            \subsection{関数の極限}
                次の関数の極限について述べる。数列と違い、無限大以外に近づける場合も出てくる。ひとまず定義域$x\in I=(a,b)$である関数$f(x)$と
                定数$c\in I$を考えよう。$x$の値を$c$に限りなく近づけたとき、$f(x)$の値がある一定の値$C$に近づくとする。
                このとき
                \begin{equation}
                    \lim_{x\to c}f(x)=C
                \end{equation}
                と表し、$f(x)$は\textbf{収束する}という。また、$C$を$x$を$c$に近づけたときの$f(x)$の\textbf{極限値}という。記号の使い方は数列と同じなので馴染みやすい。
                もちろん$f(x)\to C\quad(x\to c)$という書き方もできる。
                
                \paragraph{例}$\displaystyle\lim_{x\to 2}x^2 = 4$ $x<2$の点からでも$x>2$の点からでも$2$に近づければ$f(x)=4$に近づく。これはグラフを見ても直感的にわかる。\\
                
                いま$c$は$f(x)$の定義域に含まれている状態で考えてみるが、実は含まれていなくてもよい。
                例えば、$f(x)=\frac{x^2-1}{x-1}$は$x=1$で定義出来ないが、$x\neq 1$で$f(x)=x+1$であるから$x\to 1$の
                極限を取ると値は$2$に近づく。このような例で極限と代入との違いがはっきりとわかる。\\

                いま近づけている値は有限の値を想定しているが、数列のように無限大(小)に大きくする極限も考えることが考えることができる。
                \begin{equation}
                    \lim_{x\to\infty}f(x)\quad \lim_{x\to-\infty}f(x)
                \end{equation}
                のように書く。例えば、$\displaystyle\lim_{x\to \pm\infty}\frac{1}{x}=0$。
                
                \noindent
                極限の性質の公式は、数列と同様であるためここでは述べない。\\

                さて、数列の極限と同様、関数の極限でもより厳密な定義について考えてみよう。それは以下のようになる。
                \begin{itembox}{関数の極限の定義}
                    $f(x)$が$x\to a$で$b$に収束する$\Leftrightarrow$任意の$\varepsilon>0$が与えられたとき、それに対応してある$\delta > 0$が\fbox{$0<|x-a|<\delta$のとき$|f(x)-b|<\varepsilon$}
                    となるように定められる。
                \end{itembox}
                これはいわゆる\textbf{$\varepsilon-\delta$論法}と呼ばれるもので、ぶっちゃけめっちゃ難しい。ただこれも落ち着いてみれば数列の極限の定義\footnote{$\varepsilon-N$論法という。}と似通っているところがあることに気づける。

                数列の極限との違いは$x$の範囲の制限にある。数列の場合は$x>N$だったが、関数の場合は$|x-a|<\delta$となっている。$N,\delta$の役割は同じなので今はただ記号を変えているだけと考えてよい。
                $|x-a|$は$x$と$a$との数直線上での差、つまり二つの点の距離を表しているので、$|x-a|<\delta$はその距離が$\delta$より小さいときというのを表している。
                あとは数列の場合と大体同じで、どんなに$f$と極限値が近づいていても($\varepsilon$を小さくしても)それに対応する$\delta$の値が定められる($x$が$a$に近づく)ということになる。
                \newpage

                $\varepsilon-\delta$論法を使って、実際に収束することを証明してみる。
                \paragraph*{例}$\displaystyle\lim_{x\to 2}x^2=4$を証明する。\\
                $0<|x-2|<\delta$とするとき、$|x^2-4|=|x-2|\cdot|x+2|=|x-2|\cdot|(x-2)+4|\leq |x-2|^2+4|x-2|<\delta^2+4\delta$
                であるため、$\varepsilon=\delta^2+4\delta\leftrightarrow \delta = -2+\sqrt{4+\varepsilon}$となるように$\delta$を取ればよい。このとき$0<|x-2|<\delta\rightarrow|x^2-4|<\varepsilon$を満たすので証明が終わる。$\square$\\

                試しに、$\varepsilon = 0.1$を代入すると$\delta \thickapprox 0.02484$であるため、$x=2.02483$のとき$|x^2-4|<\varepsilon$を満たすはずである。
                実際、$|x^2-4|=0.0999770256<0.1=\varepsilon$となっていて満たしている。今回は具体的に$\delta(\varepsilon)$を求めたが、このやり方にとらわれなくても四角で囲った条件を満たすように$\delta$が
                取れればよい。\\

                ここまで関数の収束について述べたが、ある一定の値に近づかずそのまま値が無限大に増大する場合などについて考えてみる。
                例えば、$x^3$は$x\to\infty$で$x^3\to\infty$である。このような場合\textbf{無限大に発散する}という。
                もちろん負の無限大に発散する場合も考えられる。一方で、$\sin x$は$x\to\infty$で値が無限大に増大するわけではないが、
                値が一つに定まることもない。このような場合は\textbf{振動する}という。\footnote{$\sin x$のグラフを見れば「振動する」という言い方がぴったりだとわかる。}\\

                次に重要な極限の公式を述べる。
                \begin{align}
                    \lim_{x\to 0}&\frac{\sin x}{x}=1\label{eq:limit of sin/x}\\
                    \lim_{x\to 0}&\left(1+x\right)^\frac{1}{x} = e\label{eq:define_e}
                \end{align}
                式\eqref{eq:define_e}はネイピア数の定義である。実際は左辺の極限が収束することを証明しないといけないが、ここでは割愛する。

                では式\eqref{eq:limit of sin/x}を証明する。
                \paragraph{証明}下図のような単位円を考える。
                \begin{figure}[h]
                    \centering
                    \includegraphics[keepaspectratio,scale=0.3]{img/QuuNote/circleFor_sin_div_xLimit.png}
                    \caption{単位円}
                \end{figure}

                このとき$\triangle OAC< OBC < \triangle OBD$である。それぞれ$\frac{1}{2}OA\cdot AC,\frac{1}{2}OC^2x,\frac{1}{2}OB\cdot BD$なので
                
                \begin{alignat}{3}
                    \frac{1}{2}OA\cdot AC&<& \frac{1}{2}OC^2x &<& \frac{1}{2}OB\cdot BD\notag\\
                    \frac{1}{2}OC^2 \sin x\cos x & < & \frac{1}{2}OC^2 x & < & \frac{1}{2}OC^2\tan x\notag\\
                    \sin x\cos x & < & x & < & \tan x\notag\\
                    \cos x & < & \frac{x}{\sin x} & < & \frac{1}{\cos x}\notag\\
                    \cos x & < & \frac{\sin x}{x} & < & \frac{1}{\cos x}\label{eq:Hasamiuti}
                \end{alignat}
                よって、$x\to 0$の極限を取れば$\displaystyle \cos x\to 1,\frac{1}{\cos x}\to 1$より$\displaystyle\frac{\sin x}{x}\to 1$となる。$\square$\\

                最後の不等式\eqref{eq:Hasamiuti}のような不等式のとき、両側の極限値が一致すれば、間に挟まれた極限値も等しくなる。これを\textbf{はさみうちの原理}という。
                はさみうちの原理ではうまく挟み込める不等式をつくる必要があるので、慣れるまで時間がかかる。

            \clearpage
            \subsection{極限の計算}
                この節では実際に極限の計算方法について学ぶ。単純な場合は代入と同様に計算してよいが$\frac{0}{0}$などの形になる場合は式を変形する必要がある。

                \paragraph{例1}次の極限を求めよ。
                    \begin{equation*}
                        \lim_{x\to \infty}\frac{x^3+5x^2+x+2}{x^3+x+10}
                    \end{equation*}
                    $\frac{1}{x}\to 0(x\to \infty)$の結果を利用する。分子と分母に$\frac{1}{x^3}$をかけて
                    \begin{equation*}
                        \lim_{x\to \infty}\frac{1+\frac{5}{x}+\frac{1}{x^2}+\frac{2}{x^3}}{1+\frac{1}{x^2}+\frac{10}{x^3}}=\frac{1+0+0+0}{1+0+0}=1
                    \end{equation*}
                
                \paragraph{例2}次の極限を求めよ。
                    \begin{equation*}
                        \lim_{x\to 0}\frac{x}{1-\sqrt{x+1}}
                    \end{equation*}
                    $(a-b)(a+b)=a^2-b^2$を用いる。分子と分母に$1+\sqrt{x+1}$をかけて
                    \begin{equation*}
                        \lim_{x\to 0}\frac{x(1+\sqrt{x+1})}{(1-\sqrt{x+1})(1+\sqrt{x+1})}=\lim_{x\to 0}\frac{x(1+\sqrt{x+1})}{-x}=-(1+\sqrt{0+1})=-2
                    \end{equation*}
                
                \paragraph{例3}次の極限を求めよ。
                    \begin{equation*}
                        \lim_{x\to\infty}\frac{2^x+1}{3^x}
                    \end{equation*}
                    $x$が十分大きいとき$2^x\ll 3^x$であるため、直感的に極限値は0だとわかる。
                    \begin{equation*}
                        \lim_{x\to\infty}\left(\left(\frac{2}{3}\right)^x+\frac{1}{3^x}\right)=\lim_{x\to\infty}\left(\frac{2}{3}\right)^x+\lim_{x\to\infty}\frac{1}{3^x}=0+0=0
                    \end{equation*}
                    二項目は指数関数の性質$a^x\quad(a<1)$の場合を用いた。三項目は$x\to\infty$のとき$3^x\to\infty$であることを用いた。

                \paragraph{例4}次の等式を証明せよ。
                    \begin{equation*}
                        \lim_{x\to \infty}\left(1+\frac{1}{x}\right)^x = \lim_{x\to 0}\left(1+x\right)^\frac{1}{x} 
                    \end{equation*}
                    $x=\frac{1}{t}$と置くと、$x\to\infty$で$t\to0$だから
                    \begin{equation*}
                        \lim_{x\to \infty}\left(1+\frac{1}{x}\right)^x = \lim_{t\to 0}\left(1+\frac{1}{\frac{1}{t}}\right)^\frac{1}{t}=\lim_{t\to\infty}(1+t)^\frac{1}{t}
                    \end{equation*}
                    よって等式が成り立つ。\\

                もちろんこれ以外にも極限の計算を行う際に用いるテクニックは存在するが、もう少し勉強を進めないと使うことができない。その時が来るまで楽しみにしていてほしい。
                なお、これらのテクニックのほとんどは数列の極限にも用いることができる。
            \clearpage
            \subsection{関数の連続}
                次に、関数の連続について考えていく。ひとまず定義から述べる。関数$f(x)$が$x=a$で\textbf{連続}であるとは、次の三つの条件を満たすことである。
                \begin{enumerate}
                    \item $f(a)$が定義されている。
                    \item $\displaystyle\lim_{x\to a}f(x)$が存在する。
                    \item $\displaystyle f(a)=\lim_{x\to a}f(x)$である。
                \end{enumerate}
                極限の場合は$x=a$で値が存在していなくてもよかったが、連続では$x=a$での値も必要となる。

                連続の条件2について、極限が存在するとはどういうことなのか考えてみよう。関数の極限$x\to a$では、\underline{$x$をどのように$a$
                に近づけても同じ極限値を取る}必要がある。
                どのように近づけても、と言われて困るかもしれないが、単に$x>a$の点と$x<a$の点から近づける場合を考えておけばよい。\footnote{二変数関数になると少し事情は変わる。`xy平面上のどの点から'近づけても同じになる必要がある。}

                このうち、$x>a$の点から近づける場合、すなわち数直線の右側から近づける場合を$\displaystyle\lim_{x\to a+0}f(x)$と表し、\textbf{右側極限値}と呼ぶ。
                同様に$x<a$の点から近づける場合は$\displaystyle\lim_{x\to a-0}f(x)$と表し、\textbf{左側極限値}と呼ぶ。この右側極限値と左側極限値が等しくなる時、
                極限は存在し、その値は
                \begin{equation}
                    \lim_{x\to a}f(x)=\lim_{x\to a+0}f(x)=\lim_{x\to a-0}f(x)
                \end{equation}
                となる。

                関数$f(x)$がある区間$I$で連続であるとき、$f(x)$は$I$で\textbf{連続関数}であるという。例えば、$f(x)=\sin x$は区間$(-\infty,\infty)$で連続関数である。
                一般に、初等関数は値が定義される(無限大にならないなど)全ての$x$について連続である。

                関数がある区間で連続であるといったが、そもそも区間の端での連続はどう定義すればよいだろうか。
                例えば、関数$f(x)=\sqrt{x}$は明らかに$x\geq 0$のすべての$x$で定義されているが$x=0$において連続の条件
                が適用できない。そこで、一般に関数が$x\geq a$で定義されているとき
                \begin{equation}
                    \lim_{x\to a+0}f(x)=f(a)
                \end{equation}
                が成り立てば、$x=a$において連続であるとする。こう定義することで、$\sqrt{x}$が区間$x\geq 0$で連続と定義できる。
                同様にして$x\leq b$である場合の端でも連続が定義される。この場合
                \begin{equation}
                    \lim_{x\to b-0}f(x)=f(b)
                \end{equation}
                のように左極限を取ることに注意。\\

                以下連続関数の性質について述べる。まず連続関数$f(x),g(x)$について
                \begin{equation}
                    f(x)\pm g(x)\quad f(x)g(x)\quad \frac{f(x)}{g(x)}
                \end{equation}
                は連続関数である。ただし、最後の式は$g(x)\neq 0$であるとする。このことから、連続関数の多項式も連続関数であることがわかる。\\

                \noindent
                例えば、$f(x)=x^n(n\in\mathbb{N})$は区間$(-\infty,\infty)$で連続であるため、多項式
                \begin{equation}
                    P(x)=a_nx^n+a_{n-1}x^{n-1}+\cdots+a_{1}x+a_0
                \end{equation}
                も区間$(-\infty,\infty)$で連続である。ほかにも、以下の\textbf{有理関数}
                \begin{equation}
                    R(x)=\frac{a_nx^n+a_{n-1}x^{n-1}+\cdots+a_{1}x+a_0}{b_nx^n+b_{n-1}x^{n-1}+\cdots+b_{1}x+b_0}
                \end{equation}
                も分母が0にならない限り連続である。

                また、次の\textbf{合成関数}$f(g(x))$を考えてみると、$f(x),g(x)$が連続関数であるかぎり$f(g(x))$も連続関数である。
                例えば$\log(\sqrt{x}+1)$は$x\geq 0$のすべての$x$で連続である。\\

                関数$f(x)$が区間$[a,b]$で連続関数である場合、次の二つが成り立つ。
                \begin{screen}
                    $f(a)\neq f(b)$なら$f(a)\leq k \leq f(b)$である任意の$k$について
                    \begin{equation*}
                        f(c)=k \label{eq:中間値の定理}
                    \end{equation*}
                    となる点$c\in[a,b]$が少なくとも一つ存在する。(\textbf{中間値の定理})
                \end{screen}
                \begin{screen}
                    $f(x)$は区間$[a,b]$で必ず最大値$M$と最小値$m$を取る。つまり
                    \begin{equation*}
                        m=f(x_m)\leq f(x)\leq f(x_M)=M
                    \end{equation*}
                    となる点$x_m,x_M\in[a,b]$が必ず存在する。
                \end{screen}
                文字だけだとわかりずらいが、グラフを見ればむしろ当たり前のことのように感じる。
                \begin{figure}[h]
                    \centering
                    \includegraphics[keepaspectratio,scale=0.3]{img/QuuNote/ContinuousFuncGraph.png}
                    \caption{連続関数}\label{fig:連続関数,中間値の定理,最大最小}
                \end{figure}

                図\ref{fig:連続関数,中間値の定理,最大最小}を見ると、区間$[1,3]$において関数は連続であり、その端での値
                おおよそ$-3$と$9$の間のすべて値に対して、対応する$x\in[-1,3]$が存在していることがわかる。これが中間値の定理
                の主張である。また区間における最大値と最小値も存在している(それぞれ$x=-1,3$の点)ことがわかる。
                ちなみに、点A,Bはそれぞれその周囲の点の間では最大・最小の値である。これらをそれぞれ極大値、極小値とよび、総称して
                \textbf{極値}という。
                \newpage
                また、中間値の定理において$f(a)$と$f(b)$の符号が異なる場合、方程式$f(x)=0$は
                区間$[a,b]$に実数解を少なくとも持つことがわかる。\footnote{注:実数解を(少なくとも一つ)もつことは中間値の定理からわかるが、実数解を持たないことは中間値の定理ではいえない。例えば$f(x)=x^2,[-1,2]$では$f(-1),f(2)>0$であるが区間内に実数解をもつ。}これは中間値の定理の応用である。
                \paragraph{例}方程式$x^3-3x-1=0$は区間$[0,2]$に少なくとも一つの実数解を持つかどうか答えよ。

                $f(0)=0-0-1=-1<0,f(2)=8-6-1=1>0$より、$f(0)$と$f(2)$の符号が異なるため方程式は区間$[0,2]$で少なくとも一つの実数解をもつ。\\

                いままでは連続関数の性質について述べたが、連続の条件が一つでも満たされていない場合についても考えてみよう。
                このとき関数$f(x)$は$x=a$で\textbf{不連続}であるという。例えば$\frac{1}{x}$は$x=0$で不連続である。
                一方で関数$\displaystyle g(x)=\frac{x^2-1}{x-1}$も$x=1$で不連続であるが、$x\neq1$では$g(x)=x+1$で連続である。
                そこで、
                \begin{equation}
                    g(x)=\left\{\begin{array}{lr}\displaystyle\frac{x^2-1}{x-1}&(x\neq 1)\\ x+1&(x=1)\end{array}\right.
                \end{equation}
                のように改めて定義しなおすことで、この関数は$x=1$で連続にできる。各自確かめてみよ。
            \clearpage
            \basicquestion 以下の問いに答えよ。

                \paragraph{問1}以下の数列の一般項を示し、それらが収束するかどうか答えよ。\\
                $(1)\{a_n\}=\frac{2}{1},\frac{3}{2},\frac{4}{3}\cdots$\hspace{3mm}
                $(2)\{b_n\}=\frac{1}{3},\frac{1}{9},\frac{1}{27}\cdots$\hspace{3mm}
                $(3)\{c_n\}=1,-1,1,-1\cdots$\\
                $(4)\{d_n\}=a,a+d,a+2d,a+3d\cdots(a,dは定数)$

                \paragraph{問2}以下を証明せよ。
                \begin{equation*}
                    \lim_{n\to\infty}(a_n+b_n)=a+b
                \end{equation*}
                ただし、$\displaystyle\lim_{n\to\infty}a_n=a,\lim_{n\to\infty}b_n=b$とする。{\scriptsize ヒント:$\varepsilon-N$論法を用いる。}

                \paragraph{問3}以下計算せよ。\\

                \noindent
                $(1)\displaystyle \lim_{x\to 0}\frac{\sin2x}{x}$\hspace{3mm}
                $(2)\displaystyle \lim_{x\to 2\pi}\sin \frac{x}{2}+x^2$\hspace{3mm}
                $(3)\displaystyle \lim_{x\to\infty}\frac{x^2+x+1}{x^3+1}$\hspace{3mm}
                $(4)\displaystyle \lim_{x\to 0}\frac{x^2}{1-\cos x}$\hspace{3mm}
                $(5)\displaystyle \lim_{x\to 0}\tan x$\hspace{3mm}
                $(6)\displaystyle \lim_{x\to\infty}\frac{2^x+3^x}{2^x-3^x}$\\
                $(7)\displaystyle \lim_{x\to 0}\frac{\sqrt[3]{8+x}-2}{x}$\hspace{3mm}
                $(8)\displaystyle \lim_{x\to 0}\frac{e^x-1}{x}$
                
                \paragraph{問4}次の関数が()内の点において連続であるかどうか調べよ。\\
                $(1)f(x)=x^2\quad(x=2)$\hspace{1mm}
                $(2)f(x)=\sin x\quad(x=\frac{\pi}{2})$\hspace{1mm}
                $(3)f(x)=\sqrt{1-x^2}\quad(x=1)$\hspace{1mm}
                $(4)f(x)=\frac{1}{\sqrt{x}}\quad(x=0)$\\
                $(5)f(x)=|x|\quad(x=0)$\hspace{1mm}
                $(6)\displaystyle f(x)=\left\{\begin{array}{lr}\displaystyle x\sin\frac{1}{x}&(x\neq 0)\\1&(x=0)\end{array}\right.$\\

                \paragraph{問5}方程式$\sin x=x$が区間$[0,\frac{\pi}{2}]$に実数解をもつかどうか調べよ。
            \clearpage
            \section{第I部演習問題}
                \paragraph{問1} 以下の計算をせよ。ただし$(2)$において$-1<x<1$とする。
                    \begin{alignat*}{9}
                        &[1]\log\sqrt{2+\sqrt{3}} &[2]&\sin^{-1}x+\cos^{-1}x &[3]&\cos\left(\frac{5\pi}{12}\right) & [4]&\log(e^{x^2}) & [5]&\sin(12\pi)\\
                        &[6]\lim_{x\to 0}x^n(n\in\mathbb{N}) &[7]&\lim_{n\to 0}x^n &[8]&\lim_{n\to 0}(\sqrt{n^2+3n}-n) &[9]&\lim_{n\to\infty}(\sqrt[3]{n+1}-\sqrt[3]{n}) & [10]&\lim_{x\to +0}x^x\\
                        &[11]\lim_{x\to +0}\frac{\sin x}{\sqrt{x}} &[12]& \lim_{x\to\infty}\frac{\sin(x)}{x} &[13]&\lim_{n\to\infty}\sum_{k=1}^{n}k\cdot\left(\frac{k}{n}\right) &[14]&\lim_{t\to 0}\frac{(x+t)^2-x^2}{t}
                    \end{alignat*}
                \paragraph{問2}次の式について以下の問いに答えよ。
                    \begin{equation}
                        \lim_{n\to n}a_n=a \Rightarrow \lim_{n\to\infty}\frac{a_1+a_2+\cdots+a_n}{n}=a
                    \end{equation}
                    \begin{description}
                        \item[(1\textrm{)}] 上式を証明せよ。
                        \item[(2\textrm{)}] $\log\left(\frac{2}{1}\times\frac{3}{2}\times\cdots\times\frac{n+1}{n}\right)-\log\frac{n+1}{n}$を求めよ。
                        \item[(3\textrm{)}] $\displaystyle \lim_{n\to\infty}\frac{\log n}{n}$を求めよ。
                    \end{description}
                
                \paragraph{問3}
                    次の関数が$[\quad]$の点で連続であるかどうか答えよ。
                    \begin{align*}
                        (1)&\sign x = \left\{\begin{array}{cc}\displaystyle 1 & (x>0) \\ 0 & (x=0) \\ -1 & (x<0)\end{array}\right.&\quad [x=0]\\
                        (2)&f(x)=\lim_{n\to \infty}f_n(x)\quad (f_n(x)=x^n\quad(0\leq x\leq 1))&\quad [x=1]
                    \end{align*} 
                    (1)の関数は\textbf{符号関数}という。${\rm sgn}(x)$と書く場合もある。

                \paragraph{問4}
                    数学において$0^0$は$1$と定義されたりそもそも定義されなかったりする。さて、$0^0=1$と定義する立場での根拠について数列$\{a_n\}=n^n$を用いて極限の観点から述べよ。
                
                \linktoMOKUZI
                
    \clearpage
    \part{微分法$f'$}
    \vspace{\stretch{1}}
    \begin{screen}
        いよいよ微分積分の``微分''の話に移る。微分法は、関数の挙動について解析するときに用いる。具体的な計算は公式に当てはめるだけなので
        そこまで難しくない。それにもかかわらず微分の応用例は幅広い。理屈がわかったらあとは練習あるのみである。
        また、この章ではテイラー展開などについても扱う。
    \end{screen}
    \clearpage
    \section{導関数}
        \subsection{平均変化率・微分係数}
            一次関数$y=\alpha x+b$の傾き$\alpha$を求める方法は直線上の二点の座標がわかればよく、それらを$(x_1,y_1),(x_2,y_2)$と置けば、
            \begin{equation}
                \alpha=\frac{y_2-y_1}{x_2-x_1}
            \end{equation}
            で求まる。また分子と分母はそれぞれ$(x_1,y_1)$からの$x$方向$y$方向の変化分だと考えられるので、それらを$\Delta x=x_2-x_1,\Delta y=y_2-y_1$と置けば
            \begin{equation}
                a= \frac{\Delta y}{\Delta x}
            \end{equation}
            となる。これを一般の関数$y=f(x)$に拡張することを考える。ここで注意しておいてほしいのは、
            一般の関数で考える際は$(x_1,y_1),(x_2,y_2)$の取り方によって傾きの値が変わってしまうことである。

            例えば、$y=x^2$について$(1,1^2)$から$(2,2^2)$の傾きは$\frac{4-1}{2-1}=3$だが、
            $(3,3^2)$から$(4,4^2)$の傾きは$\frac{16-9}{4-3}=7$となってしまう。つまり$x$の増分が同じであっても
            $y$の増分が同じであるとは限らないのである。

            とはいえ、傾きの式を$y=f(x)$の場合で拡張するからなにか定義の式が変わるわけではない。やはり二点の座標について
            \begin{equation}
                \frac{f(x_2)-f(x_1)}{x_2-x_1}\quad\left(=\frac{y_2-y_1}{x_2-x_1}\right)
            \end{equation}
            となる。この$x$の増分と$y$の増分の比のことを\textbf{平均変化率}という。
            \begin{figure}[h]
                \centering
                \includegraphics[scale=0.3]{img/QuuNote/heikinhenkaritu.png}
                \caption{平均変化率と直線}
            \end{figure}

            上図のように、平均変化率は二点を結んだ直線$l$の傾きを表している。また$a=x_1,b=x_2$と置き、$a$と$b$の差を$h=b-a$と置けば、
            \begin{equation}
                \frac{f(a+h)-f(a)}{h}
            \end{equation}
            と表すこともできる。
            
            次に点$x=b$を点$x=a$に限りなく近づける場合を考えよう。これは$a$と$b$との距離が限りなく小さくなることを意味するので$h\to 0$の極限である。すなわち
            \begin{equation}
                \lim_{h\to 0}\frac{f(a+h)-f(a)}{h}
            \end{equation}
            となる。この極限値が存在する場合、$f(x)$は$x=a$で\textbf{微分可能}であるという。また、その値を$x=a$における$f(x)$の\textbf{微分係数}といい、$f'(a)$と表す。
            \clearpage
            $h$は右側(正の側)から0に近づく場合と左側(負の側)から0に近づく場合がある。
            前者を\textbf{右方微分係数}といい
            \begin{equation}
                f'(a+0)=\lim_{h\to +0}\frac{f(a+h)-f(a)}{h}
            \end{equation}
            と表す。同様に後者を\textbf{左方微分係数}といい
            \begin{equation}
                f'(a-0)=\lim_{h\to -0}\frac{f(a+h)-f(a)}{h}
            \end{equation}
            と表す。微分可能とは$f'(a+0)$と$f'(a-0)$が存在して、$f'(a+0)=f'(a-0)$となることと同義である。
            もし$f(x)$が$x\geq a$で定義されている場合は、$x=a$における右方微分係数が存在していればよく、$x\leq a$で定義されている場合は左方微分係数が存在していればよい。
            \\

            では次に微分係数の幾何学的な意味について考えていこう。そのためには$b$を$a$に徐々に近づけた場合の$a-b$を結ぶ直線を書くとわかりやすい。
            \begin{figure}[h]
                \centering
                \includegraphics[scale=0.5]{img/QuuNote/differentialCoefficieant_Graph.png}
                \caption{点$b$を徐々に近づけた場合の直線の変化}
            \end{figure}

            この図を見ると$b$が$a$に近づくにつれて、二点を結んだ直線も点$a$の\textbf{接線}に近づいていることがわかる。
            つまり$x=a$における微分係数は点$a$の接線の傾きを表している。
        \clearpage
        \subsection{微分可能性と連続性}
            今度は関数が$f(x)$が$x=a$で微分可能であることと連続であることの違いについて考えていこう。
            一見するとこれら二つは同値であるかのように見える。\footnote{昔の数学者たちの間でも長らく連続関数は明らかに微分可能と考えられていたらしい。だからこそワイエルシュトラスの$W(x)$関数
            のような連続なのにいたるところで微分不可能な関数が発見されたときは、数学界に大きな衝撃を与えたそうだ。\\参考書籍:\url{https://www.iwanami.co.jp/book/b480065.html}}
            実際初等関数は連続である区間についてすべて微分可能である。では初等関数ではない関数である$y=|x|$で考えてみよう。
            もちろん$x>0$では$y=x$、$x<0$では$y=-x$であるので、微分可能である。(実際に試すとよい)
            しかし$x=0$においては微分可能ではない。\footnote{幾何学的に言えば接線が二本引けてしまうことが理由となる。つまり微分可能性とはその点においてただ一つ接線が引けることと理解できる。\label{微分可能性の幾何学的意味}}それを今から示す。

            微分可能であることは、右方微分係数と左方微分係数が一致すればよいことであるが
            \begin{equation}
                \frac{|0+h|-|0|}{h}=\frac{|h|}{h}
            \end{equation}
            なので、
            \begin{align}
                y'(+0)&=\lim_{h\to +0}\frac{|h|}{h}=\lim_{h\to+0}\frac{h}{h}=1\\
                y'(-0)&=\lim_{h\to -0}\frac{|h|}{h}=\lim_{h\to -0}\frac{-h}{h}=-1
            \end{align}
            となり右方微分係数と左方微分係数が一致しない。したがって$y=|x|$は$x=0$で微分可能ではないので、
            $y=|x|$は連続であるが微分可能でない関数であることがわかる。

            では逆に$f(x)$が$x=a$で微分可能であるときはどうであろうか。$f(x)$が$x=a$で微分可能であるとき
            \begin{equation}
                \lim_{h\to 0}(f(a+h)-f(a))=\lim_{h\to 0}\frac{f(a+h)-f(a)}{h}\cdot h=\lim_{h\to 0}f'(a)\cdot h=0 
            \end{equation}
            また、$\displaystyle \lim_{h\to 0}(f(a+h)-f(a))=\lim_{h\to 0}f(a+h)-\lim_{h\to 0}f(a)=\lim_{h\to 0}f(a+h)-f(a)$なので、
            \begin{equation}
                \lim_{h\to 0}f(a+h)=f(a)
            \end{equation}
            が成り立つ。$h=b-a$であったことを思い出すと、$h\to 0$のとき$b\to a$なので
            \begin{equation}
                \lim_{b\to a}f(b) = f(a)
            \end{equation}
            つまり$x=a$で$f(x)$は連続の条件を満たす。\\\\
            \noindent
            以上をまとめると\fbox{$f(x)$が$x=a$で微分可能$\Rightarrow$$f(x)$は$x=a$で連続}である。 \\

            ちなみに、$x=a$で$f(x)$が不連続である場合は、対偶を取ればわかるように微分可能ではない。これは微分係数の定義からもわかる。
        \clearpage
        \subsection{導関数の定義}
            関数$f(x)$の微分係数は$f'(a)$であり、これは$x=a$の点において$y=f(x)$上の接線の傾きを意味していることは前回学んだ。
            では次に任意の$x$の値に対して$f(x)$の接線の傾きを返す関数を考えてみよう。この間数は以下のように定義できる。
            \begin{equation}
                f'(x)=\lim_{h\to 0}\frac{f(x+h)-f(x)}{h}
            \end{equation}
            このような関数$f'(x)$を$f(x)$の\textbf{導関数}という。$x=a$のときの導関数の値が$f'(a)$であり、これは$x=a$の微分係数である。

            導関数には以下のような書き方がある。
            \begin{equation}
                \frac{dy}{dx},\quad\frac{df}{dx},\quad y',\quad f'(x),\quad\frac{d}{dx}y,\quad\frac{d}{dx}f(x),\quad D_xf(x),\quad Df(x)
            \end{equation}
            このうち$\frac{d}{dx}$\footnote{読み方は分子から。「でぃーわいでぃーえっくす」など。}の書き方はライプニッツ、$f',y'$の書き方はラグランジュによるものである。$D_x,D$はコーシーによる。
            ライプニッツの書き方は導関数の意味が明瞭であるが書く際に場所を取ってしまったりするので簡潔なラグランジュの書き方も使われる。(コーシーの書き方は演算子であることが明瞭である気がする。)このノートはどちらもその
            場合に応じて使い分けていく。
            ちなみに、変数が時間である場合など、物理関係では$\dot{x}(t)$などとして表すこともある。これはニュートンの書き方である。\\

            試しに、$f(x)=x$の導関数を求めてみる。
            \begin{equation}
                f'(x)=\lim_{h\to 0}\frac{f(x+h)-f(x)}{h}=\lim_{h\to 0}\frac{(x+h)-x}{h}=1
            \end{equation}
            したがって、$f(x)=x$の導関数は$f'(x)=1$であることがわかる。これは定数関数であるので$x$の値によらない。
            つまり$y=x$はどの点でも傾きが同じであることがわかる。実際グラフを想像すれば直線で傾きは一定である。\\

            一般に$f,g$が微分可能、$a=定数$であるとき以下の公式が成り立つ。
            \begin{alignat}{3}
                &(a)' &&= 0\\
                &(a f(x))' &&= af'(x)\\
                &(f(x)\pm g(x))' &&=f'(x)\pm g'(x)
            \end{alignat}
            どれも定義からすぐに導出できる。気になる人は試してみてほしい。
        \clearpage
        \basicquestion 以下の問いに答えよ。

            \paragraph{問1}次の関数の(\hspace{2mm})内の区間での平均変化率を求めよ。
            \begin{alignat*}{2}
                &(1)y=x^2+1 &&(x=1\to 3)\\
                &(2)y=x^3+x^2+x+1 &&(x=-1\to 1)\\
                &(3)y=\sin x &&(x=\frac{\pi}{6}\to\frac{\pi}{3})
            \end{alignat*}

            \paragraph{問2}次の関数を微分せよ。\\
            \noindent
            $(1)y=x+1$\hspace{5mm}
            $(2)y=5$\hspace{5mm}
            $(3)y=x^3$\hspace{5mm}
            $(4)y=ax+b$\hspace{5mm}
            $(5)y=a(x+p)^2+q$

            
            \paragraph{問3}$y=\sqrt{x}\hspace{1mm}(x\geq 0)$が微分可能であるか調べよ。

            \paragraph{問4}以下の式を証明せよ。
            \begin{equation*}
                \frac{d}{dx}(af(x))=a\frac{d}{dx}f(x)
            \end{equation*}
            ただし、$f(x)$が微分可能とする。
        \clearpage
        \section{導関数の計算}
            \subsection{基礎的な関数の導関数}
                \noindent
                ここでは初等関数の基礎的な関数$x^n,\sin x,e^x,\log x$などの導関数を求めていく。\\

                まず、$e^x$から考えてみよう。ひとまず定義に従えば
                \begin{equation}
                    \lim_{h\to 0}\frac{e^{x+h}-e^{x}}{h}=e^x\lim_{h\to 0}\frac{e^h-1}{h}
                \end{equation}
                と変形できる。そこで、出てきた極限について逆数をとり$t=e^h-1$とおいて計算していく。
                \begin{equation}
                    \lim_{h\to 0}\frac{h}{e^h-1}=\lim_{t\to 0}\frac{\log(t+1)}{t}=\lim_{t\to 0}\log(1+t)^\frac{1}{t}
                \end{equation}
                と変形できるので、$e$の定義式\eqref{eq:define_e}より真数は$e$となる。したがってこの極限は$1$となるので
                \begin{equation}
                    (e^x)'=e^x \cdot \lim_{h\to 0}\frac{e^h-1}{h}=e^x \cdot 1 =e^x
                \end{equation}
                つまり$e^x$は\underline{微分しても形が変わらない}のである。
                また、この結果を使えば一般の指数関数$a^x$の導関数も簡単に求まる。$(e^{ax})'=ae^{ax}$が成り立つ(演習問題)ことから
                \begin{equation}
                    (a^x)'=(e^{\log(a^x)})'=(e^{x\log a})'=e^{x\log a}\log a=a^x\log a
                \end{equation}

                つぎに、$\log x$について考えてみる。これも定義に従って
                \begin{equation}
                    \lim_{h\to 0}\frac{\log(x+h)-\log x}{h}=\lim_{h\to 0}\frac{\log\left(\frac{x+h}{x}\right)}{h}=\lim_{h\to 0}\log\left(1+\frac{h}{x}\right)^\frac{1}{h}
                \end{equation}
                ここで$t=\frac{h}{x}$と置けば
                \begin{equation}
                    \lim_{t \to 0}\log(1+t)^\frac{1}{xt}=\lim_{t\to 0}\frac{\log(1+t)^\frac{1}{t}}{x}=\frac{\log(e)}{x}=\frac{1}{x}
                \end{equation}
                と求まる。

                次に、$x^n$で求めてみよう。ただし$n$は自然数であるとする。一般に、
                \begin{equation}
                    (x+a)^n=x^n+{}_nC_1ax^{n-1}+{}_nC_2a^2 x^{n-2}+\cdots +{}_nC_ma^{m}x^{n-m}+\cdots {}_nC_1a^{n-1}x+a^n\quad(\text{\textbf{二項定理}})\label{eq:二項定理}
                \end{equation}
                が成り立つので
                \begin{alignat}{1}
                    (x^n)'&=\lim_{h\to 0}\frac{(x+h)^n-x^n}{h}=\lim_{h\to 0}\frac{{}_nC_1hx^{n-1}+{}_nC_2h^2 x^{n-2}+\cdots +{}_nC_mh^{m}x^{n-m}+\cdots +{}_nC_1h^{n-1}x+h^n}{h}\\
                    &=\lim_{h\to 0} {}_nC_1 x^{n-1}+(h\text{についての多項式})=nx^{n-1}
                \end{alignat}
                要するに$x^n$の微分は\underline{肩をおろして1を引く}のである。
                \clearpage
                最後に$\sin x$の導関数を求めてみよう。こちらも定義通りに計算して
                \begin{equation}
                    (\sin x)'=\lim_{h\to 0}\frac{\sin(x+h)-\sin x}{h}=\lim_{h\to 0}\frac{2\cos(\frac{2x+h}{2})\sin(\frac{h}{2})}{h}=\lim_{h\to 0}\cos (x+\frac{h}{2})\cdot \lim_{h\to 0}\frac{\sin\frac{h}{2}}{\frac{h}{2}}=\cos x
                \end{equation}
                最後の極限は式\eqref{eq:limit of sin/x}を利用した。\\

                これで基礎となる関数の導関数はほとんど導出できた。以下公式としてまとめる。一部導出していないものもあるが演習問題で導出する。これくらいは後々の計算のためにも覚えておくとよい。

                \begin{screen}
                    \begin{alignat*}{2}
                        &(x^n)'=nx^{n-1} && (a^x)'=a^x\log a\\
                        &(e^x)'=e^x && (\log x)'=\frac{1}{x}\\
                        &(\sin x)'=\cos x \quad&& (\cos x)'=-\sin x
                    \end{alignat*}
                \end{screen}
            \clearpage
            \subsection{積の微分}
                今度は 関数同士の積について、公式を導出してみよう。$f,g$は微分可能であるとする。このとき、
                \begin{equation}
                    (f(x)g(x))'=\lim_{h\to 0}\frac{f(x+h)g(x+h)-f(x)g(x)}{h}
                \end{equation}
                である。当然このままでは計算できないので少し式をいじってあげよう。仮定より分子にどうにかして$f(x+h)-f(x)$もしくは$g(x+h)-g(x)$
                を作れないか考えてみる。そこで分子に$-f(x+h)g(x)+f(x+h)g(x)=0$を加えてあげると
                \begin{equation}
                    \frac{f(x+h)(g(x+h)-g(x))+g(x)(f(x+h)-f(x))}{h}=\frac{f(x+h)(g(x+h)-g(x))}{h}+\frac{g(x)(f(x+h)-f(x))}{h}
                \end{equation}
                とできる。あとは$f,g$が微分可能であるので$h\to  0$の極限を取れば
                \begin{equation}
                    (f(x)g(x))'=f(x)g'(x)+f'(x)g(x)\label{eq:積の微分公式}
                \end{equation}
                こうして積の微分公式が得られた。では、この公式を用いて$\displaystyle y=x^{-n}=\frac{1}{x^{n}}$の導関数も求めてみよう。
                \begin{equation}
                    (1)'=\left(\frac{x^n}{x^{n}}\right)'=nx^{n-1}\cdot \frac{1}{x^{n}}+x^n\cdot \left(\frac{1}{x^n}\right)'=\frac{n}{x}+x^{n}y'=0
                \end{equation}
                であるため、$y'=$の形に整理すれば
                \begin{equation}
                    y'=\frac{-n}{x^{n+1}}=(-n)\cdot x^{(-n)-1}\label{eq:積微分による1/x^nの微分}
                \end{equation}
                つまり、任意の整数について$(x^n)'=nx^{n-1}$が成り立つことが示せた。\footnote{この例は積の微分公式を使う練習としてはあまり適していない。商の微分を用いるか合成関数の微分法を用いることで見通しよく求められる。}\\

                もう少し簡単な問題で積の微分公式を使う練習をしてみよう。
                \paragraph{例題}$y=x\sin x$のとき$y'$を求めよ。\\
                $f=x,g=\sin x$とおいて式\eqref{eq:積の微分公式}を適応すると$f'=1,g'=\cos x$だから
                \begin{equation}
                    y'=fg'+f'g=x\cos x+\sin x
                \end{equation}
                と求まる。
            \clearpage
            \subsection{商の微分}
                積の微分について考えたら次は商の微分についても考えたくなるものである。商の微分公式について以下にのべる。
                \begin{equation}
                    \left(\frac{f(x)}{g(x)}\right)'=\frac{f'(x)g(x)-f(x)g'(x)}{\left\{g(x)\right\}^2}\label{eq:商の微分公式}
                \end{equation}
                もちろん$f,g$は微分可能であり$g(x)\neq 0$であるとする。頭の痛くなる形をしていて、覚えるのに苦労しそうである。
                覚え方は人によってさまざまだろうが、例えば「分子は積の微分の符号反転で分母は二乗する」という覚え方もある。導出については
                演習問題とする。

                商の微分について$f=1$の特別な場合を知っておくと計算が早くなることがある。試しに公式に代入してみると
                \begin{equation}
                    \left(\frac{1}{g}\right)'=\frac{0\cdot g-1\cdot g'}{g^2}=-\frac{g'}{g^2}
                \end{equation}
                となる。最後の符号を忘れないように注意しよう。ここまで覚える必要はないが知っておけば多少楽できる。\\

                ではこれを使って$y=x^{-n}$の導関数を導出してみよう。$f=1,g=x^n$として公式に代入すればよい。
                \begin{equation}
                    y'=-\frac{(x^n)'}{(x^n)^2}=-\frac{nx^{n-1}}{(x^n)^2}=-\frac{n}{x^{n+1}}=(-n)\cdot x^{(-n)-1}
                \end{equation}
                となり、式\eqref{eq:積微分による1/x^nの微分}と同じ結果が得られた。こちらの方が自然な導出である。

                次に$y=\tan x$の導関数を導出してみよう。この関数は$\sin x$などと違って定義から計算すると骨が折れる。しかし、商の微分公式を用いれば
                \begin{equation}
                    y'=\left(\frac{\sin x}{\cos x}\right)'=\frac{\cos^2 x-(-\sin^2 x)}{\cos^2 x}=\frac{1}{\cos^2 x}
                \end{equation}
                と楽に導出できる。
            \clearpage
            \subsection{合成関数の微分}
                では次に合成関数の微分について考えてみる。$y=f(g(x))$として考えてみる。このとき
                \begin{equation}
                    y'=\lim_{h\to 0}\frac{f(g(x+h))-f(g(x))}{h}=\lim_{h\to 0}\frac{f(g(x+h))-f(g(x))}{g(x+h)-g(x)}\cdot \frac{g(x+h)-g(x)}{h}
                \end{equation}
                であるので、$h'=g(x+h)-g(x)$と置けば、$h\to 0$で$h'\to0$なので、
                \begin{equation}
                    \lim_{h'\to 0}\frac{f(g(x)+h')-f(g(x))}{h'}\cdot \lim_{h\to 0}\frac{g(x+h)-g(x)}{h}=f'(g(x))\cdot g'(x)
                \end{equation}
                となる。$u=g(x)$とおいてこれをライプニッツの記号で書けば
                \begin{equation}
                    \frac{dy}{dx}=\frac{dy}{du}\frac{du}{dx}\label{eq:合成関数の微分}
                \end{equation}
                こうすれば`形式的に'約分しているように見ることができる。ちなみにこの合成関数の微分の公式を\textbf{チェイン・ルール}という。
                
                合成関数の微分公式を用いれば一般の$m$について$(x^m)'=mx^{m-1}$が証明できる。それを今から示す。\footnote{これは対数微分法でも示せる。そもそも対数微分法も今回のやり方も本質的には同じである。}
                \begin{equation}
                    (x^m)'=(e^{\log x^m})'=(e^{m\log x})'=e^{m\log x}\cdot (m\log x)'=x^{m}\cdot \frac{m}{x}
                \end{equation}
                と計算できるので、
                \begin{equation}
                    (x^m)'=mx^{m-1}
                \end{equation}
                
                ほかにも$(2x+1)^5$を微分するとき、通常なら展開して項別微分する必要があるが、合成関数を用いれば
                \begin{equation}
                    \left\{(2x+1)^5\right\}'=5(2x+1)^4\cdot (2x+1)'=10(2x+1)^4
                \end{equation}
                と非常に簡単に求められる。より一般的には
                \begin{equation}
                    (f(ax+b))'=af'(ax+b)
                \end{equation}
                である。
            \clearpage
            \subsection{逆関数の微分}
                前回までで和・差・積・商のすべての場合について、微分の公式を導入した。これさえあればどんな関数でも微分できそうであるが、実はそうではない。
                例えば$\arcsin x$は、$\sin x$の逆関数であること以外何も関数についてわかっていない。このような関数の微分について考えよう。

                まず、一価単調連続関数$f(x)$について、逆関数を$y=f^{-1}(x)$と表すことにすれば、導関数の定義より
                \begin{equation}
                    \lim_{h\to 0}\frac{f^{-1}(x+h)-f^{-1}(x)}{h}
                \end{equation}
                ここで$y=f^{-1}(x)$と置くと、$x=f(y)$である。また、$Y=f^{-1}(x+h)$と置けば$x+h=f(Y)$である。したがって$h=f(Y)-f(y)$である。$h\to 0$のとき$Y\to y$だから$h'=Y-y$と置けば$h'\to 0$である。
                \begin{equation}
                    \lim_{h'\to 0}\frac{h'}{f(y+h')-f(y)}=\lim_{h'\to 0}\frac{1}{\displaystyle\frac{f(y+h')-f(y)}{h'}}
                \end{equation}
                ここで$f(x)$が微分可能であるとすれば、この極限は収束して
                \begin{equation}
                    \frac{dy}{dx}=\frac{1}{\displaystyle\frac{dx}{dy}}\label{eq:逆関数の微分公式}
                \end{equation}
                となる。これが逆関数の微分公式である。形式的には$dy,dx$を一つの量とみなして変形した形と一致する。\\

                ではこの公式を用いて件の$\arcsin x$を求めてみることにしよう。$y=\arcsin x$とすれば当然$x=\sin y$であるから、
                \begin{equation}
                    \frac{dy}{dx}=\frac{1}{\displaystyle\frac{dx}{dy}}=\frac{1}{\displaystyle\frac{d}{dy}\sin y}=\frac{1}{\cos y}=\frac{1}{\sqrt{1-\sin^2 y}}=\frac{1}{\sqrt{1-x^2}}
                \end{equation}
                と求まった。つぎに$y=\log x$について、公式を適用してみよう。逆関数が$e^x$であることに注意すれば
                \begin{equation}
                    \frac{dy}{dx}=\frac{1}{\frac{d}{dy}e^{y}}=\frac{1}{e^y}=\frac{1}{x}
                \end{equation}
                となり、公式が正しいことも確認できる。
            \clearpage
            \subsection{微分}
                ここでは微分という概念について述べる。突然だが微分可能な関数$f$について$y=f(x)$のグラフを考えてみる。この時グラフ上の点$(x,y)$
                は、必ず接線を持つはずである。\footref{微分可能性の幾何学的意味}このとき、その接線上の点$(X,Y)$について$dx=X-x,dy=Y-y$となるような
                座標の変動$dx,dy$を定義してあげると
                \begin{equation}
                    dy=f'(x)dx
                \end{equation}
                は、点$(x,y)$の接線の方程式と全く同じものであることがわかる。この時$dx=\Delta x$とすると、
                グラフ上の座標の変動$\Delta y$は必ずしも$dy$と等しくはならない。このような$dx,dy$をそれぞれ$x,y$の\textbf{微分}という。\\

                これらの意味するところについて述べたいのはやまやまだが、ここでは記号的な話にのみ焦点を当てる。形式的な話であるので得られた結果が正しい可能性はないし、
                そもそもここからすべての導関数が導出できるわけではないだろうが、便利なので紹介する。

                まずは、演算の規則について、次の三つがある。
                \begin{enumerate}
                    \item $dy=f'(x)dx$
                    \item $d(u\pm v)=du\pm dv$
                    \item $d(uv)=vdu+udv$
                \end{enumerate}
                
                これを用いて一つ導関数を求めてみよう。例えば陰関数$F(x,y)=x^2+y^2-r^2=0$について、$\frac{dy}{dx}$を求めてみることにする。まず、$x,y$を両辺に分け、それぞれ$d$(ディー)して
                \begin{equation}
                    d(x^2-r^2)=d(y^2)
                \end{equation}
                演算の規則2より、$d(x^2-r^2)=d(x^2)-d(r^2)$と分けられる。この時、$r$は$x,y$によらないと考えているので$d(r^2)=0$である。よって、
                \begin{equation}
                    2xdx=2ydy
                \end{equation}
                式を変形すると、
                \begin{equation}
                    \frac{dy}{dx}=\frac{x}{y}
                \end{equation}
                となる。まるで魔法にかけられたみたいにあっさり求まってしまったが、この結果は正しいのだろうか?試しに$y>0$として元の式を変形すると$y=\sqrt{r^2-x^2}$となる。
                このとき、$x$微分すると
                \begin{equation}
                    \frac{dy}{dx}=\frac{x}{\sqrt{r^2-x^2}}=\frac{x}{y}
                \end{equation}       
                と確かに一致することがわかる。        
            \clearpage
            \basicquestion 以下の問いに答えよ。
            
                \paragraph{問1}以下の関数の導関数を求めよ。\\
                $(1)y=x^2+1$\hspace{3mm}
                $(2)y=\cos x+\sin x$\hspace{3mm}
                $(3)y=\log 2x$\hspace{3mm}
                $(4)y=\cos^{-1}x$\hspace{3mm}
                $(5)y=e^{\tan x}$\hspace{3mm}
                $(6)y=e^{x^2}\cos 2x$\\
                $\displaystyle(7)y=\log(x+\sqrt{x^2+1})$\hspace{3mm}
                $(8)y=f^{-1}(x)\quad (f(x)=x^3)$

                \paragraph{問2}以下を示せ。\\
                (1)商の微分公式\eqref{eq:商の微分公式}を示せ。\\
                (2)$(e^{ax})'=ae^{ax}$を導関数の定義から導出せよ。\\
                (3)$\cos x$の導関数を$\sin^2 x+\cos ^2 x=1$の関係を利用して導出せよ。ただし$\cos x\neq 0$であるとする。

                \paragraph{問3}以下の関数の導関数を工夫して求めよ。\\{\scriptsize ヒント:(1)(2)両辺対数を取るか$e^{\log f(x)}$の形にせよ。(3)普通に解いてもよいが$x=\tan \theta$と置いて合成関数の微分公式を用いると楽に出せる。}\\
                $\displaystyle(1)y=\sqrt[3]{\frac{x^2+1}{(x+1)^2}}\quad(x>-1)$\hspace{3mm}
                $(2)y=x^x$\hspace{3mm}
                $\displaystyle(3)y=\sin^{-1}\frac{x}{\sqrt{1+x^2}}$

                \paragraph{問4}双曲線関数について、$\sinh x,\cosh x$の導関数を導出せよ。

                \paragraph{問5}以下の公式について、後の問いに答えよ。
                \begin{eqnarray}
                    \sin\alpha\cos\beta=\frac{1}{2}\left\{\sin(\alpha+\beta)+\sin(\alpha-\beta)\right\}\label{eq:和積の公式sinαcosβ}
                \end{eqnarray}
                \begin{enumerate}\setcounter{enumi}{0}\renewcommand{\labelenumi}{(\arabic{enumi})}
                    \item 公式\eqref{eq:和積の公式sinαcosβ}を示せ。
                    \item 公式\eqref{eq:和積の公式sinαcosβ}の両辺を$\alpha$微分せよ。
                    \item 公式\eqref{eq:和積の公式sinαcosβ}の両辺を$\beta$微分せよ。
                \end{enumerate}
                
            \clearpage
            \section{微分法の応用}
                \subsection{関数の増減}
                    この章からは微分法の応用について述べる。まずは、微分法を用いて関数の増減について調べていこう。ひとまず
                    微分係数が何を示しているかを復習しておくと、これは$y=f(x)$のグラフにおいてその点の接線の傾きを示しているのだった。
                    傾きの大きさ(度合い)は一旦無視してその符号にのみ着目すると、傾きが正であるとき(一次関数のグラフと同様)その点周辺で$f(x)$は増加しており
                    逆に負の場合は減少していることがわかる。すなわち次の結果が得られる。
                    \begin{screen}
                        ある$x$の区間$I$について
                        \begin{itemize}
                            \item $f'(x)>0$である場合は$y=f(x)$は区間$I$で\underbar{単調増加}
                            \item $f'(x)<0$である場合は$y=f(x)$は区間$I$で\underbar{単調減少}
                        \end{itemize}
                        が成り立つ。
                    \end{screen}
                    例えば、$y=x^2$は$(x^2)'=2x$なので、$x>0$で単調増加しており$x<0$で単調減少している。

                    次に、グラフが単調増加から単調減少もとい単調減少から単調増加に変わる点について考察をしていこう。
                    ひとまず$y=f(x)$が$x=a$で単調増加から単調減少に変わった場合を考えてみる。この時微小な$\varepsilon>0$
                    を用いて$f'(a-\varepsilon)>0,f'(a+\varepsilon)<0$と書け、$x=a$で符号が変わっていることが想像できる。
                    符号が変わる点は$0$であることと同じであるため、$x=a$について$f'(a)=0$だとわかる。同様に単調減少から単調増加
                    に変わる場合も同じく$f'(a)=0$となる。また、$f'(a-\varepsilon)>0,f'(a+\varepsilon)<0$は点$a$の周り$[a-\varepsilon,a+\varepsilon]$
                    において$f(a)$が最大値であることを意味する。同様に、$f'(a-\varepsilon)<0,f'(a+\varepsilon)>0$の場合も最小値であることを意味する。
                    このような値を総称して\textbf{極値}と呼ぶことは第一部で学んだ。以上をまとめると、次の結果が得られる。
                    \begin{screen}\centerline{
                        $f(a)$が極値$\Rightarrow f'(a)=0$}
                    \end{screen}
                    このとき逆は成り立たないことに注意しなければならない。例えば、$y=x^4-x^3$は$y'=4x^3-3x^2=x^2(4x-3)$となるため、$x=0,\frac{3}{4}$で極値を取りそうである。
                    実際、$y'(\frac{3}{4}-\varepsilon)<0,y'(\frac{3}{4}+\varepsilon)>0$であるため、$x=\frac{3}{4}$では極値を取る。
                    しかし、$y'(0-\varepsilon)<0,y'(0+\varepsilon)<0$であるため、$x=0$では極値を取っていない!その前後の微分係数の符号が変わっているとき
                    を極値というわけである。逆に対偶を取れば、$f'(a)\neq 0$で$f(a)$が極値を取ることもない。そのため、$f'(a)=0$となる$x=a$を調べれば極値の``候補''がわかると考えるとよいのである。
                    \clearpage
                    関数の増減を調べると、グラフを書くことが容易になる。その際表の形にまとめて置くとグラフを書きやすい。
                    \begin{table}[h]
                        \centering
                        \begin{tabular}{|c||c|c|c|c|c|}\hline
                            $x$ & $\cdots$ & $0$ & $\cdots$ & $\frac{3}{4}$ & $\cdots$ \\\hline
                            $f'(x)$ & $-$ & $0$ & $-$ & $0$ & $+$ \\\hline
                            $f(x)$ & $\searrow $ & $0$ & $\searrow$ & $-\frac{27}{256}$ & $\nearrow$ \\\hline
                        \end{tabular}
                        \caption{$f(x)=x^4-x^3$の増減表}\label{ta:f(x)=x^4-x^3の増減表}
                    \end{table}

                    実際のグラフは以下のようになる。
                    表\ref{ta:f(x)=x^4-x^3の増減表}の矢印とグラフの増減が一致していることがわかる。
                    \begin{figure}[h]
                        \centering
                        \includegraphics[scale=0.4]{img/QuuNote/x^4-x^3Graph.png}
                        \caption{$f(x)=x^4-x^3$のグラフ}\label{fig:x^4-x^3グラフ}
                    \end{figure}
                \clearpage
                \subsection{関数の変曲点}
                    さて、図\ref{fig:x^4-x^3グラフ}に着目すると$x=0$の前後で同じ単調減少でもすこし形が異なる様子が見られる。
                    $x<0$ではグラフの形が\textbf{下に凸}になっているが、$0<x<\frac{3}{4}$では\textbf{上に凸}になっている。
                    この違いは何なのだろうか。結論から述べるとこれは導関数の増減の変化から生じたものである。
                    例えば、$f'(x)$の導関数を求めると$f''(x)=12x^2-6x$であり$x=0$で$f''=0$である。つまり$x=0$で$f'(x)$の増減の仕方が変化した
                    わけである。$x<0$では$f''>0$なので$f'$は単調増加する、すなわち$f(x)$の減少の割合が小さくなっていくのである。これは下に凸にあたる。
                    一方$0<x(<\frac{1}{2})$では$f'$は単調減少する、つまり$f(x)$の減少の割合が大きくなっていくのである。これは上に凸にあたる。
                    ところで$f''$は$x=\frac{1}{2}$でも$0$になる。つまり$x>\frac{1}{2}$で$f''>0$なのでこのとき$f'$はまた単調増加し、$f(x)$の減少の割合が小さくなる。
                    これは上に凸に当たる。こうして減少の割合が小さくなっていき、$x=\frac{4}{3}$を超えると$f'>0$となる。いま$f'$は単調増加するわけだから、これ以降は$f(x)$
                    の増加の割合が大きくなっていくわけである。

                    以上をまとめると次の結果が得られる。
                    \begin{screen}
                        ある$x$の区間$I$について
                        \begin{itemize}
                            \item $f''(x)>0$である場合は$y=f(x)$は区間$I$で\underline{下に凸}
                            \item $f''(x)>0$である場合は$y=f(x)$は区間$I$で\underline{上に凸}
                        \end{itemize}
                        が成り立つ。ただし、$f(x)$が二階微分可能であるとする。
                    \end{screen}
                    ここで二階微分という言葉が出てきたが、これは単に二回微分できるという意味である。ここで
                    $f''=0$を満たす点$(x,y)$のことを\textbf{変曲点}という。名前から推察できるように下に凸・上に凸というのは
                    グラフの曲がり方のことを言っており、変曲点はその曲がり方が変わる点である。今回の場合は$(0,0),(\frac{1}{2},-\frac{1}{16})$である。これを用いればグラフの形をより正確に書くことができる。

                    では二階微分を用いて先ほどの表\ref{ta:f(x)=x^4-x^3の増減表}を書き直してみると、
                    \begin{table}[h]
                        \centering
                        \begin{tabular}{|c||c|c|c|c|c|c|c|}\hline
                            $x$ & $\cdots$ & $0$ & $\cdots$ & $\frac{1}{2}$ & $\cdots$ & $\frac{3}{4}$ & $\cdots$ \\\hline
                            $f'(x)$ & $-$ & $0$ & $-$ & $-$ & $-$ & $0$ & $+$ \\\hline
                            $f''(x)$ & $+$ & $0$ & $-$ & $0$ & $+$ & $+$ & $+$ \\\hline  
                            $f(x)$ & \ser  & $0$ & \sel & $-\frac{1}{16}$ & \ser &$-\frac{27}{256}$ & \ner \\\hline
                        \end{tabular}
                        \caption{$f(x)=x^4-x^3$の増減表(変曲点付き)}\label{ta:f(x)=x^4-x^3の増減表(変曲点付き)}
                    \end{table}

                    矢印を直線ではなく若干曲げることで上に凸・下に凸を表現した。
                \clearpage
                \subsection{関数の最大・最小}
                    さて、ここまでグラフについての応用を述べたが、それ以外にも関数の増減を調べることでわかることがある。
                    それは特定の区間内における$f(x)$の最大値・最小値である。具体的に言えばその区間内の極値と区間の端点の値とを
                    比べればよい。なぜなら、端点から極値までの値は単調に増加・減少するからである。そしてここから、$f(x)$が
                    極大値(極小値)を一つだけ持つなら、それが$f(x)$の最大値(最小値)であることもわかる。\\

                    関数の最大・最小がわかることは、関数の大小関係を比べる際に役立つ。例えば、$x\geq \sin x\quad(0\leq x < 2\pi)$
                    を示すとする。$f(x)=x-\sin x$とおいて微分すると$f'=1 - \cos x = 0$となる。$[0,2\pi)$の区間でこの方程式の解は$x=0$のみであり、
                    $x>0$で$f'>0$であるため、$[0,2\pi$)で$f$は単調増加である。すなわち$x=0$が$f$の最小値でありその値は$f(0)=0$となるため、$f(x)\geq 0$と
                    書ける。よって移項して$x\geq \sin x$だと示せた。
                    \begin{figure}[h]
                        \centering
                        \includegraphics[scale=0.5]{img/QuuNote/x-sinxGraph.png}
                        \caption{$y=x,y=\sin x$のグラフ}
                    \end{figure}
                \clearpage
                \subsection{物理と微分}
                    この章の最後に物理と微分との関係について述べよう。力学では速度・加速度というものを学んだ。例えば速度は、「平均の速度」と「瞬間の速度」というものがあったと思う。
                    瞬間の速度の定義は
                    \begin{equation}
                        \frac{\Delta x}{\Delta t}
                    \end{equation}
                    であり、瞬間の速度は$\Delta t$を極めて小さくしていった場合の速度であった。この極めて小さく,というのは数学的には$\Delta t\to 0$の極限の操作をとることである。
                    つまり、瞬間の速度というのは平均の速度$\bar{v}$を$\Delta t\to 0$の極限値であり、これは変位$x(t)$の時間微分に等しい。
                    数式で書けば
                    \begin{equation}
                        v=\frac{dx}{dt}=\dot{x}(t)
                    \end{equation}
                    である。同様に、「瞬間の加速度」も
                    \begin{equation}
                        a=\frac{dv}{dt}=\frac{d^2 x}{dt^2}=\ddot{x}
                    \end{equation}
                    記号$d^2/dt^2$は二階微分を表す。このことから有名なニュートンの運動方程式も
                    \begin{equation}
                        F=m\ddot{x}
                    \end{equation}
                    という微分方程式になることがわかる。さて、ここからは少し背伸びしてベクトル解析の内容に入る。
                    というのも、変位や力など物理で出てくる量というのは大抵ベクトルが多いわけで、スカラー量の微分だけでは
                    応用例を紹介するのも難しいわけである。ベクトルの微分というと難しそうに聞こえるが、実は単純である。
                    ベクトルの関数$\bm{f}(t) =[f(t),g(t)]$\footnote{これをベクトル値関数という。}であるので、これを$t$で微分することは各成分を微分することであり、
                    $\dot{\bm{f}}=[f',g']$である。例えば運動方程式は
                    \begin{equation}
                        \bm{F}=m\ddot{\bm{x}}\label{eq:ニュートンの運動方程式}
                    \end{equation}
                    となる。また、位置ベクトルを$\bm{r}$とすると速度ベクトルは$\bm{v}=\dot{\bm{r}}$であり、加速度ベクトルは$\bm{a}=\ddot{\bm{r}}$である。
                    では、これを用いて等速円運動の速度ベクトルと加速度ベクトルを導いてみよう。簡単のため原点を中心とする円上の運動で、円の半径は$1$であり、$t=0$で座標は$(1,0)$とする。
                    このとき、位置ベクトルは$\bm{r}(t)=[\cos t,\sin t]$である。このとき、速度ベクトルは$\bm{v}=[-\sin t,\cos t]$であり、
                    加速度ベクトルは、$\bm{a}=[-\cos t,-\sin t]=-\bm{r}$となる。また、$\bm{r}\cdot \bm{v}=\bm{v}\cdot\bm{a}=0$である。
                    したがって、等速円運動では加速度ベクトルは常に円の中心を向き、位置ベクトルと速度ベクトル、速度ベクトルと加速度ベクトルはそれぞれ直行することがわかる。

                    さて、この結果を式\eqref{eq:ニュートンの運動方程式}に代入してみると、次のようになる。
                    \begin{equation}
                        \bm{F}=m\ddot{\bm{x}}=-m\bm{r}
                    \end{equation}
                    つまり、等速円運動では原点に向かって力が発生するのである。これを\textbf{向心力}という。一方、等速円運動する物体と
                    同じ座標系では加速度は0なわけだから、慣性の法則が成り立つように$m\ddot{\bm{r}}$という力を考え式を立てると$(-m\ddot{\bm{r}})+(m\ddot{\bm{r}})=m\cdot (0)$となる。
                    すなわち、等速円運動する座標系では向心力と逆向きの見かけ上の力が働いていることになる。これを\textbf{慣性力}といい、特に円運動の場合は\textbf{遠心力}という。
                    ちなみに等速円運動に限らず、物体の運動の加速度は接線方向の加速度ベクトルと法線方向のベクトルの加速度ベクトルの和として表せ、
                    これを運動方程式に代入した時の法線加速度の項が向心力になる。
                \clearpage
                \basicquestion 以下の問いに答えよ。

                \paragraph{問1}次の関数のグラフをかけ。\\
                $(1)y=x^3-x^2$\hspace{3mm}
                $\displaystyle(2)y=\frac{1}{1+x^2}$\hspace{3mm}
                $\displaystyle(3)y=e^{-x^2}$\hspace{3mm}
                $\displaystyle(4)y=\frac{\log x}{x}$

                \paragraph{問2}次の関数の(\hspace{1mm})内での最大・最小を求めよ。\\
                $(1)y=x^5-x^3+1\quad(-1\leq x\leq 1)$\hspace{3mm}
                $(2)e^x(x-1)\quad(-1\leq x\leq 2)$

                \paragraph{問3}$e^x \geq x+1 \quad (x\geq0)$を示せ。

                \paragraph{問4}図\ref{fig:直流回路}の直流回路について、最初に流れる電流$I$は以下の式で表される。
                \begin{equation*}
                    I=\frac{E}{R_0+R}\quad(\text{オームの法則})
                \end{equation*}
                また、可変抵抗器$R$で消費される電力$P$は$P=I^2R$である。この時、可変抵抗器で消費される最大の電力$P_{max}$
                の値を求めよ。また、この時の可変抵抗器の値を求めよ。

                \begin{figure}[h]
                    \centering
                    \includegraphics[scale=0.5]{img/QuuNote/tyokuryuukairo.png}
                    \caption{直流回路}\label{fig:直流回路}
                \end{figure}
            \clearpage
            \section{微分法諸定理}
                \subsection{$n$階微分とライプニッツの公式}
                    $y=f(x)$の導関数は$f'(x)$と表記される。では二階微分した場合はどうなるのかというと、これは変曲点の話ですでに出てきた通り$f''(x)$
                    である。これに準じて、三階微分、四階微分、...も$f'''(x),f''''(x),\cdots$と続く。しかしこれだと勝手が悪いので$'$の数だけ上に添え字で書くことが多い。
                    つまり、$f'''(x)=f^{(3)}(x),f''''(x)=f^{(4)}(x)$である。また、$n$階の導関数$f^{(n)}(x)$は$y^{(n)},D_x^{(n)} y,\displaystyle \frac{d^n y}{dx^n}$と書いたりもする。
                    これらは$n$次導関数ともいう。

                    \paragraph{例}$y=a^{x}$は$y'=a^x\log a,y''=a^x(\log a)^2,y^{(3)}=a^x(\log a)^3$である。一般に$y^{(n)}=a^x(\log a)^n$\\

                    さて、ここで積の$n$階微分について考えてみよう。\footnote{足し算・引き算は線形性より$(u\pm v)^{(n)}=u^{(n)}\pm v^{(n)}$がすぐわかってつまらない。かといって商の微分や合成関数の微分は計算が複雑すぎて
                    公式を導出するのも大変である。(なおかつその公式も複雑である。)}とりあえず、$n=1,2$のときで計算してみると
                    \begin{align*}
                        (uv)^{(1)}&=uv'+u'v\\
                        (uv)^{(2)}&=(uv'+u'v)^{(1)}=(uv')'+(u'v)'=u'v'+uv''+u''v+u'v'\\
                        &=u''v+2u'v'+uv''
                    \end{align*}
                    $n=1$のときは積の微分公式である。$n=2$は二項定理に形が似ている。$n=0$は微分していないことを意味するので、$u^{(2)}v^{(0)}+2u^{(1)}v^{(1)}+u^{(0)}v^{(2)}$
                    と書けばより二項定理っぽいことがわかる。実際$(u+v)^2=u^2v^0+2u^1v^1+u^0v^2$で形が一緒である。ちなみに$n=3$も$(uv)^{(3)}=u^{(3)}v+3u^{(2)}v^{(1)}+3u^{(1)}v^{(2)}+uv^{(3)}$
                    となって形が二項定理と一緒である。よって$n$階微分は次のような形になると予想される。
                    \begin{equation*}
                        (uv)^{(n)}=_nC_0u^{(n)}v+_nC_1u^{(n-1)}v^{(1)}+_nC_2u^{(n-2)}v^{(2)}+\cdots + _nC_mu^{(n-m)}v^{(m)}+\cdots+_nC_nuv^{(n)}
                    \end{equation*}
                    ではこれを数学的帰納法を用いて証明しよう。まず$n=1$のときは明らかに成り立つ。一般に$n=i$のときまで成り立つと仮定すると$n=i+1$のときは
                    \begin{equation*}
                        (uv)^{(i+1)}=\left((uv)^{(i)}\right)'
                    \end{equation*}
                    となるので、
                    \begin{equation*}
                        \cdots + _iC_m(u^{(i-m)}v^{(m)})'+\cdots=\cdots_iC_{(m-1)}(u^{(i+2-m)}v^{(m-1)}+u^{(i+1-m)}v^{(m)}) + _iC_m(u^{(i+1-m)}v^{(m)}+u^{(i-m)}v^{(m+1)})+\cdots
                    \end{equation*}
                    式を整理して
                    \begin{equation*}
                        _iC_0u^{(i+1)}v+\left(_iC_0+_iC_1\right)u^{(i)}v^{(1)}+\cdots+\left(_iC_{(m-1)}+_iC_m\right)u^{(i+1-m)}v^{(m)}+\cdots+_iC_iuv^{(i+1)}
                    \end{equation*}
                    ここで、$_iC_0=_iC_i=1$であるため$_iC_0=_{i+1}C_0,_iC_i={i+1}C_{i+1}$である。また、一般に$_{i+1}C_m=_iC_{m-1}+_iC_m$であるので、
                    \begin{equation*}
                        (uv)^{(i+1)}=_{i+1}C_0u^{(i+1)}v+_{i+1}C_1u^{(i)}v^{(1)}+\cdots+_{i+1}C_{m}u^{(i+1-m)}v^{(m)}+\cdots+_{i+1}C_{i+1}uv^{(i+1)}                    
                    \end{equation*}
                    前述したように$n=1$のときは成り立つので、数学的帰納法より予想が正しいことが証明された。
                    \clearpage
                    したがって次の公式が得られる。
                    \begin{equation}
                        (uv)^{(n)}=_nC_0u^{(n)}v+_nC_1u^{(n-1)}v^{(1)}+_nC_2u^{(n-2)}v^{(2)}+\cdots + _nC_mu^{(n-m)}v^{(m)}+\cdots+_nC_nuv^{(n)}\label{eq:ライプニッツの公式}
                    \end{equation}
                    これは\textbf{ライプニッツの法則}と呼ばれ、記号$\sum$を使って次のようにも書ける。
                    \begin{equation}
                        (uv)^{(n)}=\sum_{k=0}^{n} {_nC_k}u^{(n-k)}v^{(k)}
                    \end{equation}
                    こちらの方が簡潔に書ける。さらに、二項係数を$\dbinom{n}{k}$と書くことにすれば
                    \begin{equation}
                        (uv)^{(n)}=\sum_{k=0}^{n} \dbinom{n}{k}u^{(n-k)}v^{(k)}
                    \end{equation}
                    二項係数は$\binom{n}{k}$を使う教科書などのほうが多く、より一般的な書き方のようである。どちらの記号がよくてどちらがよくないかの判断は
                    こちらからは何とも言えないが、二項係数的な意味のときは$\binom{n}{k}$、組み合わせ論の話のときには$_nC_k$で使い分ければいいのではないだろうか。
                    これは好みのような気もする。
                \clearpage
                \subsection{ロルの定理}
                    ここではロルの定理について述べる。先に定理の内容についてあげておく。
                    \begin{itembox}{\textbf{ロルの定理}}
                        $[a,b]$で連続で$(a,b)$で微分可能な関数$f(x)$について、$f(a)=f(b)$ならば$f'(c)=0$となるような点$c\in(a,b)$が存在する。
                    \end{itembox}
                    これはグラフを書けば明らかである。$x$が$a\to b$に移動するとき、仮に$f(x)$が定数関数でなければ、
                    始めは$f(x)>f(a)$もしくは$f(x)<f(a)$になるはずである。しかし、$f(b)=f(a)$なので$x$が$b$に近くなるにつれて$f(a)$に近づかなければならない。
                    つまりどこかで増加から減少(もしくはその逆)になる点\footnote{これが極値であることはもう既知であろう。}が存在するはずなのである。$f(x)$が定数関数のときは明らかで$f'(x)=\frac{d}{dx}(\text{定数})=0$
                    である。

                    この定理は図を見ると意味がつかみやすい。区間を$[0,\pi]$でとっても$[-\pi,0]$でとっても$[-\pi,\pi]$でとっても良いが、
                    取った区間の間では必ず$f'(c)=0$となる$c$が存在しているのが一目でわかる。

                    \begin{figure}[h]
                        \centering
                        \includegraphics[scale=0.5]{img/QuuNote/rollTheory.png}
                        \caption{ロルの定理}
                    \end{figure}
                \clearpage
                \subsection{平均値の定理}
                    ここでは平均値の定理について述べる。こちらもロルの定理と同様に定理の内容からあげておく。
                    \begin{itembox}{\textbf{平均値の定理}}
                        $[a,b]$で連続で$(a,b)$で微分可能な関数$f(x)$について、
                        \begin{equation}
                            f'(c)=\frac{f(b)-f(a)}{b-a}\label{eq:平均値の定理}
                        \end{equation}
                        となるような点$c\in(a,b)$が存在する。
                    \end{itembox}
                    左辺は、$f(x)$の$(a,b)$上のある点での接線の傾きである。一方右辺は、以前述べた平均変化率の式そのままである。
                    すなわち平均値の定理は、点$(a,f(a)),(b,f(b))$を結んだ直線の傾きと接線の傾きが一致するような点が区間$(a,b)$に
                    少なくとも一つ存在することを示している。

                    それではお待ちかねの証明を示そう。まず、
                    \begin{equation*}
                        g(x)=\frac{f(b)-f(a)}{b-a}(x-a)+f(a)-f(x)
                    \end{equation*}
                    としておくことにする。このとき$g(x)$は$[a,b]$で連続で$(a,b)$で微分可能である。また$g(a)=g(b)=0$であることがわかる。
                    よってロルの定理が使えるので、$g'(c)=0$を満たす$c\in(a,b)$が存在するといえる。$g'(c)$を計算すると
                    \begin{equation*}
                        g'(c)=\frac{f(b)-f(a)}{b-a}-f'(c)=0
                    \end{equation*}
                    すなわち式\eqref{eq:平均値の定理}が成り立つ。$\square$

                    平均値の定理は様々な式の形に書き換えることができる。例えば
                    \begin{equation}
                        f(b)=f(a)+(b-a)f'(c)
                    \end{equation}
                    などはすぐ思いつくだろう。また、$a<c<b$なのだから$0<\theta<1$とすれば$c=a+\theta(b-a)$と書けるので
                    \begin{equation}
                        f(b)=f(a)+(b-a)f'(a+\theta(b-a))\quad(0<\theta<1)
                    \end{equation}
                    ともかける。またこれを微分係数のときと同じ要領で$h=b-a$と置けば
                    \begin{equation}
                        f(a+h)=f(a)+hf'(a+\theta h)\quad(0<\theta<1)\label{eq:平均値の定理1}
                    \end{equation}
                    とできる。さらに$a=x,h=\Delta x$と置けば
                    \begin{equation}
                        f(x+\Delta x)=f(x)+\Delta xf'(x+\theta\cdot\Delta x)\quad (0<\theta<1)\label{eq:平均値の定理x用}
                    \end{equation}
                    となる。もちろんこれらは式\eqref{eq:平均値の定理}を書き換えただけに過ぎないわけだが、目的によって
                    使い分けができるようになる。例えば、$\sqrt{5}$の近似値を平均値の定理で求めてみることにする。元の式からでは
                    どう求めるか見当もつかないが、式\eqref{eq:平均値の定理1}を用いれば$\sqrt{5}=\sqrt{4+1}$であるため、$f(x)=\sqrt{x}$と置けば
                    \begin{equation*}
                        f(4+1)=f(4)+f'(4+\theta)=2+\frac{1}{2\sqrt{4+\theta}}\thickapprox 2+\frac{1}{2\sqrt{4}}=2.25
                    \end{equation*}
                    として比較的自然に求められる。ちなみに$\sqrt{5}=2.23606\dots$であるので近似値の精度はそこまでよくない。
                \clearpage
                \subsection{コーシーの平均値の定理}
                    先ほど述べた平均値の定理を一般化してみよう。次の二つの関数$f,g$を考える。
                    この二つの関数は$[a,b]$で連続で、$(a,b)$で微分可能であるとする。またこの区間内では$g'\neq 0$であるとする。
                    この時平均値の定理から
                    \begin{equation*}
                        g(b)-g(a)=(b-a)g'(c_1)\quad(a<c_1<b)
                    \end{equation*}
                    $g'(c_1)$は仮定によって$0$ではないので、当然左辺も$0$ではない。そこで、
                    \begin{equation*}
                        \lambda=-\frac{f(b)-f(a)}{g(b)-g(a)}
                    \end{equation*}
                    とおき、関数
                    \begin{equation*}
                        F(x)=f(x)+\lambda g(x)
                    \end{equation*}
                    を作る。このとき、
                    \begin{align*}
                        F(a)&=f(a)+\lambda g(a)=\frac{f(a)(g(b)-g(a))-(f(b)-f(a))g(a)}{g(b)-g(a)}=\frac{f(a)g(b)-f(b)g(a)}{g(b)-g(a)}\\
                        &=\frac{(f(a)-f(b))g(b)+(g(b)-g(a)f(b))}{g(b)-g(a)}=f(b)+\lambda g(b)=F(b)
                    \end{align*}
                    であるので、ロルの定理が使えて、
                    \begin{equation*}
                        F'(c)=0\quad(a<c<b)
                    \end{equation*}
                    よって、
                    \begin{equation*}
                        F'(c)=f'(c)+\lambda g'(c)=0\leftrightarrow -\lambda = \frac{f'(c)}{g'(c)}
                    \end{equation*}
                    したがって次の\textbf{コーシーの平均値の定理}が得られる。
                    \begin{equation}
                        \frac{f(b)-f(a)}{g(b)-g(a)}=\frac{f'(c)}{g'(c)}\quad(a<c<b)\label{eq:コーシーの平均値の定理}
                    \end{equation}
                    ここで$g(x)=x$と置けば、平均値の定理\eqref{eq:平均値の定理}である。平均値の定理はコーシーの平均値の定理と区別して
                    ラグランジュの平均値の定理とも呼ばれる。
    
                \clearpage
                \subsection{ロピタルの定理}
                    極限値の計算を行っているときに、しばし
                    \begin{equation*}
                        \frac{0}{0},\quad\frac{\infty}{\infty},\infty -\infty,\infty\cdot\infty
                    \end{equation*}
                    などの\textbf{不定形}に遭遇するだろう。こういう不定形の極限値の計算で頭を抱えた経験も多いはずだ。不定形の解消には
                    様々な方法があるが、ここでは$0/0,\infty/\infty$の不定形に対して抜群な効果を発揮するロピタルの定理を示そう。
                    
                    例えば、$x\to a$の極限で$f(x)/g(x)\to 0/0$になってしまったとしよう。この時、$f'(x)/g'(x)$が極限値$b$に収束するならば、
                    $f(x)/g(x)$も同じ極限値$b$に収束する。なぜならコーシーの平均値の定理より、$f(a)=g(a)=0$のとき、$x>a$である$x$に対して
                    \begin{equation*}
                        \frac{f(x)}{g(x)}=\frac{f(x)-f(a)}{g(x)-g(a)}=\frac{f'(x)}{g'(c)}\quad(a<x<c)
                    \end{equation*}
                    ここで$x\to a$とすれば$c\to a$である。今$x>a$で考えているが、$x<a$の場合も同様なので、
                    \begin{equation}
                        \lim_{x\to a}\frac{f(x)}{g(x)}=\frac{f'(a)}{g'(a)}\label{ロピタルの定理}
                    \end{equation}
                    これを\textbf{ロピタルの定理}\footnote{ド・ロピタルの法則とも。}をいう。また、$f'(a)/g'(a)$が不定形$0/0$になるのなら
                    その時はもう一度ロピタルの定理を適用してあげて$f''(a)/g''(a)$と拡張できる。またロピタルの定理は$\infty/\infty$の不定形にも
                    適用できる。\footnote{証明は難しいので省略。$\varepsilon-\delta$論法を使えばできる。}

                    ロピタルの定理を用いて極限値を求めてみる。
                    \paragraph{例}
                        \begin{align*}
                            &(1) \lim_{x\to 0}\frac{\cos x-1}{x}=\lim_{x\to 0}\frac{-\sin x}{1}=0\\
                            &(2) \lim_{x\to \infty}\frac{e^{x}}{x}=\lim_{x\to\infty}\frac{e^x}{1}=\infty\\\
                            &(3) \lim_{x\to \infty}\frac{x^3+2x^2+1}{2x^3+1}=\lim_{x\to \infty}\frac{3x^2+4x}{6x^2}=\lim_{x\to\infty}\frac{6x+4}{12x}=\lim_{x\to\infty}\frac{6}{12}=\frac{1}{2}
                        \end{align*}
                        (1)は$0/0$、(2)は$\infty/\infty$の不定形である。また(3)はロピタルの定理を三回適用した場合である。
                    以下にロピタルの定理の間違った適用例を上げよう。
                    \begin{equation*}
                        \lim_{x\to 0}\frac{x}{\cos x}=\lim_{x\to 0}\frac{1}{-\sin x}=-\infty
                    \end{equation*}    
                    これは$0/0$の不定形でも$\infty/\infty$の不定形でもない。そもそも不定形ですらない。ロピタルの定理は使う前に条件をみたすかを確認しておきたいものである。
                    
                \clearpage
                \subsection{テイラーの定理}
                    最後に、テイラーの定理について述べよう。これはコーシーの平均値の定理とはまた違った平均値の定理\eqref{eq:平均値の定理}の一般化である。
                    先に定理について述べる。
                    \begin{itembox}{\textbf{テイラーの定理}}
                        関数$f(x)$が$[a,b]$で$n$階まで連続な導関数をもち、$(a,b)$で$n+1$階微分可能であるとき、ある点$c\in(a,b)$が存在し
                        \begin{equation}
                            f(b)=f(a)+f'(a)(b-a)+\frac{1}{2!}f''(a)(b-a)^2+\cdots+\frac{1}{n!}f^{(n)}(a)(b-a)^n+R_{n+1}\label{eq:テイラーの定理}
                        \end{equation}
                        ただし$R_{n+1}$は剰余項といい、
                        \begin{equation}
                            R_{n+1}=\frac{1}{(n+1)!}f^{(n+1)}(c)(b-a)^{n+1}\quad(a<c<b)
                        \end{equation}
                    \end{itembox}
                    $n=0$のときは平均値の定理\eqref{eq:平均値の定理}と同じである。証明は複雑であるが、平均値の定理と同様に証明できる。(証明は演習問題とする。)

                    平均値の定理のときと同様に、テイラーの定理から様々な表式を求めてみよう。平均値の定理と同様に$c=a+\theta(b-a)\quad(0<\theta<1)$と置く。
                    さらに$b=x$とすれば
                    \begin{alignat}{1}        
                        f(x)=&{}f(a)+f'(a)(x-a)+\frac{1}{2!}f''(a)(b-a)^2+\cdots+\frac{1}{n!}f^{(n)}(a)(x-a)^{n}\notag\\
                        &{}+\frac{1}{(n+1)!}f^{(n+1)}(a+\theta(x-a))(x-a)^{n+1}\label{eq:テイラー展開}
                    \end{alignat}
                    が得られる。これを関数$f(x)$の点$a$における\textbf{テイラー展開}という。さらに、テイラー展開の特別な場合として$a=0$を代入すれば
                    \begin{equation}
                        f(x)=f(0)+f'(0)x+\frac{1}{2!}f''(0)x^2+\cdots+\frac{1}{n!}f^{(n)}(0)x^n+\frac{1}{(n+1)!}f^{(n+1)}(\theta x)x^{n+1}\label{eq:マクローリン展開}
                    \end{equation}
                    これを関数$f(x)$の\textbf{マクローリン展開}という。テイラーの定理というとあまり馴染みのない言葉であるが、
                    テイラー展開/マクローリン展開と聞くと、理工学ではよく出てくるので知っている人も多いはずだ。
                    教科書などでは$R_{n+1}$を省略して$+\cdots$で終わっているものもあるかもしれないが、厳密に言えば間違いである。

                    \paragraph{マクローリン展開の例}$0<\theta<1$であるとする。
                        \begin{align}
                            e^x&=1+\frac{x}{1!}+\frac{x^2}{2!}+\cdots+\frac{x^n}{n!}+R_{n+1}\quad &&R_{n+1}=e^{\theta x}\frac{x^{n+1}}{(n+1)!}\label{eq:e^xのマクローリン展開}\\
                            \sin x &=x-\frac{x^3}{3!}+\frac{x^5}{5!}+\cdots+\frac{(-1)^{n-1}x^{2n-1}}{(2n-1)!}+R_{2n+1}\quad &&R_{2n+1}=\frac{(-1)^nx^{2n+1}}{(2n+1)!}\cos\theta x \label{eq:sin xのマクローリン展開}\\
                            \cos x &=1-\frac{x^2}{2!}+\frac{x^4}{4!}+\cdots+\frac{(-1)^{n}x^{2n}}{(2n)!}+R_{2n+2}\quad &&R_{2n+2}=\frac{(-1)^{n+1}x^{2n+2}}{(2n+2)!}\cos\theta x\label{eq:cos xのマクローリン展開}\\
                            \log(1+x) &= x-\frac{x^2}{2}+\frac{x^3}{3}+\cdots+(-1)^{n-1}\frac{x^n}{n}+R_{n+1}\quad &&R_{n+1}=\frac{(-1)^nx^{n+1}}{n+1}\left(\frac{1}{1+\theta x}\right)^{n+1}\label{eq:log(x+1)のマクローリン展開}
                        \end{align}
                    
                    さて、このような展開を用いれば関数$f(x)$を近似できるわけだが、より良く近似する際には項の数を増やし、$R_{n+1}$を
                    できるだけ小さくする必要があると予想できる。そこで、数列$\{R_n\}=R_1,R_2,R_3,\dots,R_n,\dots$を考え、数列$\{R_n\}$
                    が0に収束するときを考えてみよう。このとき
                    \begin{equation}
                        \lim_{n\to \infty}R_n=0
                    \end{equation}
                    であるので、より多くの項を取る($n\to\infty$)ほど、展開はよい近似になる。よって
                    \begin{equation}
                        f(x)=f(a)+f'(a)(x-a)+\cdots+f^{(n)}(a)\frac{(x-a)^n}{n!}+\cdots\label{eq:テイラー級数(テイラー展開)}
                    \end{equation}
                    と書く。最後の$\cdots$はどこまでも項を足していくことを意味する。この時\eqref{eq:テイラー級数(テイラー展開)}を\textbf{テイラー級数}という。
                    特に$a=0$なら、
                    \begin{equation}
                        f(x)=f(0)+f'(0)x+\cdots+f^{(n)}(0)\frac{x^n}{n!}+\cdots\label{eq:マクローリン級数(マクローリン展開)}
                    \end{equation}
                    となり、これを\textbf{マクローリン級数}と呼ぶ。

                    式\eqref{eq:テイラー級数(テイラー展開)}や式\eqref{eq:マクローリン級数(マクローリン展開)}は無限個の項を足し合わせており、このようなものを\textbf{無限級数}\footnote{加法はふつう有限のときに定義されるわけだから、無限個の和というのは形式的な表現である。実際の定義はのちに述べる。}という。
                    無限級数については第IV部で詳しく述べることにする。が面白い性質があるので先回りして紹介しておこう。
                    その性質とは、項別微分・項別積分ができることである。積分についてはまだ扱っていないので項別微分のみに着目しよう。これは名前の通り、項別に微分することができるということである。
                    例えば、$\sin x$のマクローリン級数は
                    \begin{equation}
                        \sin x=x-\frac{x^3}{3!}+\frac{x^5}{5!}+\cdots+\frac{(-1)^{n-1}x^{2n-1}}{(2n-1)!}+\cdots
                    \end{equation}
                    であり、左辺を微分すると$\cos x$である。一方右辺も項別に微分すると
                    \begin{equation}
                        \cos x = 1-\frac{x^2}{2!}+\frac{x^4}{4!}+\cdots+\frac{(-1)^{n-1}x^{2n-2}}{(2n-2)!}+\cdots
                    \end{equation}
                    となり$\cos x$のマクローリン級数\footnote{$2n-2=2(n-1)$なので$n'=n-1$と置けばきれいになる。}と一致する。この結果を用いれば$f(x)$の$n$次の微分係数を求めることなしに
                    マクローリン級数が求められることになる。
                \clearpage
                \basicquestion 以下の問いに答えよ。
                \paragraph{問1}次の関数の三次導関数を求めよ。\\
                $(1)y=(2x+1)^3$\hspace{3mm}
                $(2)\displaystyle y=\frac{1}{1+x}$\hspace{3mm}
                $(3)y=\sin(2x)$\hspace{3mm}
                $(4)y=xe^x$\hspace{3mm}
                $\displaystyle(5)y=\frac{\sin(2x)}{1+x}$

                \paragraph{問2}次の関数の$n$次導関数を求めよ。\\
                $(1)y=e^{-x}$\hspace{3mm}
                $(2)y=\cos x$\hspace{3mm}
                $(3)y=x^n$\hspace{3mm}
                $(4)y=\log(1+x)$

                \paragraph{問3}次の極限をロピタルの定理を用いて求めよ。\\
                $\displaystyle(1)\lim_{x\to 0}\frac{x-\log(1+x)}{x^2}$\hspace{3mm}
                $\displaystyle(2)\lim_{x\to 0}\frac{\sin^{-1}x}{x} $\hspace{3mm}
                $\displaystyle(3)\lim_{x\to \infty}\frac{\log x}{x}$\hspace{3mm}
                $\displaystyle(4)\lim_{x\to+0}x^x$\hspace{3mm}
                $\displaystyle(5)\lim_{x\to\infty}\log(1+e^x)^{\frac{1}{x}}$

                \paragraph{問4}次の極限をテイラー展開またはマクローリン展開を用いて求めよ。\\
                $\displaystyle(1)\lim_{x\to 0}\frac{x-\sin x}{x^3}$\hspace{50mm}
                $\displaystyle(2)\lim_{x\to \infty}\left\{\sqrt{x^2-3x+1}-x\right\}$

                \paragraph{問5}次の問いに答えよ。
                \begin{enumerate}\setcounter{enumi}{0}\renewcommand{\labelenumi}{(\arabic{enumi})}
                    \item $e^x$のマクローリン展開\eqref{eq:e^xのマクローリン展開}を示せ。
                    \item (1)の$x$に$ix$を代入せよ。
                    \item $\sin x$のマクローリン展開と$\cos x$のマクローリン展開と(2)から次の\textbf{オイラーの公式}\footnote{この公式は抜群に役に立つ。数学以外の様々な顔でも出し、特に$x=\pi$を代入した結果は人類の至宝とも称される。覚えておいて損はないと断言できる。}を導け。
                        \begin{equation}
                            e^{ix}=\cos x+i\sin x
                        \end{equation}
                \end{enumerate}
                
                \paragraph{問6}関数$f(x)$が$[a,b]$で$n$階まで連続な導関数を持ち、$(a,b)$で$n+1$階微分可能であるとする。また、次の二つをつくる。
                \begin{equation*}
                    g(x)=-f(b)+f(x)+f'(x)(b-x)+\frac{1}{2!}f''(x)(b-x)^2+\cdots+\frac{1}{n!}f^{(n)}(x)(b-x)^n+K(b-x)^{n+1}
                \end{equation*}
                \begin{equation*}
                    K=\frac{1}{(b-a)^{n+1}}\left[f(b)-\left\{f(a)+f'(a)(b-a)+\frac{1}{2!}f''(a)(b-a)^2+\cdots+\frac{1}{n!}f^{(n)}(a)(b-a)^n\right\}\right]
                \end{equation*}
                この時、テイラーの定理\eqref{eq:テイラーの定理}を証明せよ。
            \clearpage
            \section{第II部演習問題}
            \paragraph{問1}次の関数を微分せよ。\\
                $[1]y=x^3+x^2+x+1$\hspace{3mm}
                $[2]y=\cos^2 x-\sin^2 x$\hspace{3mm}
                $[3]y=\sinh(2x)$\hspace{3mm}
                $[4]y=\log\log\log x$\hspace{3mm}
                $[5]y=\log_{10}x$\\
                $[6]y=\cos(2\cos^{-1}x)$\hspace{3mm}
                $[7]y=\sin^{4}(3x)$\hspace{3mm}
                $\displaystyle[8]y=\frac{1}{\sqrt[3]{x^2+1}}$
            
            \paragraph{問2}次の関数のグラフをかけ。\\
                $[1]y=x^2e^x$\hspace{40mm}
                $[2]y=x^2\log x$\hspace{40mm}
                $[3]y=e^x\cos x$
            \paragraph{問3}次の極限を求めよ。\\
                $\displaystyle[1]\lim_{x\to 1}\frac{1+\cos\pi x}{(x-1)^2}$\hspace{10mm}
                $\displaystyle[2]\lim_{x\to 0}\frac{\sin x-x\cos x}{x^2\log(1+x)}$\hspace{10mm}
                $\displaystyle[3]\lim_{x\to\infty}\left\{\sqrt{(x+a)(x+b)}-\sqrt{(x-a)(x-b)}\right\}$
            \paragraph{問4}平均値の定理\eqref{eq:平均値の定理}を用いて$x>0$ならば$\sin x<x$を示せ。

            \paragraph{問5}次の問いに答えよ。
                \begin{enumerate}
                    \item $y=\tan^{-1}x$をマクローリン展開せよ。
                    \item 次を示せ。
                    \begin{equation*}
                        \frac{\pi}{4}=1-\frac{1}{3}+\frac{1}{5}-\cdots
                    \end{equation*}
                \end{enumerate}
            \paragraph{問6}次のマクローリン展開を証明し、これを用いて$\log 2$の値を小数点以下4位まで求めよ。
                \begin{equation*}
                    \log\left(\frac{1+x}{1-x}\right)=2\left(x+\frac{x^3}{3}+\frac{x^5}{5}+\cdots\right)
                \end{equation*}
            
            \paragraph{問7}$(1+x)^{\alpha}\quad(\alpha\text{は実数})$のマクローリン展開を求め、二項定理\eqref{eq:二項定理}を証明せよ。

            \paragraph{問8}以下の問いに答えよ。
                \begin{enumerate}\setcounter{enumi}{0}\renewcommand{\labelenumi}{(\arabic{enumi})}
                    \item $x,y$が変数$t$の関数として与えられているとき、$y$を$x$の関数(またはその逆)として考えられる。このとき、次の\textbf{媒介変数表示の微分公式}を導け。
                    \begin{equation}
                        \frac{dy}{dx}=\frac{\frac{dy}{dt}}{\frac{dx}{dt}}\label{eq:媒介変数の微分公式}
                    \end{equation}
                    \item 次の関数について$dy/dx$を求めよ。\\
                    $[1]x=\sin \theta,y=\cos\theta$\hspace{30mm}
                    $[2]x=t^3+2t,y=-t^2+3t$
                \end{enumerate}
            
            \paragraph{問9}次の関数の与えられた$x$の値に対応する点における接線の方程式を求めよ。\\
                $[1]y=x^3-3x^2\quad(x=3)$\hspace{10mm}
                $[2]y=\tan x\quad(x=0)$\hspace{10mm}
                $\displaystyle[3]y=\frac{\sin x}{x}\quad(x=\pi)$
            \paragraph{問10}$y=a^x$を$a$で微分せよ。

            \paragraph{問11}次の関数の導関数の$x=0$での連続性を調べよ。
                \begin{equation}
                    f(x)=\left\{\begin{array}{cc}
                        x^2\sin\frac{1}{x}&(x\neq 0)\\
                        0&(x=0)
                    \end{array}\right.
                \end{equation}
            \clearpage
            \paragraph{問12}1つの平面の両側に二点$A,B$が与えられているとする。
            動点$P$がこの平面の両側でそれぞれ一定の速さ$v_a,v_b$で運動するとき、$P$が
            $A$から$B$まで最短の時間で行くべき経路を求めよ。(解析概論より)
            \begin{figure}[h]
                \centering
                \includegraphics[scale=0.3]{img/QuuNote/snellQuestion.png}
                \caption{最短経路を求めよ}
            \end{figure}

            \paragraph{問13}次の\textbf{微分方程式}について以下の問いに答えよ。
                \begin{equation*}
                    y'-y^2-1=0
                \end{equation*}
                \begin{enumerate}\renewcommand{\labelenumi}{(\arabic{enumi})}
                    \item $y=\tan x$が方程式の解の一つであることを確かめよ。
                    \item (1)以外にどのような解が考えられるか。考えられる解を一つ答えよ。
                \end{enumerate}
            
            \paragraph{問14}指数関数$e^x$は$x$のどんな正のべきよりも早く増加し、対数関数$\log x$は
            $x$のどんな正のべきよりもゆっくり増加することを示せ。   
            
            \linktoMOKUZI
    \clearpage

    \part{積分$\int$}
        \vspace{\stretch{1}}
        \begin{screen}
            ついに微分積分の``積分''の話に移る。積分法は、微分方程式を解いたり、面積を求めたり...と様々な応用例がある。
            微分と違って具体的なイメージがしやすい反面、公式をそのまま適用できる場合が少なく、計算が難しい。
            微分と同じで慣れるまで問題をたくさん解くことで、ある程度感覚がつかめてくる。積分には定積分と不定積分の
            二つがあり、これらは互いに独立した概念である。順番としては不定積分,定積分,の順に扱う。
            歴史的には定積分,微分,不定積分の順に発明されたというのだから、面白い。なお、ここで扱うのはリーマン積分
            である。
        \end{screen}
        \clearpage
        \section{不定積分}
            \subsection{不定積分とは}
                これまで扱ってきた微分は現象の微小な変化を解析するものだった。しかし、現実には現象からある瞬間の変化を調べることに加えて、
                ある一瞬の変化から現象自身を得る必要も出てくる。実際の自然現象などは常に変化し続けており、それを定式化(方程式)にするには、
                ある瞬間の変化量を使うほうがその現象を本質的に見ることができる。例を示そう。例えば、空気抵抗を無視した
                物体の落下運動は最初の落下地点の高さをを$x_0$とすると$x(t)=-\frac{1}{2}gt^2+x_0$と表すことができる。
                しかし、変位の二階時間微分が加速度であることを考えれば$\ddot{x}=-g$と簡潔に表せる。
                これだけ見ると、別に微分を使って表す必要もなさそうである。では空気抵抗がある場合を考えてみよう。
                空気抵抗が速度に定数$k$で比例すると仮定すれば、微分を使わずに表すと\footnote{間違っているかも?各自で確認することをお勧めする。}
                \begin{equation}
                    x(t)=-\frac{m^2g}{k^2}e^{\frac{k}{m}t}+\frac{mg}{k}t+\frac{m^2g}{k^2}-\frac{mg}{k}+x_0
                \end{equation}
                のようになる。ちなみに$m$は質点の質量である。一方後者の方法で表すと
                \begin{equation}
                    m\ddot{x}=k\dot{x}-mg\label{eq:空気抵抗ありのときの落下運動}
                \end{equation}
                明らかに後者の方が簡単である。しかも後者の良いところは、初期条件($t=0$で$x=x_0$など)に関係なく同じ表式が得られるところで、
                これが現象の本質を表していることを示している。

                一方で、物理現象の本質を明らかにすることとは別に実際にその表式(今回の場合は変位)を得たい場合もある。
                その場合は式\eqref{eq:空気抵抗ありのときの落下運動}のような\textbf{微分方程式}では困るわけである。
                $y'=x$のような簡単な微分方程式なら$(x^2)'=2x$から簡単に解が予想できるが、$y'=x^2$や$y'=\sin 2x$など
                問題のたびに微分から予想していては大変である。また、式\eqref{eq:空気抵抗ありのときの落下運動}みたいに式が複雑になっていくと
                解の一つを予想するだけで苦労してしまう。そこで、微分と逆の演算である\textbf{積分}が登場する。この積分を学べば
                $y'(x)=f(x)$タイプの微分方程式はある程度統一的に解くことができるようになるのである。\\

                ここで、不定積分を(狭い範囲で)定義しよう。ある連続関数$f(x)$に対して
                \begin{equation}
                    F'(x)=f(x) \label{eq:原始関数定義}
                \end{equation}
                となる関数$F(x)$が存在した時、この関数$F(x)$を$f(x)$の原始関数、もしくは\textbf{不定積分}\footnote{実はこれは不定積分ではなく、原始関数の定義である。一般に原始関数と不定積分が等しくなる保証はないわけだが、
                $f$が連続関数であればこれらは同義になる。我々は今基本的に連続関数のみ扱いたいわけだから、これらを同じものとして扱っている。}という。
                この時$f(x)$を被積分関数といい、不定積分を次のように表す。
                \begin{equation}
                    F(x)=\int f(x)dx = \int dx f(x) \label{eq:不定積分の書き方}
                \end{equation}
                上のように$f(x)$と$dx$の順序はどちらでもよい。$f$があまりにも複雑なら最後に$dx$をつけ忘れてしまうかもしれない。その場合は後者の方がいい。
                一方多項式の間に積分が挟まれていて、どこからどこまでが$f$なのかがわかりにくいときなどは前者を使えばいい。

                ある関数の不定積分が$F(x)$なら、当然それに定数を加えた$F(x)+C$も$f$の不定積分になる。なぜなら
                \begin{equation*}
                    (F(x)+C)'=F'(x)+(C)'=F'(x)=f(x)
                \end{equation*}
                となるから当然である。つまり不定積分は定数の分だけ``不定''なのである。このような任意定数を積分定数という。
                
                また、定義から明らかに
                \begin{equation}
                    \int \frac{df}{dx}dx = f(x)+C
                \end{equation}
                であることがわかる。もちろん常にこんな簡単な積分ばかりではない。

                \paragraph{例}$\displaystyle\int x dx$を求める。$(x^2)'=2x$だから$x=\frac{1}{2}(x^2)'$
                \begin{equation*}
                    \int x dx = \int \frac{1}{2}(x^2)'dx=\frac{1}{2}\int (x^2)'dx=\frac{1}{2}x^2+C\quad (\text{$C$は積分定数})
                \end{equation*}

                上記の例のように、不定積分は微分演算の逆演算である。このことを利用して次節で様々な公式を導出する。
            \clearpage
            \subsection{不定積分の公式}
                ここでは不定積分の性質と公式についてまとめる。以下積分定数を省略する。まずは加法性について。
                \begin{equation}
                    \int \left\{f(x)\pm g(x)\right\}dx=\int f(x)dx\pm \int g(x)dx \label{eq:不定積分の加法性}
                \end{equation}
                である。これは定義からわかる。$f,g$の不定積分を$F(x),G(x)$と表すことにすれば
                \begin{equation}
                    (F(x)\pm G(x))'=(F(x))'\pm(G(x))'
                \end{equation}
                だから、これを両辺不定積分すれば
                \begin{equation}
                    F(x)\pm G(x)=\int \left\{(F(x))'+(G(x))'\right\}dx=\int \left\{f(x)\pm g(x)\right\}dx
                \end{equation}
                となり式\eqref{eq:不定積分の加法性}を得る。

                次に$f(ax+b)$について考えて見る。これも定義から$(F(ax+b))'=\frac{1}{a}F'(ax+b)$なのだから
                \begin{equation}
                    \int f(ax+b)dx=\frac{1}{a}F(ax+b)\label{eq:f(ax+b)の不定積分}
                \end{equation}

                では、それぞれの関数の不定積分を見ていこう。まずは$x^n$について考える。$n\neq -1$のとき
                $x^{n+1}$を微分すると、$(n+1)x^{n}$となるから次の公式が得られる。
                \begin{equation}
                    \int x^n dx = \frac{1}{n+1} x^{n+1}\quad (n\neq -1) \label{eq:x^nの不定積分}
                \end{equation}
                また、$n=-1$のときは$(\log |x|)'=\frac{1}{x}$であるため、
                \begin{equation}
                    \int \frac{dx}{x} = \log |x| \label{eq:1/xの不定積分} 
                \end{equation}

                三角関数については、$(\sin x)'=\cos x,(\cos x)'=-\sin x$であるから
                \begin{equation}
                    \int \sin xdx = -\cos x,\quad \int \cos x dx = \sin x \label{eq:sin,cosの積分}
                \end{equation}
                $\tan x$についても、$(\tan x)'=\frac{1}{\cos ^2x}$であるため
                \begin{equation}
                    \int \frac{1}{\cos^2 x}dx=\tan x \label{eq:tanの不定積分}
                \end{equation}

                指数関数$e^x$についてはどうだろうか。$(e^x)'=e^x$であるから
                \begin{equation}
                    \int e^x dx = e^x \label{eq:expの不定積分}
                \end{equation}
                である。一般の指数関数$a^x$は$(a^x)'=a^x \log a$より
                \begin{equation}
                    \int a^x dx =\frac{a^x}{\log a} \label{eq:a^xの不定積分}
                \end{equation}
                一方対数関数$\log x$の不定積分を求めるには、あるテクニック/公式が必要なのでここでは省略する。\\

                以上で述べたことは、原理を理解するのはもちろん公式として暗記しておくことを勧める。

            \clearpage
            \subsection{置換積分法}
                前節では、単純な関数の積分について述べた。しかし、これらの公式だけではほとんどの積分には太刀打ちできない。例えば
                \begin{equation*}
                    \int x(x^2+1)^{1000}dx
                \end{equation*}
                など、理論上は展開して計算できるが1000乗の展開など想像するだけでぞっとする。また、
                \begin{equation*}
                    \int x\sin(x^2)dx
                \end{equation*}
                などのような複雑な関数の積分はどのようの解けばよいだろうか。

                そこで、このような複雑な積分を計算する方法として\textbf{置換積分}法を紹介する。
                \begin{itembox}{置換積分の公式}
                    関数$\phi(t)$について$x=\phi(t)$と置くとき
                    \begin{equation}
                        \int f(x)dx = \int f(\phi(t))\phi'(t)dt \label{eq:置換積分}
                    \end{equation}
                    が成り立つ。
                \end{itembox}
                \paragraph{証明} 合成関数の微分法によって
                \begin{equation}
                    \frac{d}{dt}F(\phi(t))=F'(\phi(t))\phi'(t)=f(\phi(t))\phi'(t) 
                \end{equation}
                また、$x=\phi(t)$より
                \begin{equation}
                    \frac{d}{dx}F(\phi(t))=\frac{d}{dx}F(x)=f(x)
                \end{equation}
                よってそれぞれ$x,t$で不定積分すれば式\eqref{eq:置換積分}が得られる。\\

                \paragraph{例1}
                \begin{equation*}
                    \int x\sin(x^2)dx
                \end{equation*}
                を求める。$t=g(x)=x^2$と置くと、$g'(x)=2x$であるため、
                \begin{equation*}
                    \int \frac{1}{2}\sin(g(x))g'(x)dx=\int \frac{1}{2}\sin t dt = -\frac{1}{2}\cos t = -\frac{1}{2}\cos x^2
                \end{equation*}\\

                式\eqref{eq:置換積分}の$\phi'(t)$は$\frac{dx}{dt}$のことであるため、書き直すと
                \begin{equation}
                    \int f(\phi(t))\frac{dx}{dt}dt =\int f(x)dx 
                \end{equation}
                となり、あたかも$\frac{dx}{dt}$が$dt$で約分されているように見える。また、$t=g(x)$と置いて$\frac{dt}{dx}$を求める行為は
                形式的には両辺をそれぞれの文字$t,x$で微分して$dt,dx$を付けたものと等しくなる。つまり、$dt=g'(x)dx$を代入したように見える。
                この書き方は便利なのでしばし用いられるが、あくまで形式的なものに過ぎない。
                \clearpage
                \paragraph{例2}
                \begin{equation*}
                    \int \frac{dx}{\sqrt{1+x^2}}
                \end{equation*}
                を求める。$x=\sinh t$と置換すると$dx = \cosh t dt$であるため
                \begin{equation*}
                    \int \frac{\cosh t}{\sqrt{1+(\sinh t)^2}}dt=\int \frac{\cosh t}{\cosh t}dt=\int dt = t = \sinh^{-1} x
                \end{equation*}
                もちろんこのままでもいいが、いい機会なので$\sinh^{-1}x$の具体的な式を求めてみよう。まず、$\sinh$の定義より
                \begin{equation*}
                    x = \sinh t = \frac{e^t-e^{-t}}{2}=\frac{e^{2t}-1}{2e^t}
                \end{equation*}
                よって、両辺に$e^t$をかけて、$e^t$に関する二次方程式を解くと
                \begin{equation*}
                    e^{t} = \frac{2x\pm \sqrt{4x^2+4}}{2}=x\pm\sqrt{x^2+1}
                \end{equation*}
                $e^t>0$より、符号は$+$のみである。したがって、対数を取ると
                \begin{equation}
                    t=\sinh^{-1}x=\log\left(x+\sqrt{x^2+1}\right)\label{eq:arcsinh}
                \end{equation}\\

                さて、例3では式\eqref{eq:置換積分}の右辺から左辺への変形であった。置換積分は$x=\dots$と置く場合と
                $t=\dots$と置く場合とがある。どちらの場合でも対応できるようにしておきたいものである。ちなみに、いままで置換の変数に
                $t$を用いてきたが、こだわらなくてよい。つまり$u=x^2$と置いたり$x=\cos\theta$と置いても何も差し支えない。
            \clearpage
            \subsection{部分積分法}
                置換積分は合成関数の微分から導出されたものであるが、同じように積の微分から新たな公式を導出することができる。それが\textbf{部分積分法}
                であり、置換積分では計算できない積分などに使う。
                \begin{itembox}{部分積分法}
                    関数$f(x),g(x)$について
                    \begin{equation}
                        \int f(x)g'(x) dx = f(x)g(x) - \int f'(x)g(x)dx \label{eq:部分積分法}
                    \end{equation}
                    が成り立つ。
                \end{itembox}
                \paragraph{証明}積の微分公式より
                \begin{equation}
                    (fg)'=f'g+fg' \leftrightarrow fg' = (fg)'-f'g
                \end{equation}
                あとは両辺を積分すれば式\eqref{eq:部分積分法}が得られる。

                \paragraph{例1}
                \begin{equation*}
                    \int x\sin x dx
                \end{equation*}
                を求める。$f(x)=x,g'(x)=\sin x$とすると
                \begin{equation*}
                    \int x(-\cos x)'dx = -x\cos x + \int \cos xdx = -x\cos x+\sin x
                \end{equation*}
                となる。

                例1で、$f,g$をうまく選ばないと悲惨な結果になる。例えば$f=\sin x,g=x$と選ぶと
                \begin{equation*}
                    \int x\sin xdx=\frac{1}{2}x^2\sin x - \int \frac{1}{2}x^2\cos xdx
                \end{equation*}
                とまあより複雑な積分になってしまう。

                \paragraph{例2}$\log x$の不定積分を求める。$(x)'=1$より
                \begin{equation*}
                    \int \log xdx=\int (x)'\log xdx=x\log x - \int x\cdot \frac{1}{x}dx=x\log x-\int dx=x\log x-x
                \end{equation*}
                したがって次の公式が得られた。
                \begin{equation}
                    \int \log xdx=x\log x-x\label{eq:log xの不定積分}
                \end{equation}\\

                部分積分は、$fg$のうちすくなくとも片方の関数が微分したら周期的に元に戻る関数であるときによく使われる印象である。\footnote{例えば、$\sin x$は4回微分したら元の関数に戻る。$e^x$は1回である。}
                また、積分漸化式の問題でも用いられる。
            \clearpage
            \subsection{有理関数の不定積分}
                有理関数$R(x)$の不定積分については次の重要な性質がある。
                \begin{screen}\centering
                    有理関数$R(x)$は\underbar{必ず不定積分できる}。
                \end{screen}
                ここでいう``不定積分できる''とは、その関数の不定積分が初等関数の範囲で表せられることを言う。\footnote{一般に、初等関数の原始関数が初等関数になる\underbar{ことは起こりえない}。$e^{-x^2},\sin x/x$などが有名な例である。}
                今からこれを示すが、この証明を理解することよりも有理関数の不定積分ができることを覚えておくことのほうが重要である。\\

                さて、有理関数において分母の次数より分子の次数が高いとき、それは多項式と分母の次数のほうが高い有理関数の和で表すことができる。これには正式の除算を用いればよい。
                例えば、$\frac{x^3+2}{x+1}=x^2-x+1+\frac{1}{x+1}$
                
                多項式の積分の方が積分できるのはすぐわかるので、注目すべきは有理関数のほうである。この時この有理関数は
                \begin{equation}
                    \frac{1}{(ax+b)^n}\text{と}\frac{px+q}{\left\{(x-a)^2+b^2\right\}^n}
                \end{equation}
                の和の形に必ず分解することができる。\footnote{参考:\url{https://www.math.titech.ac.jp/~hoya/Teaching/calculusI2023PDF/bubunbunsu.pdf}}
                
                まず前者の積分であるが、これは簡単である。公式を用いて
                \begin{equation}
                    \int \frac{dx}{(ax+b)^n} = \left\{\begin{array}{lc}
                        \displaystyle \frac{1}{a(-n+1)}\frac{1}{(ax+b)^{n-1}} & (n\neq 1) \\
                        \displaystyle \frac{1}{a}\log|ax+b| & (n=1)
                    \end{array}\right.
                \end{equation}
                となる。後者は少し難しい。式を変形すると
                \begin{equation*}
                    \frac{px+q}{\left\{(x-a)^2+b^2\right\}^n}=\frac{p}{2}\frac{2(x-a)}{\left\{(x-a)^2+b^2\right\}^n}+\frac{(q+pa)}{\left\{(x-a)^2+b^2\right\}^n}
                \end{equation*}
                と分けられる。右辺第一項の積分は、$t=(x-a)^2+b^2$の置換を行うと
                \begin{equation*}
                    \frac{p}{2}\int \frac{dt}{t^n}=\frac{p}{2(-n+1)}\frac{1}{t^{n-1}}=\frac{p}{2(-n+1)}\frac{1}{\left\{(x-a)^2+b^2\right\}^{n-1}}
                \end{equation*}
                となる。問題は二項目の積分であり、これを直接解こうとすると大変である。ここは、積分を$I_n$と置いて$n$についての漸化式を作る。定数部分は関係ないので省略する。
                \begin{align*}
                    I_n &= \int \frac{dx}{\left\{(x-a)^2+b^2\right\}^n}=\int \frac{dt}{(t^2+b^2)^n} = \frac{t}{(t^2+b^2)^n} - \int (-2n)\cdot\frac{t^2}{(t^2+b^2)^{n+1}}dt\\
                        &= \frac{t}{(t^2+b^2)^n}+2n\int \frac{t^2+b^2-b^2}{(t^2+b^2)^{n+1}}dt=\frac{t}{(t^2+b^2)^n}+2nI_n-2nb^2I_{n+1}
                \end{align*}
                $n+1=m$と置き、式を整理すると、
                \begin{equation}
                    I_m = \frac{1}{b^2}\left[\frac{2m-3}{2m-2}I_{m-1}+\frac{1}{2(m-1)}\frac{(x-a)}{\left\{(x-a)^2+b^2\right\}^{m-1}}\right] \label{eq:有理関数の不定積分漸化式}
                \end{equation}
                \clearpage
                式\eqref{eq:有理関数の不定積分漸化式}から、$I_m$を求めるには$I_{m-1}$を、$I_{m-1}$を求めるには$I_{m-2}$を・・・とどんどん遡っていき$m=2$まで行くと、不定積分は求められる。
                なぜなら$m=2$つまり$n=1$のとき
                \begin{equation*}
                    I_1 = \int \frac{dx}{(x-a)^2+b^2}=\frac{1}{b}\tan^{-1}\left(\frac{x-a}{b}\right) 
                \end{equation*}
                となるからである。以上の結果から有理関数の不定積分は\underline{常に求められる}。\\

                とはいえ、有理関数の不定積分を求めることは簡単なことではなく、計算は煩雑になりがちである。
                大事なのは、冒頭でも述べた通り不定積分できるという認識を持っておくことである。

                \paragraph{例1}
                \begin{equation*}
                    \int \frac{dx}{x^2-1}
                \end{equation*}
                を求める。分母を因数分解すると$x^2-1=(x-1)(x+1)$であるため、
                \begin{equation*}
                    \frac{1}{x^2-1}=\frac{1}{(x-1)(x+1)}=\frac{1}{2}\left(\frac{1}{x-1}-\frac{1}{x+1}\right)
                \end{equation*}
                と部分分数分解して
                \begin{equation*}
                    \int \frac{dx}{x^2-1}=\frac{1}{2}\left(\int \frac{dx}{x-1}-\int \frac{dx}{x+1}\right)=\frac{1}{2}\left(\log|x-1|-\log|x+1|\right)=\frac{1}{2}\log\left|\frac{x-1}{x+1}\right|
                \end{equation*}

                上の例では部分分数分解が簡単にできるが、中には部分分数分解すること自体が大変な場合もある。
                \paragraph{例2}
                \begin{equation*}
                    \frac{1}{x^3+1}
                \end{equation*}
                を部分分数分解する。分母を因数分解すると$x^3+1=(x+1)(x^2-x+1)$より
                \begin{equation*}
                    \frac{1}{x^3+1}=\frac{1}{(x+1)(x^2-x+1)}=\frac{A}{x+1}+\frac{Bx+C}{x^2-x+1}
                \end{equation*}
                と置いて\footnote{二項目は分母が二次式なので分子は$(n-1)=1$次式で置く。}、この恒等式から$A,B,C$を求めればよい。\footnote{$1/(x^2-1)$のときも同様に求めてよい。ただこの場合は簡単なので省いただけ。}
                両辺の分母に$(x+1)(x^2-x+1)$を掛けると
                \begin{equation*}
                    1 = A(x^2-x+1)+(Bx+C)(x+1)=(A+B)x^2+(-A+B+C)x+(A+C)
                \end{equation*}
                となるため、
                \begin{equation*}
                    \left\{\begin{array}{rrccccl}
                        &A&+&B& &&=0\\
                        -&A&+&B&+&C & = 0\\
                        &A&&&+&C &=1
                    \end{array}\right.
                \end{equation*}
                の連立方程式を解く。あとは代数の計算なので普通に解けばよい。係数的にクラメルの公式が簡単である。
                $(A,B,C)=(\frac{1}{3},-\frac{1}{3},\frac{2}{3})$より
                \begin{equation*}
                    \frac{1}{x^3+1}=\frac{1}{3}\left(\frac{1}{x+1}+\frac{-x+2}{x^2-x+1}\right)
                \end{equation*}

            \clearpage
            \basicquestion 以下の問いに答えよ。積分定数は省略してもよい。

            \paragraph{問1}次の不定積分を求めよ。\\
                $(1)\displaystyle \int x^3dx$\hspace{3mm}
                $(2)\displaystyle \int \sin(-x) dx$\hspace{3mm}
                $(3)\displaystyle \int \left(e^{2x}+\frac{1}{x^2}\right)dx$\hspace{3mm}
                $(4)\displaystyle \int \frac{dx}{ex}$\hspace{3mm}
                $(5)\displaystyle \int \left(\tan^2 x + 1\right)dx$\hspace{3mm}
                $(6)\displaystyle \int \sin x\cos xdx$\\

            \paragraph{問2}次の不定積分を(\hspace{1mm})内の置換によって求めよ。\\
                $(1)\displaystyle \int x\cos(x^2)dx \quad (t=x^2)$\hspace{3mm}
                $(2)\displaystyle \int x^2(x^3-1)^2dx\quad (t=x^3-1)$\hspace{3mm}
                $(3)\displaystyle \int \frac{dx}{x^2+1}\quad (x=\tan \theta)$\\
                $(4)\displaystyle \int \frac{dx}{\sqrt{1-x^2}}\quad (x=\sin \theta)$\hspace{3mm}
                $(5)\displaystyle \int \sin^3 x\cos x dx \quad(t=\sin x)$\\
            
            \paragraph{問3}次の不定積分を部分積分法で求めよ。{\scriptsize (5)ヒント:$I=$と置き、二回部分積分せよ。}\\
                $(1)\displaystyle \int x\cos xdx$\hspace{3mm}
                $(2)\displaystyle \int xe^x dx$\hspace{3mm}
                $(3)\displaystyle \int e^x\cos xdx$\hspace{3mm}
                $(4)\displaystyle \int \log(x^2+1)dx$\hspace{3mm}
                $(5)\displaystyle \int e^x\sin xdx$

            \paragraph{問4}次の不定積分を部分分数分解を用いて求めよ。\\
                $(1)\displaystyle \int \frac{dx}{x^2-3x+2}$\hspace{20mm}
                $(2)\displaystyle \int \frac{dx}{(x^2+1)(x-2)}$\hspace{20mm}
                $(3)\displaystyle \int \frac{dx}{x^2-a^2}$\\

            \paragraph{問5}以下証明せよ。
                \begin{enumerate}\setcounter{enumi}{0}\renewcommand{\labelenumi}{(\arabic{enumi})}
                    \item $\displaystyle \int \sinh x dx = \cosh x$を示せ。
                    \item $\sin x,\cos x$の有理関数を$f(\sin x,\cos x)$と表す\footnote{例えば、$\frac{\sin x}{1+\cos x}$は、$f(X,Y)=\frac{X}{1+Y}$の形である。引数が二つあると二変数関数みたいに見えるが実際には$x$だけの関数である。}
                    とする。このとき不定積分
                        \begin{equation*}
                            \int f(\sin x,\cos x)dx
                        \end{equation*}
                        は$t=\tan\frac{x}{2}$の置換により必ず積分できることを示せ。
                \end{enumerate}
        \clearpage
        \section{定積分}
            \subsection{面積を求めるには}
                不定積分の話はいったん忘れて、一度素朴な話題に移ろう。古来より様々な図形の面積が求められた。
                もっとも有名で美しい図形でよく上げられるのは円である。図形の面積の求め方には大きく二つある。この二つの方法で円の面積を求めてみることにしよう。

                まず、最初に思いつく方法としては\textbf{取りつくし法}だろう。これはすでに面積が既知である図形(例えば長方形)などの図形で、円を
                埋め尽くして面積を求める方法で、やり方は至極簡単である。
                \begin{figure}[h]
                    \centering
                    \includegraphics[scale=0.35]{img/QuuNote/Tritukusi.png}
                    \caption{取りつくし法による円の面積}
                \end{figure}

                ただ、取りつくし法の問題点は一般性がないことである。例えば上記の例では正方形と二種類の二等辺三角形で円の面積を近似しているが、
                正$n$角形で近似する方法もある。しかし、対象の図形が楕円などで異なった場合は、その図形に応じて取りつくしの方法も変わってしまう。

                そこで古代から考えられてきたもう一つの方法として\textbf{区分求積法}がある。
                これは図形を短冊状の図形に区切ってそれぞれの面積を面積を求め、それらを足す方法である。
                \begin{figure}[h]
                    \centering
                    \includegraphics[scale=0.35]{img/QuuNote/kubunkyuseki.png}
                    \caption{区分求積法による半円の面積}
                \end{figure}

                これの良いところは。図形の形によらず同じ方法で図形の面積が求められることである。そのため図形の上部を
                を適当な関数で表せればよいことになる。区分求積法も古来より用いられてきた方法であるが、実際に面積を求めるのは
                簡単なことではない。そこで、\textbf{定積分}が登場する。
            \clearpage
            \subsection{定積分の定義}
                以下に定積分の定義を述べよう。まず、xy平面上の関数$y=f(x)$の区間$[a,b]$での面積を求めることを考える。簡単にするために、$[a,b]$で$f\geq0$であるとする。
                $[a,b]$の間に$x_0=a,x_1,x_2,\dots,x_n=b$の$n+1$個の点を考え、$z_k \in [x_{k-1},x_{k}]$である任意の点$z_1,z_2,z_3,\dots,z_{n}$を考える。
                $\Delta x_k=x_{k}-x_{k-1}$と定義すると、$y=f(x)$と直線$x=a,x=b$とx軸で囲まれた面積$S$の近似値$S_{\Delta}$は次のように与えられることがわかる。
                \begin{equation}
                    S_{\Delta} = \sum_{k=1}^{n}f(z_k)\Delta x_{k} \label{eq:リーマン和の定義}
                \end{equation}
                式\eqref{eq:リーマン和の定義}で定義された量$S_{\Delta}$を\textbf{リーマン和}\footnote{本によっては積和と呼ぶこともある。このノート中はどちらとも使っているかもしれない。}と呼ぶ。この積和に$n\to \infty$の極限を取る\footnote{高専の教科書では$\Delta x_k \to 0$としているが、このノートでは解析概論等の数学書に合わせ$n\to 0$とする。どちらも本質は同じである。}、つまり
                分割の数を極限まで大きくして、ある一定の値に収束したとき、これを
                \begin{equation}
                    \int_a^b f(x)dx = \lim_{n\to\infty}\sum_{k=1}^{n}f(z_k)\Delta x_k \label{eq:定積分の定義}
                \end{equation}
                と表す。この左辺を$y=f(x)$の区間$[a,b]$での\textbf{定積分}といい、この値は$S$に等しい。(下図\eqref{fig:定積分図}参照)

                \begin{figure}[h]
                    \centering
                    \includegraphics[scale=0.35]{img/QuuNote/integral_picture.png}
                    \caption{定積分の定義}\label{fig:定積分図}
                \end{figure}

                記号$\int_a^b f(x)dx$について、$a,b$をそれぞれ積分下限、積分上限という。また、変数$x$を\textbf{積分変数}という。
                定積分は数値であるので、積分変数によってその結果は変わらない。すなわち
                \begin{equation}
                    \int_a^b f(x)dx = \int_a^b f(t)dt
                \end{equation}
                のように、積分変数を$x$としても$t$としても同じである。式\eqref{eq:定積分の定義}について、右辺の極限が存在するとき、
                $f(x)$は\textbf{積分可能である}という。$f(x)$が連続関数ならば常に積分可能である。

                先ほどの過程では$f(x)\geq0$としてきたが、もちろん$f\leq0$の場合でも同様に定義される。積和の定義から、$f\leq0$であれば
                定積分は``負''の面積を与えることがわかる。
                \clearpage
                さて、試しに$f(x)=c(\text{定数})$の場合の定積分を求めてみよう。これは長方形の面積を求めることになるから当然答えは$c(b-a)$である。
                \begin{equation*}
                    \int_a^b cdx = \lim_{n\to 0}\sum_{k=1}^{n} c\Delta x_k = c\lim\sum (x_{k}-x_{k-1})
                \end{equation*}
                あとは$\Sigma$の定義から
                \begin{equation*}
                    c\lim\sum (x_k-x_{k-1})=c\lim\left\{(x_1-x_0)+(x_2-x_1)+\cdots+(x_{n-1}-x_{n-2})+(x_n-x_{n-1})\right\}=c\lim (x_{n}-x_0)
                \end{equation*}
                $x_n=b,x_0=a$であるから結局
                \begin{equation}
                    \int_a^b cdx = c(b-a) \label{eq:int_cdx}
                \end{equation}
                となる。なお、上記のように$\sum,\lim$の記号を一部省略して書くことはしばし用いられる。このノートでも以降頻繁に用いる。

                $f(x)=c$というもっとも単純な場合でさえこの計算量なのだから、$f$がもっと複雑になると計算量も途方のないものになる。しかし安心してほしい。
                数値計算等を除いて式\eqref{eq:定積分の定義}を用いて積分を計算することはほとんどない。実はもっと賢いやり方が存在するのである。
            \clearpage
            \subsection{定積分の性質}
                ここでは定積分の諸性質についてまとめる。仮定として$f,g$は区間$[a,b]$で連続であるとする。
                \begin{enumerate}\setcounter{enumi}{0}\renewcommand{\labelenumi}{(\arabic{enumi})}
                    \item $\displaystyle \int_a^b \left\{f(x)\pm g(x)\right\}dx=\int_a^b f(x)dx\pm\int_a^b g(x)dx\label{eq:定積分の加法性}$
                    \item $\displaystyle \int_a^b cf(x)dx=c\int_a^b f(x)dx \quad (c\text{は定数})$
                    \item $x\in [a,b]$で$f\geq 0$ならば$\displaystyle\int_a^b f(x)dx\geq 0$
                    \item $x\in [a,b]$で$f\geq g$ならば$\displaystyle\int_a^b f(x)dx\geq \int_a^b g(x)dx\label{eq:関数と積分値の大小関係}$
                    \item $\displaystyle \int_a^b f(x)dx = f(c)(b-a)\quad (c\in (a,b))\label{eq:積分の平均値の定理}$
                    \item $\displaystyle \int_a^bf(x)dx=\int_a^c f(x)dx+\int_c^b f(x)dx \quad(c\in(a,b))\label{eq:定積分の区間についての加法性}$
                    \item $\displaystyle \int_a^b f(x)dx = -\int_b^a f(x)dx\label{eq:int[a,b]=-[b,a]}$
                    \item $\displaystyle \int_a^a f(x)dx=0\label{eq:int[a,a]=0}$
                \end{enumerate}
                式\eqref{eq:積分の平均値の定理}は(積分の)\textbf{第一平均値定理}\footnote{平均値の第一定理とも。文脈によっては単に平均値の定理とも呼ぶこともある。}と呼ばれる。性質\eqref{eq:int[a,b]=-[b,a]}は性質というよりは
                このように定める(定義)として考えるとよい。\footnote{線積分を学ぶと、この定義が見通しよくなる。なぜなら左辺は$C:a\to b$の曲線に沿った積分$\int_C$であるためである。}数学書の中にはこれを証明させるものもあるが気にしなくてよい。性質\eqref{eq:定積分の区間についての加法性}も同様である。
                性質\eqref{eq:定積分の加法性},\eqref{eq:int[a,b]=-[b,a]]}から性質\eqref{eq:関数と積分値の大小関係}と性質\eqref{eq:int[a,a]=0}は容易に証明できるので演習問題とする。よって\eqref{eq:積分の平均値の定理}とを示す。

                \paragraph{平均値の定理の証明}
                    $f(x)$が定数関数なら明らかに式\eqref{eq:積分の平均値の定理}が成り立つ。よって$f(x)$は区間$[a,b]$で最大値$M$と最小値$m$を取るとする。この時
                    \begin{equation*}
                        m\leq f(x) \leq M
                    \end{equation*}
                    である。性質\eqref{eq:関数と積分値の大小関係}より
                    \begin{equation*}
                        \int_a^b mdx \leq \int_a^b f(x)dx \leq \int_a^b Mdx
                    \end{equation*}
                    $m,M$は$x$によらない定数なので
                    \begin{equation*}
                        m(b-a) \leq \int f(x)dx \leq M(b-a)
                    \end{equation*}
                    各辺を$b-a$で割ると
                    \begin{equation*}
                        m \leq \frac{1}{(b-a)}\int_a^b f(x)dx \leq M
                    \end{equation*}
                    $f(x)$は$[a,b]$で連続関数だから、中間値の定理\eqref{eq:中間値の定理}より、
                    \begin{equation*}
                        \frac{1}{b-a}\int_a^b f(x)dx = f(c)
                    \end{equation*}
                    となるような点$c$が区間$(a,b)$内に存在する。よって、平均値の定理\eqref{eq:積分の平均値の定理}が証明された。$\square$
                \clearpage
                平均値の定理はしばし次のように書くこともある。
                \begin{equation}
                    \int_{a}^{b}f(x)dx = (b-a)f(a+\theta(b-a))\quad (0<\theta < 1) \label{eq:積分の平均値の定理θ}
                \end{equation}
                これは$\theta$をラグランジュの平均値の定理\eqref{eq:平均値の定理1}と同じように使っている。

                性質\eqref{eq:定積分の区間についての加法性}では$a<c<b$としているが、性質\eqref{eq:int[a,b]=-[b,a]}を規約することで、この条件がなくても
                常に\eqref{eq:定積分の区間についての加法性}が成り立つ。

                \vspace{\stretch{1}}
                \hrulefill

                余談だが、
                \begin{equation*}
                    \int_I dx = \int_{a}^{b}dx = b-a
                \end{equation*}
                は$I=[a,b]$の長さを表す。同様にして二重積分についても
                \begin{equation*}
                    \iint _D dx dy = S(D)
                \end{equation*}
                は領域$D$の面積$S(D)$を表す。ここまでくればもちろん三重積分についても
                \begin{equation*}
                    \iiint _D dx dy dz = V(D)  
                \end{equation*}
                が領域$D$の体積$V(D)$だと予想がつく。(実際そうである。)このノートや高専2年生では多重積分のタの字も出てこないが、
                こういうちょっとした知識を学ぶだけでも楽しい。
            \clearpage
            \subsection{微積分学の基本定理}
                ここでは解析学上もっとも重要な定理(言いすぎ?)である微積分学の基本定理について述べる。最も重要なのでていねいに導出することにする。

                まず、関数$f(x)$について、$F(x)$を次のように定める。
                \begin{equation*}
                    F(x)=\int_a^x f(x)dx
                \end{equation*}
                このとき、$F(x+\Delta x)$は区間の加法性\eqref{eq:定積分の区間についての加法性}より
                \begin{equation*}
                    F(x+\Delta x)=\int_a^{x+\Delta x}=\int_a^x + \int_{x}^{x+\Delta x}=F(x)+\int_x^{x+\Delta x}f(x)dx
                \end{equation*}
                よって、移項して平均値の定理\eqref{eq:積分の平均値の定理θ}より
                \begin{equation*}
                    F(x+\Delta x)-F(x)=\Delta xf(x+\theta\Delta x)\quad (0<\theta<1)
                \end{equation*}
                式を整理すると
                \begin{equation*}
                    f(x+\theta \Delta x) = \frac{F(x+\Delta x)-F(x)}{\Delta x}
                \end{equation*}
                ここで、$\Delta x\to 0$の極限を取ると
                \begin{equation*}
                    f(x)=F'(x)
                \end{equation*}
                すなわち
                \begin{equation}
                    f(x)=\frac{d}{dx}\int_a^x f(x)dx\label{eq:微積分学の基本定理}
                \end{equation}
                これが\textbf{微積分学の基本定理}である。この定理は微分と積分が互いに逆の演算であることを示している。
                すなわち$F$は$f$の原始関数になる。$f$が\underline{連続であれば}それが可能なのである。

                $F$が$f$の原始関数であるということは$F+C$も$f$の原始関数である。
                \begin{equation*}
                    F(x)+C = \int_a^x f(x)dx
                \end{equation*}
                ここで$x=a$とすると性質\eqref{eq:int[a,a]=0}より
                \begin{equation*}
                    F(a)+C=0 \leftrightarrow C=-F(a)
                \end{equation*}
                よって、$x=b$とすれば
                \begin{equation}
                    \int_a^b f(x)dx = F(b)-F(a) \label{eq:微積分学の第二基本定理} 
                \end{equation}
                これを\textbf{微積分学の第二基本定理}という。この表現に倣って\eqref{eq:微積分学の基本定理}は微積分学の第一基本定理と呼ばれる。

                式\eqref{eq:微積分学の第二基本定理}の右辺は次の便利な表記法がある。
                \begin{equation*}
                    \left[F(x)\right]_a^b = F(x)\left|_a^b\right. = F(b)-F(a)
                \end{equation*}
                \clearpage
                不定積分については、以前すでに不完全であるが定義した。そこで、本当の不定積分の定義を述べようと思う。そもそも、不定積分は
                定数分の不定性はなく、ある点$a$で固定して定義される。\footnote{本によっては、$\int_a^x$と$\int_b^x$は定数分の差しかないので下限を省略して$\int $を不定積分と定義するものもある。}
                \begin{equation}
                    F(x)=\int_{a}^{x}f(x)dx
                \end{equation}
                これが点$a$における不定積分の定義である。ちなみにイギリスかどこかの数学書では不定積分を
                \begin{equation*}
                    \int^x f(x)dx
                \end{equation*}
                と表記するらしい。

                定積分の計算で公式\eqref{eq:微積分学の第二基本定理}を用いるときは、積分範囲に気を付けなければならない。例えば、
                \begin{equation*}
                    \int_{-1}^{1} \frac{dx}{x} = \left[\log|x|\right]_{-1}^1=\log|1|-\log|-1|=0
                \end{equation*}
                と行うのは不適である。被積分関数が$x=0$で不連続になってしまうからである。多価関数の場合は、関数が連続になるように
                分枝を取らなければならない。
            \clearpage
            \subsection{定積分の計算}
                ここでは定積分の計算について述べる。
                \paragraph{置換積分法}
                    関数$x=\phi(t)$について、$\phi(t)$と$\phi'(t)$が$[\alpha,\beta]$で連続であり$a=\phi(\alpha),b=\phi(\beta)$とする。このとき
                    \begin{equation}
                        \int_{a}^{b}f(x)dx = \int_{\alpha}^{\beta}f(\phi(t))\phi'(t)dt\label{eq:置換積分_定積分}
                    \end{equation}
                    が成りたつ。これを置換積分という。

                    公式\eqref{eq:置換積分_定積分}の証明を行おう。
                    \begin{equation*}
                        F(x)=\int_{a}^{x}f(x)dx
                    \end{equation*}
                    と置くと
                    \begin{equation*}
                        \frac{dF}{dt}=\frac{dF}{dx}\frac{dx}{dt}=f(x)\phi'(t)
                    \end{equation*}
                    であるため、両辺を$\int_{\alpha}^{\beta}$で積分すると
                    \begin{equation*}
                        F(\phi(\beta))-F(\phi(\alpha))=\int_{\alpha}^{\beta}f(\phi(t))\phi'(t)dt
                    \end{equation*}
                    左辺は$F(b)-F(a)=\int_a^b $であるため、公式\eqref{eq:置換積分_定積分}を得る。

                \paragraph{部分積分法}
                    関数$f,g$について、どちらの関数も$[a,b]$で微分可能であり、$f',g'$が連続であれば
                    \begin{equation}
                        \int_a^b f(x)g'(x)dx = \left[f(x)g(x)\right]_a^b - \int_a^b f'(x)g(x)dx \label{eq:部分積分_定積分}
                    \end{equation}
                    が成り立つ。これを部分積分という。

                    公式\eqref{eq:部分積分_定積分}の証明を行う。まず、明らかに次が成り立つ。
                    \begin{equation*}
                        (fg)' = fg' + f'g
                    \end{equation*}
                    よって、移項すれば
                    \begin{equation*}
                        fg' = (fg)' - f'g
                    \end{equation*}
                    ここで両辺を$\int_a^b$で積分すれば公式\eqref{eq:部分積分_定積分}を得る。\\

                    まとめると、置換積分も部分積分も、不定積分のときとやっていることは変わらない。積分区間等に気を付けさえいれば不定積分と同様に
                    計算してよいのである。これも基本定理のもたらした恩恵である。
                \clearpage
                \subsection{広義積分}
                    ここでは定積分の定義を拡張した広義積分を扱う。具体的には次の二つの場合を扱う。
                    \begin{enumerate}
                        \item 積分区間内で不連続な点が有限個存在する
                        \item 積分上限・下限の片方もしくは両方がが無限大である
                    \end{enumerate}
                    今までの仮定では被積分関数は区間内で連続であったが、この広義積分では不連続な点(有限個)や区間が無限大など
                    広義の意味での定積分を考えるのである。そういう意味で広義積分というわけである。

                    \paragraph{被積分関数が不連続}
                        $f(x)$が$[a,b)$で連続であるとする。この時、極限
                        \begin{equation*}
                            \lim_{\varepsilon \to +0}\int_{a}^{b-\varepsilon}f(x)dx
                        \end{equation*}
                        が存在すれば、その極限値を
                        \begin{equation*}
                            \int_{a}^{b}f(x)dx
                        \end{equation*}
                        と表す。同様に$(a,b]$で連続である場合も、極限
                        \begin{equation*}
                            \lim_{\varepsilon\to +0}\int_{a+\varepsilon}^{b}f(x)dx
                        \end{equation*}
                        が存在すれば、その極限値を
                        \begin{equation*}
                            \int_{a}^{b}f(x)dx
                        \end{equation*}
                        このようにすれば$a\leq c \leq b$であるような$x=c$において$f(x)$が不連続である場合は、下の右辺の極限が存在すると仮定すると
                        \begin{equation*}
                            \int_{a}^{b}f(x)dx = \lim_{\varepsilon_1\to +0}\int_{a}^{c-\varepsilon_1}f(x)dx+\lim_{\varepsilon_2\to +0}\int_{c+\varepsilon}^{b}f(x)dx
                        \end{equation*}       
                        と計算すればよいことがわかる。このとき$\varepsilon_1,\varepsilon_2$はそれぞれ独立に極限を取ることに注意。\footnote{一方で$\varepsilon_1=\varepsilon_2$として極限を取ることもある。これをコーシーの主値積分という。}\\
                        上記の極限が収束するとき、\textbf{広義積分は収束する}という。\footnote{実際に広義積分が収束するのを示すためには、よくコーシーの収束条件を使われる。}\\

                        例えば、$1/\sqrt{x}$は$x=0$で不連続であるが、
                        \begin{equation*}
                            \int_{0}^{1}\frac{dx}{\sqrt{x}}=\lim_{\varepsilon\to +0}\int_{\varepsilon}^{1}\frac{dx}{\sqrt{x}}=\lim_{\varepsilon\to+0}\left[2\sqrt{x}\right]_{\varepsilon}^1=\lim_{\varepsilon\to +0}2-2\sqrt{\varepsilon}=2
                        \end{equation*}
                        よって、広義積分は収束してその値は$2$である。一方で$1/x$について$[-1,1]$で積分すると$x=0$で不連続であることに注意すれば
                        \begin{equation*}
                            \int_{-1}^{1}\frac{dx}{x}=\lim_{\substack{\varepsilon_1\to +0\\\varepsilon_2\to +0}}\int_{-1}^{-\varepsilon_1}\frac{dx}{x}+\int_{\varepsilon_2}^{1}\frac{dx}{x}=\lim_{\substack{\varepsilon_1\to +0\\\varepsilon_2\to +0}}\left(\log\varepsilon_1-\log\varepsilon_2\right)
                        \end{equation*}
                        である。この極限は存在しない(発散する)ので、広義積分は収束しない。つまりは$\displaystyle \int_{-1}^{1}\frac{dx}{x}$は無意味。
                    \clearpage
                    \paragraph{積分区間が無限大}
                        関数$f(x)$が$x\geq a$で連続であるとする。このとき極限
                        \begin{equation*}
                            \lim_{b\to \infty}\int_{a}^{b}f(x)
                        \end{equation*}
                        が存在すれば、それを
                        \begin{equation*}
                            \int_{a}^{\infty}f(x)dx
                        \end{equation*}
                        と表す。同様にして
                        \begin{align*}
                            \int_{-\infty}^{b}f(x)dx=\lim_{a\to -\infty}\int_{a}^{b}f(x)dx\\
                            \int_{-\infty}^{\infty}f(x)dx=\lim_{\substack{a\to -\infty\\b\to\infty}}\int_{a}^{b}f(x)dx
                        \end{align*}
                        も定義される。\\
                        上式の極限が収束するとき、\textbf{広義積分は収束する}という。\\

                        これも例を挙げてみよう。例えば
                        \begin{equation*}
                            \int_{0}^{\infty}\frac{dx}{1+x^2}=\lim_{b\to\infty}\int_{0}^{b}\frac{dx}{1+x^2}=\lim_{b\to\infty}\left[\arctan(x)\right]_0^b=\lim_{b\to\infty}\arctan b=\frac{\pi}{2}
                        \end{equation*}
                        と計算できる。ちなみにこの結果、すなわち$1/(1+x^2)$の$[0,\infty)$の広義積分が$\pi/2$になるという事実は覚えておくとよい。
                        一方で、
                        \begin{equation*}
                            \int_{1}^{\infty}\frac{dx}{\sqrt{x}}=\lim_{b\to\infty}\int_{1}^{b}\frac{dx}{\sqrt{x}}=\lim_{b\to \infty}\left[2\sqrt{x}\right]_1^b=\lim_{b\to \infty}2(\sqrt{b}-1)
                        \end{equation*}
                        は右辺の極限が発散するので、広義積分$\displaystyle\int_{1}^{\infty}\frac{dx}{\sqrt{x}}$は意味を持たないことがわかる。\\

                        今あげた計算を見れば、広義積分について大まかに理解できると思う。本来連続な区間でしか定義していない定積分を不連続点を含む区間や、無限区間で
                        定義することがわかればよい。どちらの場合にせよ、連続な区間に分けて最後に極限を取るのである。このことを\underline{十分理解できている}なら、実際の計算では
                        \begin{equation*}
                            \int_{0}^{\infty}\frac{dx}{1+x^2}=\left[\arctan x\right]_0^\infty = \frac{\pi}{2}
                        \end{equation*}
                        と略記してもよい。無限区間の積分は、この微分積分以外にも当たり前のように顔を出す。ぜひともこの機会に慣れておきたいものである。
                \clearpage
                \basicquestion 以下の問いに答えよ。
                    \paragraph{問1}定積分の性質\eqref{eq:定積分の加法性}を用いて性質\eqref{eq:関数と積分値の大小関係}を、性質\eqref{eq:int[a,b]=-[b,a]}を用いて性質\eqref{eq:int[a,a]=0}を示せ。

                    \paragraph{問2}定積分の定義より$\int_{0}^{1}xdx$を計算せよ。ただし$z_k=x_k$とし、$[0,1]$は丁度$n$等分するものとする。

                    \paragraph{問3}以下の定積分の値を求めよ。ただし$0\leq e < 1$である。\\
                    $(1)\displaystyle \int_{0}^{\pi}\sin x dx$\hspace{3mm}%それにしてもあのsinの曲線下の面積がこんなに簡単に求まるなんて、ただただ驚くばかりである。
                    $(2)\displaystyle \int_{0}^{\frac{1}{2}}\frac{dx}{\sqrt{1-x^2}}$\hspace{3mm}
                    $(3)\displaystyle \int_0^2 x^2e^{x}dx$\hspace{3mm}
                    $(4)\displaystyle \int_{-3}^3\left(x^5+x^4+x^3+x^2+5\right)dx$\hspace{3mm}
                    $(5)\displaystyle \int_0^\pi \frac{dx}{1+e\cos x}$

                    \paragraph{問4}次の広義積分を求めよ。\\
                    $(1)\displaystyle \int_{-1}^{-\infty}\frac{dx}{x^4}$\hspace{3mm}
                    $(2)\displaystyle \int_0^1 \log xdx$\hspace{3mm}
                    $(3)\displaystyle \int_0^1 \frac{dx}{x\log x}$\hspace{3mm}
                    $(4)\displaystyle \int_0^a \frac{dx}{\sqrt{ax-x^2}}\quad(a>0)$\hspace{3mm}
                    $(5)\displaystyle \int_0^\infty \frac{dx}{x^3+1}$

                    \paragraph{問5}以下の式を示せ。
                    \begin{equation*}
                        \int_{0}^{2\pi}\cos mx\cos nx dx = \begin{cases}
                            0 & (m\neq n)\\ \pi & (m=n)
                        \end{cases}
                    \end{equation*}

                    \paragraph{問6}次の積分(広義積分)が収束するための$k$の範囲を求めよ。
                    \begin{equation*}
                        \int_1^\infty \frac{dx}{x^{k}}
                    \end{equation*}

                    \paragraph{問7}関数$f(x)$が偶関数であることを$f=f_e$、奇関数であることを$f=f_o$と表記することにする。この時以下を示せ。
                    \begin{equation}
                        \int_{-a}^a f(x)dx = \begin{cases}
                            \displaystyle 2\int_{0}^{a}f(x)dx & (f=f_e) \\ 0 & (f=f_o)
                        \end{cases}\label{eq:定積分と偶関数・奇関数}
                    \end{equation}

                    \paragraph{問8}次の問いに答えよ。
                    \begin{enumerate}\setcounter{enumi}{0}\renewcommand{\labelenumi}{(\arabic{enumi})}
                        \item $\displaystyle \int_0^\pi \frac{\sin x}{1+\cos^2 x}dx$を求めよ。
                        \item $x\to\pi -x$の置換により$\displaystyle \int_{0}^{\pi}xf(\sin x)dx=2\int_{0}^{\pi}f(\sin x)dx$を示せ。\\ただし、$f(\sin x)$は$\sin x$の有理関数である。
                        \item $\displaystyle \int_{0}^{\pi} \frac{x\sin x}{1+\cos^2 x}dx$を求めよ。
                    \end{enumerate}
                \clearpage
                \section{積分法の応用}
                    \subsection{面積・体積} 
                        ここからは定積分の応用を述べる。まずは定積分を用いて面積・体積を求める。面積については、定積分の定義を述べる際に
                        触れたが、実は体積も求められる。

                        関数$f(x)$と$g(x)$について、区間$[a,b]$で$f(x)\geq g(x)$となるとき、曲線$y=f,y=g$と直線$x=a,x=b$で囲まれた図形の面積は
                        \begin{equation}
                            \int_{a}^{b} \left\{f(x)-g(x)\right\}dx \label{eq:曲線で囲まれた面積}
                        \end{equation}
                        特に、$f\leq 0$であれば
                        \begin{equation}
                            -\int_a^b f(x)dx \label{eq:-負の面積}
                        \end{equation}
                        例として、円の面積を求めてみよう。半径$r$の半円の上部$y\geq 0$の面積は
                        \begin{equation*}
                            S=\int_{-r}^{r}\sqrt{r^2-x^2}dx
                        \end{equation*}
                        で与えられる。$x=r\sin\theta$の置換を行うと、$dx=r\cos\theta d\theta$であるため
                        \begin{equation*}
                            S=\int_{-\frac{\pi}{2}}^{\frac{\pi}{2}}\sqrt{r^2-r^2\sin^2\theta}\cdot r\cos\theta d\theta=r^2\int_{-\frac{\pi}{2}}^{\frac{\pi}{2}}\cos^2\theta d\theta=2r^2\int_{0}^{\frac{\pi}{2}}\cos^2\theta d\theta
                        \end{equation*}
                        半角の公式$\displaystyle \cos^2\theta = \frac{1+\cos2\theta}{2}$であるため、
                        \begin{equation*}
                            S=2r^2\int_{0}^{\frac{\pi}{2}}\frac{1+\cos 2\theta}{2}d\theta=2r^2\left[\frac{\theta+\frac{1}{2}\sin 2\theta}{2}\right]_0^{\frac{\pi}{2}}=\frac{\pi}{2}r^2
                        \end{equation*}
                        よって、円の面積の公式$\pi r^2$が導けた。

                        次に体積を求めてみよう。x軸上のある点$x$において断面積$S(x)$であるとき、区間$[a,b]$での体積$V$は
                        \begin{equation}
                            V=\int_a^b S(x)dx \label{eq:体積の公式}
                        \end{equation}
                        で求まる。もちろんこれだけではよくわからないだろうから、図を見て理解しよう。
                        \begin{figure}[h]
                            \centering
                            \includegraphics[scale=0.3]{img/QuuNote/Sx_V.png}
                            \caption{立体の体積と断面積}
                        \end{figure}

                        まず、公式中の断面積というのはx軸に垂直な平面で切った時の面積となる。これに$dx$を掛けた量$S(x)dx$は
                        微小体積$dV$を表すことは容易に想像できる。その微小体積を$[a,b]$まで集めた($\int$した)ものが$V$となる。少し直感的な説明\footnote{安易に$dx$や$dV$といった量を出したが、これらはあいまいに使っていると感じると思う。ただ、$d$が「微小の」というニュアンスを持っていることはこれまでの話でなんとなく理解できるだろう。}になってしまったが、
                        ここはイメージさえできればよい。

                        公式\eqref{eq:体積の公式}を用いれば、$y=f(x)$をx軸で回転したときの回転体の公式も導出できる。これも図を見て考えればわかる。

                        \begin{figure}[h]
                            \centering
                            \includegraphics[scale=0.5]{img/QuuNote/rolling_Sx.png}
                            \caption{回転体の体積}
                        \end{figure}

                        図からわかるように、$y=f(x)$を回転させた回転体の断面積$S(x)$は円の面積の公式$\pi r^2$より$S(x)=\pi\left[f(x)\right]^2$となる。
                        したがって、回転体の体積$V$は
                        \begin{equation}
                            V=\pi\int_{a}^{b}\left[f(x)\right]^2 dx \label{eq:回転体の体積}
                        \end{equation}
                        で求まる。
                    \clearpage
                    \subsection{曲線の長さ}
                        次に曲線$y=f(x)$の``長さ''を求めるとする。こちらは直感的な説明ではなく、より厳密に述べることにする。
                        
                        まず、区間$[a,b]$に分点$x_0=a,x_1,x_2,\dots,x_{n-1},x_n=b$を定め、$\Delta x_k=x_k - x_{k-1},\Delta y_k = f(x_k)-f(y_{k-1})$とする。
                        この時、区間$[x_k,x_{k-1}]$の曲線の長さは($\Delta x_k$が十分小さいとすれば)$\displaystyle \Delta l_k = \sqrt{\Delta x_k^2+\Delta y_k^2}=\sqrt{1+\left(\frac{\Delta y_k}{\Delta x_k}\right)^2}\Delta x_k$と近似できる。
                        ここで平均値の定理\eqref{eq:平均値の定理}より、
                        \begin{equation*}
                            \Delta l_k=\sqrt{1+\left(\frac{\Delta y_k}{\Delta x_k}\right)}\Delta x_k = \sqrt{1+\left(\frac{dy}{dx}(z_k)\right)}\Delta x_k
                        \end{equation*}
                        となるような$x_{k-1}<z_k<x_k$が存在する。$y=f(x)$の区間$[a,b]$での曲線の長さは、分割の数$n$を限りなく大きくしたときの$\sum \Delta l_k$に等しいから
                        \begin{equation}
                            \int_{a}^{b} \sqrt{1+\left(\frac{dy}{dx}\right)^2}dx \label{eq:曲線の長さ}
                        \end{equation}
                        となる。

                        \begin{figure}[h]
                            \centering
                            \includegraphics[scale=0.35]{img/QuuNote/curve_length.png}
                            \caption{曲線の長さ}
                        \end{figure}

                        これを使って、半径$r$の円周の長さを求めてみる。円の方程式$x^2+y^2=r^2$より、$\displaystyle \left(\frac{dy}{dx}\right)^2=\frac{x^2}{r^2-x^2}$だから
                        \begin{equation*}
                            2\int_{-r}^{r}\sqrt{1+\frac{x^2}{r^2-x^2}}dx=4\int_{0}^{r}\frac{r}{\sqrt{r^2-x^2}}dx=4r\sin^{-1}\frac{r}{r}=2\pi r
                        \end{equation*}
                        最初の$\times 2$は、元の積分が半円の演習を求めていることによる。よって、円周の公式$2\pi r$が導けた。
                    \clearpage
                    \subsection{数値積分法}
                        最後に数値積分法について簡単であるが述べる。これまでの計算では原始関数が簡単に求まるものばかりであったが、実際は原始関数を求めるのが難しいものもある。むしろそのほうが多いといってもいい。
                        しかし、原始関数が求められなくても定積分なら(近似して)求められることもある(定義\eqref{eq:定積分の定義}参照)。
                        ここでは近似計算の公式を紹介し、実際に面積を近似してみよう。なお、以降の計算では計算量の関係から関数電卓を用いることを
                        強くすすめる。

                        \paragraph{矩形法}まずはもっとも単純な矩形法から述べる。これは積分区間をある一定の幅(\textbf{刻み幅})で区切り、
                        被積分関数を、その毎区間の左端での関数値で一定と近似して定積分を求める方法である。刻み幅を$h$と表すことにすれば、この表式は次のようになる。
                        \begin{equation}
                            \int_{a}^{b}f(x)dx \approx \sum_{k=1}^{(b-a)/h}f(a+(k-1)h)\cdot h \label{eq:矩形法}
                        \end{equation}
                        こういうのは数式や説明を見るより、実際の計算を見たほうが理解しやすいだろう。$f(x)=x^2$の$[0,2]$の定積分を
                        矩形法で計算してみる。刻み幅は$h=0.5$とする。
                        \begin{equation*}
                            \int_{0}^{2}x^2 dx \approx f(0)\cdot h + f(0.5)\cdot h+f(1.0)\cdot h + f(1.5)\cdot h = 0+0.125+0.5+1.125=1.75
                        \end{equation*}
                        よって近似値は$1.75$である。実際の値は$\frac{8}{3}\approx 2.67$なので、精度はあまりよくない。刻み幅が大きすぎるのも原因であるが、矩形法自体近似計算としてよい計算方法ではないのもある。

                        \paragraph{台形公式}次に、面積を台形の形で近似する方法を述べる。矩形法では長方形で近似するので、区間の右端の値と$f(a+(k-1)h)$との差が大きいと誤差が大きくなるのは容易に想像できるだろう。
                        しかし、台形で近似すれば、その誤差は三角形$(a+(k-1)h,f(a+(k-1)h)),(a+kh,f(a+(k-1)h)),(a+kh,f(a+kh))$の分だけ小さくなるのである。
                        
                        \begin{figure}[h]
                            \centering
                            \includegraphics[scale=0.5]{img/QuuNote/trapezoidalRule.png}
                            \caption{矩形法(青)と台形近似(橙)の比較}\label{fig:矩形法と台形公式}
                        \end{figure}

                        分割の数を$n$とし、刻み幅を$h=\frac{b-a}{n}$とすると、図\ref{fig:矩形法と台形公式}からもわかるように各台形の面積は
                        \begin{equation*}
                            \frac{1}{2}h(f(a+(k-1)h)+f(a+kh))
                        \end{equation*}
                        で表せられるのだから、この$n$個の台形面積の和を取れば
                        \begin{equation*}
                            \frac{1}{2}h(f(a)+f(a+h))+\frac{1}{2}h(f(a+h)+f(a+2h))+\cdots+\frac{1}{2}h(f(a+(n-1)h)+f(b))
                        \end{equation*}
                        したがって、以下の\textbf{台形公式}が得られる。
                        \begin{equation}
                            \int_{a}^{b}f(x)dx \approx \frac{h}{2}\left\{f(a)+2f(a+h+2f(a+2h)+\cdots+2f(a+(n-1)h)+f(b))\right\}\label{eq:台形公式}
                        \end{equation}

                        \paragraph{シンプソンの公式}最後にシンプソンの公式について述べる。こんどは面積を$n=2m$個の帯に分割する。もちろん刻み幅$h=(b-a)/n=(b-a)/2m$である。
                        この帯を2つで一つのセットと考えれば、合計$m$個のセットができる。それぞれのセットについて、各点$P_{2k-2},P_{2k-1},P_{2k}$を通る放物線で近似することを考えると、
                        面積は以下の式で与えられる。
                        \begin{equation}
                            \frac{h}{3}\left\{f(a+(2k-2)h)+4f(a+(2k-1)h)+f(a+2kh)\right\} \label{eq:シンプソン各面積}
                        \end{equation}
                        
                        \begin{figure}[h]
                            \centering
                            \includegraphics[scale=0.5]{img/QuuNote/simpsonLow.png}
                            \caption{シンプソンの公式}
                        \end{figure}

                        なぜなら、三点$\displaystyle Q_0(\alpha,y_0),Q_1\left(\frac{\alpha+\beta}{2},y_1\right),Q_2(\beta,y_2)$を通る放物線$y=Ax^2+Bx+C$の面積は
                        \begin{equation*}
                            \int_{\alpha}^{\beta}ydx = \int_{\alpha}^{\beta} \left\{Ax^2+Bx+C\right\}dx = \frac{(\beta-\alpha)}{3}\left\{A(\alpha^2+\alpha\beta+\beta^2)+\frac{3}{2}+B(\beta+\alpha)+3C\right\}
                        \end{equation*}
                        となり、放物線は$Q_0,Q_1,Q_2$を通るから
                        \begin{align*}
                            y_0 &= A\alpha^2+B\alpha+C \\
                            y_1 &= A\left(\frac{\alpha+\beta}{2}\right)^2+B\left(\frac{\alpha+\beta}{2}\right)+C\\
                            y_2 &= A\beta^2+B\beta+C 
                        \end{align*}
                        よって、
                        \begin{equation*}
                            y_0+4y_1+y_2=2\left\{A(\alpha^2+\alpha \beta+\beta^2)+\frac{3}{2}B(\alpha+\beta)+3C\right\}
                        \end{equation*}
                        したがって、
                        \begin{equation*}
                            \int_{\alpha}^{\beta}ydx=\frac{\beta-\alpha}{3}\left(y_0+4y_1+y_2\right)
                        \end{equation*}
                        が得られる。だから$Q_0=P_{2k-2},Q_1=P_{2k-1},Q_2=P_{2k}$と考えれば、各面積は式\eqref{eq:シンプソン各面積}で与えられるのである。

                        長々と書いてしまったが、$m$個の面積を上記のように近似すれば、\textbf{シンプソンの公式}が得られる。
                        \begin{equation}
                            \begin{split}
                                \int_{b}^{a}f(x)dx \approx &\frac{h}{3}\{f(a)+4f(a+h)+2f(a+2h)+4f(a+3h)+2f(a+4h)\\ &+\cdots+2f(a+(2m-2)h)+4f(a+(2m-1)h)+f(b)\}
                            \end{split}\label{eq:シンプソンの公式}
                        \end{equation}\\

                        今回は、数値積分法として三つの方法\footnote{なお、高専の情報基礎では矩形法のみ扱う。先生はシンプソンの公式などを紹介したそうであったが。}を紹介したが、式の複雑さからもわかるように最も誤差が小さく計算できるのはシンプソンの公式である。
                        プログラミングの知識がある人はそれぞれの方法で計算して誤差を比較してみると楽しい。もちろん上記以外の数値積分の方法も存在する。気になったら調べてみるとよい。
                    \clearpage
                    \basicquestion 以下の問いに答えよ。

                    \paragraph{問1}楕円$\displaystyle\frac{x^2}{a^2}+\frac{y^2}{b^2}=1\quad (a,b>0)$の面積を求めよ。

                    \paragraph{問2}$y^2=4px$と$x^2=4py\quad (p>0)$で囲まれた図形の面積を求めよ。

                    \paragraph{問3}半径$r$の球の体積の公式を導け。

                    \paragraph{問4}カテナリー$y=\cosh x$の$-1\leq x\leq 1$での長さを求めよ。

                    \paragraph{問5}底面の半径$r$、高さ$h$の円錐の体積の公式を導け。

                    \paragraph{問6}$\displaystyle \int _0^1 \frac{4}{1+x^2}dx$を台形公式とシンプソンの公式を用いて求めよ。ただし、$n=4$とする。








                
                



    \clearpage
    \part{無限級数$\sum$}
    \clearpage
    \part{終わりに}
        このノートでは、(高専2年生で扱う)一変数の微分積分+無限級数について主に扱った。ここに書いてあることが大抵わかれば、
        高専2年の数学程度は楽勝であるはずである。数学書も入門書レベルなら読めるはずである。(...読めてほしい。)
        
        さて、何度も言うようにここでは``一変数''関数の微分積分についてのみ扱った。つまり多変数関数についての微分積分は言及してないわけである。
        多変数の微分積分は一変数と違うところが多々ある。例えば、二変数関数について極限を取るときに、一変数の場合はx軸上での近づき方のみ考えていたが
        二変数関数となるとxy`平面'での近づき方を考える必要がある。つまり、x軸上,y軸上に沿って近づく場合以外に$x,y$が同時に動く場合も考えなければならない。
        微分についても多変数の場合は\textbf{偏微分}という名前になり、記号として$\partial$を用いるようになる。積分についても、例えば二変数の場合は積分領域が
        平面となる。

        多変数の微分積分の先にはどんな学問があるのかという話だが、主に「微分方程式」「複素関数論」「ベクトル解析」「フーリエ解析」があげられる。\\

        話は変わるが、このノートを読んでもっと厳密なことを勉強したい、先のことを勉強したい人のためにこのノートを作る際に参考にした書籍を一部上げることにする。
        \begin{enumerate}\setcounter{enumi}{0}\renewcommand{\labelenumi}{(\arabic{enumi})}
            \item \href{https://www.iwanami.co.jp/book/b482316.html}{\textbf{理工系の数学入門コース 新装版 微分積分}}\\ノートで紹介した内容に加えて多変数の微分積分まで扱っている。私が初めて読んだ本でもあり、初学者でも読みやすくなっている。
            このノートの大部分はこの本に影響されているといっても過言ではない。これの姉妹図書である演習版も併用するとなおよい。コラムも面白い。
            \item \href{https://www.iwanami.co.jp/book/b265489.html}{\textbf{解析概論}}\\言わずと知れた解析学の名著。著者は高木貞二。理工学生は一度はこの本を読むべきとまで言われる。実際読んでみると、確かにわかりやすい。
            私自身全部読み切れてはいないが、(それでも)一見の価値ありだと思われる。この本から学ぶことも多かった。
            \item \href{https://www.dainippon-tosho.co.jp/college_math/differential1.html}{\textbf{新 微分積分}I \textbf{改訂版}}\\高専2年生で使われる教科書。一年生の内容から円滑に理解できるように工夫されている。
            しかし、極限の定義があいまいであったり、積分が公式ゲーみたいになっているところはあまりよくない。
            \item \href{https://www.iwanami.co.jp/book/b378350.html}{\textbf{解析入門 上}}\\この本は一人で独習できるように構成されており、微分積分の本としては内容も多い。このノートの内容は上巻に当たり、中巻下巻は、多変数関数の微分積分や
            フーリエ展開、複素解析、微分形式、ルベーク積分などの解析系に加えて、集合論や線形代数の基礎までもが詰まっている。これを読んで高専の数学に挑んだらオーバーキルな気がする。
            このノート程度の内容が大体頭に入っていれば読みやすいと思う。いろいろな事柄についてこのノートより断然詳しく、厳密に扱っている。
        \end{enumerate}
    \clearpage
    \section{他分野への展望}
        ここでは冒頭で述べた分野について軽く説明していく。
        \begin{enumerate}
            \item \textbf{多変数の微分積分}\\偏微分や多重積分など多変数関数に関する微分積分を扱う。これらの概念は以降の分野でも頻繁に登場する。
            \item \textbf{ベクトル解析}\\ベクトルという量について、微分積分を含めた様々な演算を定義する。電磁気学などの物理でこれらの知識が用いられる。これらを用いることで空間上の曲線などについてその曲がり具合等が調べられたり、物理の場の概念が理解できるようなる。また、線積分という新たな積分も登場する。
            線形代数とも関連する。
            \item \textbf{複素関数論}\\今までは変数が実数である関数について扱ってきたが、これを複素数に拡張した場合の微分積分等を扱う。複素関数の積分は線積分で定義される。また無限級数も複素数に拡張して考えていく。$\zeta$関数との関連も深い。留数定理という非常に強力な定理を導出する。これを使って実数の定積分を簡単に計算できる。
            \item \textbf{常微分方程式}\\未知関数が変数を一つのみ持つ場合の微分方程式について、その計算方法を学ぶ。例を上げると$f'(x)+g(x)=0$などがある。この辺の知識は物理で必須だと思う。
            \item \textbf{フーリエ解析}\\よく見る「波」についてフーリエ級数やフーリエ変換を用いて明らかにする。関数が$\sin$と$\cos$の和で表せる、という事実はいつ聞いても衝撃である。
        \end{enumerate}
        ここでは主なものを上げたが、もちろんこれ以外にも様々な分野がある。これらを総称して\textbf{解析学}という。解析学は幾何学・代数学を含めた数学の三大分野の一角をなす。
        もっと詳しく知りたい場合は\href{https://qr.paps.jp/8Mfcc}{wikipedia}等を参照するとよい。
    \clearpage
    \section{模範解答}
        次ページから問題の模範解答を示す。
        \clearpage\color{red}
        \subsection{前提知識 基本問題解答}
            \basicanswer 
                \paragraph{問1}以下の主張のうち正しいものには〇を、間違っているものには×をつけよ。
                \begin{enumerate}
                    \item $\sqrt{9}$は無理数である。
                    \item 有理数は全て分子分母が整数である分数の形で表せる。
                    \item $i$は複素数である。
                    \item 有理数の集合は$\mathbb{Q}$として表し、無理数の集合は$\mathbb{N}$で表す。
                    \item 自然数全体の集合(区間)は$(0,\infty]$である。
                \end{enumerate}
                正しいのは1,2,3、間違っているのは4,5
                \paragraph{問2}以下の区間について、数直線上に示せ。もし数直線上に記されていない数字が出てくる場合はそれも記載せよ。
                    \begin{figure}[h]
                        \centering
                        \includegraphics[keepaspectratio,scale=0.6]{img/QuuNote/kihonmondaikaitouzu_1.png}
                        \caption{数直線}
                    \end{figure}

                    $1.\quad [2,3]$\hspace{3mm}
                    $2.\quad (3,5)$\hspace{3mm}
                    $3.\quad [-5,\pi]$\hspace{3mm}
                    $4.\quad (-2,0.5]$\hspace{3mm}
                    $5.\quad [-1,0)$\hspace{3mm}
                    $6.\quad (-\infty,0)$\hspace{3mm}
                    $7.\quad [0,\infty)$\hspace{3mm}
            \clearpage
            \basicanswer
                \paragraph{問1}次の関数が偶関数か奇関数かを判別せよ。

                \noindent
                (1)偶\hspace{3mm}
                (2)奇\hspace{3mm}
                (3)奇\hspace{3mm}
                (4)偶\hspace{3mm}
                (5)奇\hspace{3mm}
                (6)どちらでもない\hspace{3mm}
                (7)偶\hspace{3mm}
                (8)奇

                \paragraph{問2}以下の等式を証明せよ。

                \noindent
                $(1)$加法定理で$\alpha=\beta=x$と置く。$\sin(2x)=\sin x\cos x+\cos x\sin x=2\sin x\cos x$\hspace{1mm}$\cos 2x$も同様。\\
                $(2)$倍角の公式より、$\cos 2x=\cos^2 x-\sin^2 x=1-2\sin^2 x$変形して$x$に$\frac{x}{2}$を代入すれば半角の公式が得られる。$\cos^2 \frac{x}{2}$も同様。\\
                $\displaystyle(3)\sinh(x+y)=\frac{e^{x+y}-e^{-(x+y)}}{2}=\frac{e^{x+y}-e^{x-y}+e^{x-y}-e^{-x-y}}{2}=\frac{e^{x}(e^{y}-e^{-y})+e^{-y}(e^x-e^{-x})}{2}=e^x\sinh y+e^{-y}\sinh x=2\cosh x\sinh y -e^{-x}\sinh y+2\cosh y\sinh x -e^{y}\sinh x$\\
                ここで$\displaystyle e^{-x}\sinh y+e^{y}\sinh x=\frac{e^{y-x}-e^{-y-x}}{2}+\frac{e^{x+y}-e^{y-x}}{2}=\sinh(x+y)$である。\\
                よって、$\displaystyle \sinh (x+y)=2\cosh x\sinh x+2\cosh y\sinh x-\sinh (x+y)\to \sinh(x+y)=\cosh x\sinh x+\cosh y\sinh x$となる。$\cosh (x+y)$についても同様。
                
                \paragraph{問3}以下の値を求めよ。

                \noindent
                $(1)0$\hspace{3mm}
                $(2)\frac{\pi}{6}$\hspace{3mm}
                $(3)0$\hspace{3mm}
                $(4)3$\hspace{3mm}
                $(5)0$\hspace{3mm}
                $(6)1$\hspace{3mm}
                $(7)\frac{\pi}{2}-\cos\log 3\sin\log 2$
                
                \paragraph{問4}$I=271\times 314$と置けば、$\log_{10} I=\log_{10}2.71+\log_{10}3.14+4$である。よって、$\log_{10}I=4.9315$より$I=10^{4.9315}\simeq 85408$
                
                \paragraph{問5}$t=\tan\frac{x}{2}$とするとき、$\sin x,\cos x,\tan x$をそれぞれ$t$を用いた式で表せ。
                \begin{align*}
                    \sin x &=2\sin \frac{x}{2}\cos \frac{x}{2}=\frac{2\tan\frac{x}{2}}{\frac{1}{\cos^2\frac{x}{2}}}=\frac{2t}{1+t^2}\\
                    \cos x &=\cos^2 \frac{x}{2} -\sin^2 \frac{x}{2}=\frac{1-\tan^2\frac{x}{2}}{\frac{1}{\cos ^2\frac{x}{2}}}=\frac{1-t^2}{1+t^2}\\
                    \tan x &=\frac{\sin x}{\cos x}=\frac{\displaystyle\frac{2t}{1+t^2}}{\displaystyle\frac{1-t^2}{1+t^2}}=\frac{2t}{1-t^2}
                \end{align*}
            \clearpage
            \basicanswer 
            
                \paragraph{問1}以下の数列の一般項を示し、それらが収束するかどうか答えよ。\\
                $(1)\frac{n+1}{n}=1+\frac{1}{n}$収束する\hspace{3mm}
                $(2)\frac{1}{3^n}$収束する\hspace{3mm}
                $(3)(-1)^{n+1}$発散する\hspace{3mm}
                $(4)a+(n-1)d$発散する

                \paragraph{問2}仮定より$\displaystyle |a-a_n|<\frac{\varepsilon}{2}\quad(n>N_1),|b-b_n|<\frac{\varepsilon}{2}\quad(n>N_2)$となる$N_1,N_2$が存在する。ここで、$N_1,N_2$のうち大きい方をとり
                \footnote{ここで大きい方を取ることで、必ず仮定の条件を満たす。例えば$N_1>N_2$であるとき$N=N_1$と取れば、どちらの数列に対しても$\ldots<\frac{\varepsilon}{2}$となるが、$N=N_2$と取ってしまうと$N_1>n>N_2$であるような$n$に対して$|a-a_n|<\frac{\varepsilon}{2}$が保証できないのである。}
                、それを$N=\max(N_1,N_2)$と置けば$n>N$のとき
                \begin{equation*}
                    |(a+b)-(a_n+b_n)|=|(a-a_n)+(b-b_n)|\leq |a-a_n|+|b-b_n|<\frac{\varepsilon}{2}+\frac{\varepsilon}{2}=\varepsilon
                \end{equation*}
                となるような$N$が存在する。よって、証明完了。

                \paragraph{問3}以下計算せよ。\\

                \noindent
                $(1)2$\hspace{3mm}
                $(2)4\pi^2$\hspace{3mm}
                $(3)0$\hspace{3mm}
                $(4)2$\hspace{3mm}
                $(5)0$\hspace{3mm}
                $(6)-1$\hspace{3mm}
                $(7)\frac{1}{2}$\hspace{3mm}
                $(8)1$
                
                \paragraph{問4}次の関数が()内の点において連続であるかどうか調べよ。\\
                $(1)$連続\hspace{1mm}
                $(2)$連続\hspace{1mm}
                $(3)$連続\hspace{1mm}
                $(4)$不連続\hspace{1mm}
                $(5)$連続\\
                $(6)\displaystyle \lim_{x\to 0}f(x)=\lim_{x\to 0}x\sin\frac{1}{x}$に注意する。また、$-1\leq\sin\frac{1}{x}\leq 1$であるため、$-x\leq f(x) \leq x$
                ここで$x\to 0$の極限を取れば$x,-x\to 0$なのではさみうちの原理より$f(x)\to 0.$これは$f(0)=1$と一致しないので、\underbar{不連続}

                \paragraph{問5}方程式$\sin x=x$が区間$[0,\frac{\pi}{2}]$に実数解をもつかどうか調べよ。

                $x=0$で等号が成り立つので実数解は存在する。\\
                (補足)$\cos x=x$の場合でも実数解をもつか考えてみることにする。
                $f(x)=\cos x-x$と置くと、$f(x)$は$[0,\frac{\pi}{2}]$で連続であり
                \begin{equation*}
                    f(0)=1>0,\quad f\left(\frac{\pi}{2}\right)=-\frac{\pi}{2}<0
                \end{equation*}
                したがって中間値の定理より区間$[0,\frac{\pi}{2}]$に$\cos x=x$は実数解をもつ。
        \clearpage
        \subsection{微分法 基本問題解答}
            \basicanswer
                \paragraph{問1}次の関数の(\hspace{2mm})内の区間での平均変化率を求めよ。\\
                $(1)4$\hspace{3mm}
                $(2)2$\hspace{3mm}
                $(3)\frac{\sqrt{3}-1}{\pi/3}$

                \paragraph{問2}次の関数を微分せよ。\\
                \noindent
                $(1)y'=1$\hspace{5mm}
                $(2)y'=0$\hspace{5mm}
                $(3)y'=3x^2$\hspace{5mm}
                $(4)y'=a$\hspace{5mm}
                $(5)y'=2a(x+p)q$

                
                \paragraph{問3}$y=\sqrt{x}\hspace{1mm}(x\geq 0)$が微分可能であるか調べよ。

                $x=0$について微分可能性を調べると
                \begin{equation*}
                    \lim_{h\to +0}\frac{\sqrt{0+h}-\sqrt{0}}{h}=\lim_{h\to+0}\frac{\sqrt{h}}{h}=\lim_{h\to+0}\frac{1}{\sqrt{h}}
                \end{equation*}
                この極限は存在しないので、$\sqrt{x}$は($x=0$で)微分可能ではない。

                \paragraph{問4}
                \begin{equation*}
                    \frac{d}{dx}(af(x))=\lim_{h\to 0}\frac{af(x+h)-af(x)}{h}=a\lim_{h\to 0}\frac{f(x+h)-f(x)}{h}=a\frac{d}{dx}f(x)
                \end{equation*}
            \clearpage
            \basicanswer
                \paragraph{問1}以下の関数の導関数を求めよ。\\
                $(1)y'=2x$\hspace{2mm}
                $(2)y'=-\sin x+\cos x$\hspace{2mm}
                $(3)y'=\frac{1}{x}$\hspace{2mm}
                $(4)y'=\frac{-1}{\sqrt{1-x^2}}$\hspace{2mm}
                $(5)y'=\frac{e^{\tan x}}{\cos^2 x}$\\
                $(6)y'=2e^{x^2}(x\cos2x-\sin 2x)$\hspace{3mm}
                $(7)y'=\frac{1}{\sqrt{x^2+1}}$\hspace{3mm}
                $(8)y'=\frac{1}{4\sqrt[4]{x^3}}$

                \paragraph{問2}以下を示せ。\\
                (1)$\displaystyle\frac{\frac{f(x+h)}{g(x+h)}-\frac{f}{g}}{h}=\frac{f(x+h)g-fg(x+h)}{hgg(x+h)}=\frac{g\cdot(f(x+h)-f)-f\cdot(g(x+h)-g)}{hgg(x+h)}$として$h\to0$の極限を取る。もしくは積の微分公式を利用してもよい。\\
                (2)$\displaystyle\frac{e^{a(x+h)}-e^{ax}}{h}=ae^{ax}\frac{e^{ah}-1}{ah}$であり、$h\to0$のとき$ah\to 0$であるため、$(e^{ax})'=ae^{ax}$\\
                (3)$\sin^2 x+\cos ^2 x=1$の両辺を$x$微分すると$2\sin x\cos x+2\cos x(\cos x)'=0$仮定より$\cos x\neq 0$なので、両辺を$2\cos x$で割って移項すると$(\cos x)'=-\sin x$

                \paragraph{問3}以下の関数の導関数を工夫して求めよ。\\
                $\displaystyle(1)y'=\frac{2(x-1)}{3(x+1\sqrt[3]{(x+1)^2(x^2+1)^2})}$\hspace{3mm}
                $(2)y'=x^x(\log x+1)$\\
                $(3)x=\tan \theta$と置くと、$\displaystyle y=\sin^{-1}\frac{\tan \theta}{\sqrt{1+\tan^2\theta}}=\theta$なので、\\
                $\displaystyle \frac{dy}{dx}=\frac{dy}{d\theta}\frac{d\theta}{dx}=1\cdot\frac{1}{\frac{dx}{d\theta}}=\cos^2 \theta=\frac{1}{1+\tan^2\theta}=\frac{1}{1+x^2}$よって$\displaystyle y'=\frac{1}{1+x^2}$

                \paragraph{問4}双曲線関数について、$\sinh x,\cosh x$の導関数を導出せよ。

                $\sinh x,\cosh x$の定義から求めよ。$(\sinh x)'=\cosh x,(\cosh x)'=\sinh x$

                \paragraph{問5}
                \begin{enumerate}\setcounter{enumi}{0}\renewcommand{\labelenumi}{(\arabic{enumi})}
                    \item 右辺に加法定理を適用して示せ。
                    \item $\displaystyle\cos\alpha\cos\beta=\frac{1}{2}\{\cos(\alpha+\beta)+\cos(\alpha-\beta)\}$
                    \item $\displaystyle\sin\alpha\sin\beta=-\frac{1}{2}\{\cos(\alpha+\beta)-\cos(\alpha-\beta)\}$
                \end{enumerate}
            \clearpage
            \basicanswer 以下の問いに答えよ。

                \paragraph{問1}次の関数のグラフをかけ。\\
                省略

                \paragraph{問2}次の関数の(\hspace{1mm})内での最大・最小を求めよ。\\
                増減表を書いて求めよ。\\
                $(1)$最大値:$\left(\frac{3}{5}\right)^{\frac{5}{2}}-\left(\frac{3}{5}\right)^{\frac{3}{2}}$ 最小値:$-\left(\frac{3}{5}\right)^{\frac{5}{2}}+\left(\frac{3}{5}\right)^{\frac{3}{2}}$\hspace{3mm}
                $(2)$最大値:$e^2\hspace{1mm}(x=2)$ 最小値:$-1\hspace{1mm}(x=0)$

                \paragraph{問3}$e^x \geq x+1 \quad (x\geq0)$を示せ。\\
                $f(x)=e^x-x+1$と置く。この時、$f'(x)=e^x-1=0\leftrightarrow x=0$であり、$x>0$で$f'(x)>0$であるので、$f(x)$は
                $x>0$で単調増加する。よって、$f(x)$の最小値は$x=0$のときであり、その値は$f(0)=e^0-0+1=0$である。
                すなわち$f(x)\geq 0\leftrightarrow e^x\geq x+1$

                \paragraph{問4}図\ref{fig:直流回路}の直流回路について、最初に流れる電流$I$は以下の式で表される。
                \begin{equation*}
                    I=\frac{E}{R_0+R}\quad(\text{オームの法則})
                \end{equation*}
                また、可変抵抗器$R$で消費される電力$P$は$P=I^2R$である。この時、可変抵抗器で消費される最大の電力$P_{max}$
                の値を求めよ。また、この時の可変抵抗器の値を求めよ。

                \begin{equation*}
                    \frac{dP}{dR}=\frac{d}{dR}\left(\frac{E^2R}{(R_0+R)^2}\right)=\frac{E^2(R_0+R)^2-E^2R\cdot2(R_0+R)}{(R_0+R)^4}=\frac{E^2(R_0-R)}{(R_0+R)^3}=0
                \end{equation*}
                $R_0>R>0$のとき$P'>0$であり$R>R_0$のとき$P'<0$なので、$P$は$R=R_0$で最大値を取る。またその値は
                \begin{equation*}
                    P_{max}=\frac{E^2R_0}{(R_0+R_0)^2}=\frac{E^2}{4R_0}
                \end{equation*}
                一般に、直流回路網は図\ref{fig:直流回路}の回路の回路に直せるので、その際にこの公式が使われる。
            \clearpage
            \basicanswer
                \paragraph{問1}次の関数の三次導関数を求めよ。\\
                $(1)y'''=48$\hspace{3mm}
                $(2)\displaystyle y'''=\frac{-6}{(1+x)^4}$\hspace{3mm}
                $(3)y'''=-8\cos(2x)$\hspace{3mm}
                $(4)y'''=e^x(x+3)$\\
                $\displaystyle(5)y'''=\frac{12\sin(2x)}{(1+x)^2}-\frac{6\sin(2x)}{(1+x)^4}-\frac{8\cos(2x)}{1+x}+\frac{12\cos(2x)}{(1+x)^3}$

                \paragraph{問2}次の関数の$n$次導関数を求めよ。\\
                $(1)y^{(n)}=(-1)^ne^{-x}$\hspace{3mm}
                $(2)y=\cos\left(x+\frac{n\pi}{2}\right)$\hspace{3mm}
                $(3)y=n!$\hspace{3mm}
                $(4)y=\frac{(-1)^{n-1}(n-1)!}{(x+1)^{n}}$

                \paragraph{問3}次の極限をロピタルの定理を用いて求めよ。\\
                $(1)\frac{1}{2}$\hspace{2mm}
                $(2)1$\hspace{2mm}
                $(3)0$\hspace{2mm}
                $\displaystyle(4)\lim_{x\to+0}x^x=e^{x\log x}=\lim_{x\to+0}e^{\frac{\log x}{\frac{1}{x}}}$と変形。$\displaystyle\lim_{x\to+0}\frac{\log x}{\frac{1}{x}}=\lim_{x\to+0}x=0$より、答えは$e^{0}=1$
                $\displaystyle(5)\lim_{x\to\infty}\log(1+e^x)^{\frac{1}{x}}=\lim_{x\to\infty}\frac{\log(1+e^x)}{x}=\lim_{x\to\infty}\frac{e^x}{1+e^x}=\lim_{x\to\infty}\frac{e^x}{e^x}=1$

                \paragraph{問4}定数項がない$x$の多項式について$n\to0,\infty$で減少の速さが$x$で抑えられるとき$P(x)$と書く。\footnote{これはあいまいな表現であり、もっと厳密にランダウの記号として数学で定義されている。ここではfeelingで書いた。ちなみにこのランダウは理論物理学で有名なランダウではなく、数学者のランダウである。}\\
                \vspace{1mm}$(1)$$\sin x=x-\frac{1}{3!}x^3+P(x^5)$と書けるので\\
                $\displaystyle\lim_{x\to 0}\frac{x-\sin x}{x^3}=\lim_{x\to 0}\frac{x-\left(x-\frac{1}{3!}x^3+P(x^5)\right)}{x^3}=\lim_{x\to 0}\frac{\frac{1}{3!}x^3-P(x^5)}{x^3}=\lim_{x\to 0}\left\{\frac{1}{3!}-P(x^2)\right\}=\frac{1}{3!}$\\
                \vspace{3mm}\noindent
                $(2)\displaystyle\sqrt{x^2-3x+1}=x\sqrt{1-\left(\frac{3}{x}-\frac{1}{x^2}\right)}$と変形する。このとき$\sqrt{1-x}=1-\frac{1}{2}x-\frac{1}{2!}\cdot\frac{3}{4}x^2\cdots$より\\
                $\displaystyle x\sqrt{1-\left(\frac{3}{x}-\frac{1}{x^2}\right)}=x\left\{1-\frac{1}{2}\left(\frac{3}{x}-\frac{1}{x^2}\right)-\frac{1}{2!}\cdot\frac{3}{4}\left(\frac{3}{x}-\frac{1}{x^2}\right)^2+\cdots\right\}=x\left\{1-\frac{1}{2}\times\frac{3}{x}-P\left(\frac{1}{x^2}\right)\right\}$\\
                $\displaystyle=x-\frac{3}{2}-P\left(\frac{1}{x}\right)$したがって、元の極限は
                $\displaystyle\lim_{x\to \infty}\left\{x-\frac{3}{2}-P\left(\frac{1}{x}\right)-x\right\}=-\frac{3}{2}$

                \paragraph{問5}省略
                
                \paragraph{問6}
                $g(x)$は$[a,b]$で連続で、$(a,b)$で$n+1$階微分可能である。この時、$g(b)=0$がすぐわかる。また、少し計算すれば$g(a)=0$
                もわかる。すなわち$g(a)=g(b)$であるため、ロルの定理より
                \begin{equation*}
                    g'(c)=0\quad(a<c<b)
                \end{equation*}
                である。よって、
                \begin{align*}
                    g'(c)&=f'(c)-f'(c)+f''(c)(b-c)-f''(c)(b-c)+\frac{1}{2!}f'''(c)(b-c)^2+\cdots-\frac{1}{(n-1)!}f^{(n)}(c)(b-c)^{n-1}\\
                    &+\frac{1}{n!}f^{(n+1)}(c)(b-c)^{n}-K(n+1)(b-c)^{n}=0
                \end{align*}
                式を整理して
                \begin{equation*}
                    K=\frac{1}{(n+1)!}f^{(n+1)}(c)
                \end{equation*}
                これを$K$の定義に代入して整理すれば、式\eqref{eq:テイラーの定理}である。$\square$
            \clearpage

        \clearpage
        \subsection{積分 基本問題解答}
        \clearpage
        \subsection{無限級数 基本問題解答}
        \clearpage
        \subsection{演習問題解答}
        \clearpage
        \color{black}
        \section{付録}
            \subsection{人名}
                数学を学ぶ上で様々な人の名前を冠した定理や公式が登場する。このノートではカタカナで書いてきたが、実際にほかの書籍を読む際などは英名で書かれる場合が多い。そこで
                以下に対応の表を載せる。「主な定理等」にはその人が発見・発明した解析学関係のことを述べる。
                
                \begin{table}[h]\hspace{-2.5cm}
                    \begin{tabular}{|c|c|c|l|}\hline
                        名前 & 名前(英)& 主な定理等 & メモ\\\hline\hline
                        オイラー & Euler & オイラーの公式など & \begin{tabular}{l}18世紀最大の数学者であり、その業績は解析学に留まらない。\\オイラーの名を冠する定理も多い。\end{tabular}\\\hline
                        ニュートン & Newton & 流率法 & \begin{tabular}{l}古典力学の創始者。\\リンゴの落下から万有引力を思いついた逸話は非常に有名。\end{tabular}\\\hline
                        ライプニッツ & Leibniz & 微積分法の発明 & \begin{tabular}{l}ニュートンとは独立に微積分を発明させた。\\また、記号$\int\frac{d}{dx}$などはライプニッツによるものである。\end{tabular} \\\hline
                        コーシー & Cauchy & コーシーの積分定理 & \begin{tabular}{l}収束について厳密な定義を考え、のちの$\varepsilon-\delta$論法の原型となった。\\複素解析の業績がとくに有名。\end{tabular} \\\hline
                        ラグランジュ &  Lagrange & - & \begin{tabular}{l}解析力学を構築し、ニュートン力学を発展させた。\\ラグランジュの未定乗数法などでもその名前を見かける。\end{tabular}\\\hline
                        リーマン & Riemann & リーマン幾何学/リーマン積分 & \begin{tabular}{l}複素解析のパイオニア。彼が予想したリーマン予想は数学の重要な\\未解決問題の一つであり、ミレニアム懸賞問題のうちの一つである。\end{tabular}\\\hline
                        ガウス & Gauss & - & \begin{tabular}{l}オイラーと双璧をなす19世紀最大の数学者。数論の分野での功績\\が大きい。彼の名前の付いた法則・記号・単位も多い。\end{tabular}\\\hline
                        ウォリス & Wallis & ウォリス積分/ウォリスの公式 & \hspace{2mm}無限大の記号$\infty$を導入した。\\\hline
                        テイラー & Taylor & テイラーの定理/テイラー展開 & \hspace{2mm}特になし\\\hline
                        マクローリン & Maclaurin & マクローリン展開 & \hspace{2mm}特になし\\\hline
                        ベルヌーイ & Bernoulli & - & \begin{tabular}{l}ヨーロッパの数学者一族の総称。ベルヌーイ数、ベルヌーイの定理\\などベルヌーイの名前が付いたものも多い。\end{tabular}\\\hline
                        ロピタル & l'Hôpital & - & \begin{tabular}{l}初めて微分積分の書籍を出したことで有名。\\ちなみにロピタルの定理はロピタル自身が発見したのではなく\\ヨハン・ベルヌーイによるものである。\end{tabular}\\\hline
                        フーリエ & Fourier & フーリエ級数/フーリエ変換 & \begin{tabular}{l}関数を三角関数の級数で表すというアイデアから始まったフーリエ\\解析は、解析学の厳密化に貢献しただけでなく、物理や工学の幅広い\\部分で応用されている。また、ナポレオンのエジプト遠征に同行した\\ことでも有名。\end{tabular}\\\hline
                        デデキント &  Dedekind & - & \begin{tabular}{l}デデキントカット(切断)という方法で実数を定義。これは解析概論\\でも用いられている手法である。\end{tabular}\\\hline
                        {\footnotesize ワイエルシュトラス} & Weierstrass & - & \hspace{2mm}ワイエルシュトラスの$W$関数なども有名。\\\hline
                        ルベーク & Lebesgue & ルベーク積分 & ここら辺の話は集合論とも絡み合って非常に難しい。\\\hline
                    \end{tabular}                    
                \end{table}
            \clearpage
            \subsection{文字・記号とその名称}
                微積分に限らず数学ではよくギリシャ文字や様々な演算子を用いたりする。しかしそれらの文字をどう読むのかまで書いてあるものは少ない。そこで以下に対応表のつけた。
                \begin{table}[h]
                    \caption*{ギリシャ文字}
                    \centering
                    \begin{tabular}{|c|c|l||c|c|l|}\hline
                        大文字 & 小文字 & \multicolumn{1}{c||}{読み方} & 大文字 & 小文字 & \multicolumn{1}{c|}{読み方} \\\hline\hline
                        A & $\alpha$ & アルファ & N & $\nu$ & ニュー\\\hline
                        B & $\beta$ & ベータ & $\Xi$ & $\xi$ & クシー(クサイ)\\\hline
                        $\Gamma$ & $\gamma$ & ガンマ & O & $o$ & オミクロン \\\hline
                        $\Delta$ & $\delta$ & デルタ & $\Pi$ & $\pi$ & パイ\\\hline
                        E & $\varepsilon$ & イプシロン & P & $\rho$ & ロー \\\hline
                        Z & $\zeta$ & ゼータ & $\Sigma$ & $\sigma,\varsigma$ & シグマ\\\hline
                        H & $\eta$ & イータ & T & $\tau$ & タウ\\\hline
                        $\Theta$ & $\theta,\vartheta$ & シータ & $\Upsilon$ & $\upsilon$ & ウプシロン\\\hline
                        I & $\iota$ & イオタ & $\Phi$ & $\phi,\varphi$ & ファイ\\\hline
                        K & $\kappa$ & カッパ & X & $\chi$ & カイ\\\hline
                        $\Lambda$ & $\lambda$ & ラムダ & $\Psi$ & $\psi$ & プサイ\\\hline
                        M & $\mu$ & ミュー & $\Omega$ & $\omega$ & オメガ \\\hline
                    \end{tabular}
                    
                    \caption*{数学の記号}
                    \centering
                    \begin{tabular}{|c|c|l|}\hline
                        記号 & 読み方 & \multicolumn{1}{c|}{意味}\\\hline
                        $!$ & 階乗 & ある正の整数から$1$までの整数の積\\\hline
                        $!!$ & 二重階乗 & 階乗の一個飛ばし版。ex:\hspace{1mm}$5!!=5\times3\times1$\\\hline
                        $\int$ & インテグラル & 積分 \\\hline
                        $\partial$ & ラウンドディー/デル/パーシャル & 偏微分で用いられる。\\\hline
                        $f^{-1}(x)$ & エフ・インバース・エックス & $f(x)$の逆関数\\\hline
                        $\lim$ & リミット & 極限の操作。通常$\displaystyle\lim_{x\to a}$のように用いる。\\\hline
                        $\nabla$ & ナブラ & 微分作用素の一つ。$\nabla=\left(\frac{\partial}{\partial x},\frac{\partial}{\partial y}\right)$\\\hline
                        $\mathrm{div}$ & ダイバージェンス & ベクトル場に用いる演算子。ベクトル場の発散(divergence)を表す。\\\hline
                        $\mathrm{rot}$ & ローテーション & ベクトル場に用いる演算子。ベクトル場の回転(rotation)を表す。\\\hline
                        $\mathrm{grad}$ & グラディエント & スカラー場に用いる演算子。スカラー場の勾配(gradient)を表す。\\\hline
                        $\varDelta$ & ラプラス作用素/ラプラシアン & 微分作用素の一つ。ナブラ同士の内積として与えられる。\\\hline
                    \end{tabular}
                \end{table}
        \clearpage

        \thispagestyle{fancy}
        \fancyhead{}
        \fancyfoot{\centerline{\hyperref[目次]{目次に戻る}}}

        \vspace{\stretch{1}}
        \begin{figure}
            \centering
            \includegraphics[scale=0.5]{img/QuuNote/icon.png}
        \end{figure}    
        \vspace{\stretch{1}}

        

\end{document}