\documentclass[a4j,dvipdfmx]{jsarticle}

\usepackage[dvipdfmx]{graphicx}
\usepackage{amsmath,amssymb}
\usepackage{siunitx}
\usepackage{ascmac}
\usepackage[subrefformat=parens]{subcaption}
\usepackage{fancyhdr}
\usepackage{otf}
\usepackage[dvipdfmx]{hyperref}
\usepackage{pxjahyper}
\usepackage{okumacro}

\pagestyle{headings}

% \renewcommand{\thesubsection}{\arabic{subsection}}
\renewcommand{\headrulewidth}{1pt}
\renewcommand{\Re}{\operatorname{Re}}
\renewcommand{\Im}{\operatorname{Im}}

\newcounter{basic_quastion}\setcounter{basic_quastion}{1}
\newcommand{\basicquestion}{{\large 基本問題\hspace{1mm}\huge\fbox{\textbf{\arabic{basic_quastion}}}\addtocounter{basic_quastion}{1}}\thispagestyle{fancy}\lhead{$\Sigma$基本問題}\rhead{\thepage}\cfoot{}\quad}


\title{Quuノート ー微分積分\ajRoman{1}ー}
\date{最終更新 2023/12/01}
\author{責任者 Quu}

\begin{document}
    \maketitle
    \thispagestyle{empty}
    \centerline{\textbf{概要}}
    \noindent
    微分積分学入門についてのノート。\\
    主に、一変数の極限、一変数の微分・積分、実数の無限級数について扱う。
    
    \clearpage
    \tableofcontents
    \clearpage
    
    \part{前提知識}
    \vspace{\stretch{1}}
    \begin{screen}
        微分積分を学ぶうえで前提となる知識をまとめた。微分積分は主に関数の微分・積分について扱うわけだから、ある程度の関数の扱い方も知っておく必要がある。
        そのほか数の種類についてや閉区間・開区間、極限についてもまとめてある。極限は微分積分を学ぶ際にいたるところに出てきて、陰から支える縁の下の力持ち的な役割を持つ。極限は一見すると
        代入と同じように見えるが、実は違う。極限は代入だと都合が悪い時にありがたみが実感できる。極限に関連して、無限という概念も登場する。
    \end{screen}
    \clearpage
    \section{様々な`数'と数直線}
        \subsection{数の種類}
            数学を勉強するうえで、様々な数が登場する。まず一番初めに思いつくのが$1,2,3...$といった\textbf{自然数}である。
            次の自然数に$0$と負の符号をつけたものを加えた\textbf{整数}が考えられる。整数同士で足し算、引き算、掛け算を行っても
            その値は整数である。このことを\textbf{和、差、積について閉じている}という。これは$a,b\in \mathbb{Z}$となる任意の$a,b$
            について
            \begin{equation}
                a+b , a-b , a\times b \in \mathbb{Z}
            \end{equation}
            が成り立つことを意味している。

            一方割り算は整数の中に閉じていない。\footnote{例えば$1\div 2$など。}しかし、$0.5,3.14$などの\textbf{有理数}まで数を拡張すれば、その中に商は閉じている。
            つまり、$a,b \in \mathbb{Q}$となる任意の$a,b$について
            \begin{equation}
                \frac{a}{b} \in \mathbb{Q}
            \end{equation}
            となる。よって、数を有理数まで拡張すれば\textbf{四則について閉じている}ことがわかる。

            さらに数の拡張を考えよう。たとえば$x^2-2=0$を満たす$x$について考えてみるには、数を\textbf{無理数}まで拡張しなければならない。
            一般の二次方程式の解も
            \begin{equation}
                x = \frac{-b\pm \sqrt{b^2-4ac}}{2a}
            \end{equation}
            と有理数だけでは表現できないことがわかる。無理数には$\pi,e\footnote{自然対数の底またはネイピア数と呼ばれる。具体的な値は$e=2.71...$}$などの\textbf{超越数}もふくむ。

            私たちの生活の中では有理数と無理数をあわせた\textbf{実数}があれば十分事足りるが、数学の世界ではそうもいかない。先ほどの二次方程式についてより深く調べてみると、
            解を持たない条件(根号の中身が負)があることがすぐに分かる。例えば、$x^2+1 = 0$は$x^2 = -1$と変形できるが、二乗して負になるような数は実数のうちには存在しない。
            よってこの方程式は\textbf{解なし}となる。がしかし、ここで
            \begin{equation}
                i = \sqrt{-1}
            \end{equation}
            となる`数'を定義してあげると、方程式は$x=\pm i$となり、$(実数)+i$を含めた範囲に解をもつことがわかる。
            この数は、今までの実数とは異なる数であり、実数との和,差は直接計算できない。一般に実数$a,b$と$i$を用いて
            \begin{equation}
                z = a + bi
            \end{equation}
            として表した$z$を\textbf{複素数}といい、$i$を虚数単位という。さらに$a$を$z$の実部、$b$を$z$の虚部といい、それぞれ
            $a=\Re z,b=\Im z$と表す。

            先ほど、二次方程式の解を有理数だけでは表現できないといったが、実は無理数を含めてもできない。この複素数を含めることで初めてすべて表現できるようになるのだ。\footnote{もちろんこれで数の拡張が終わるわけではない。しかし微分積分を学ぶうち間は複素数まで拡張すれば事足りる。}
        \subsection{数直線}
            では、数の大小関係をわかりやすくするためにはどうすればよいだろうか。視覚的にわかりやすくするためには数直線を用いればよい。
            \begin{figure}[h]
                \centering
                \includegraphics[keepaspectratio,scale=0.5]{img/QuuNote/NumLine_1.png}
                \caption{数直線}
            \end{figure}
            
            上の例では、整数、有理数、無理数の一部を記載している。当然書いてある数以外も数直線の中には含まれている。
            むしろ、数の点の集まりとして数直線を捉える方がイメージがわきやすいかもしれない。
            ここで注意しなければならないのは、この数直線上に複素数$(\Im z\neq 0)$は含まれないといけないということである。
            数直線は\underline{実数を表す}直線なので、実数より(集合的に)大きい複素数のすべては含むことができないのである。

            では実数も含めた複素数はどう表せばよいのか。答えは単純で実数の軸\footnote{これを実軸と呼ぶ。}とは別の軸\footnote{虚軸という。}を
            加えればよい。つまり複素数は平`面'上で表せられるのである。\footnote{この平面を複素平面という。いつものy-xグラフとは見た目は同じだが感覚が違うので注意。}  
            \newpage
        \subsection{開区間・閉区間}
            実数が数直線上の一点で表せることはすでに前項で述べた。では点に続いて次は区間について考えていこう。

            区間は大きく二つある。それらはそれぞれ\textbf{開区間}、\textbf{閉区間}と呼ばれる。これらの違いは端点を含むかどうかで、
            逆に言えば端以外は同じである。例えば、$1<x<2,1\leq x\leq2$について前者は端点$x=1,2$を含まず、後者は端点を含むのである。
            端点を含まない場合が開区間、端点を含む場合が閉区間である。

            閉区間、開区間を数直線上で表すにはどうすればよいだろうか。これも数直線と同様に区間の端から端まで線を引けばよい。注意しないといけないのが
            端点で、区間が開区間か閉区間かによって端点を書き分けないといけない。開区間のときは$\circ$、閉区間のときは$\bullet$と書けばよい。

            \begin{figure}[h]
                \centering
                \includegraphics[keepaspectratio,scale=0.7]{img/QuuNote/IntervalLine.png}
                \caption{開区間と閉区間}
            \end{figure}
            例えば上図の例をみてみると、$A,B,C$の三つの区間がある。それぞれ$-4\leq x\leq -1,0<x<3,-1\leq x<1$となる。
            今までは不等号を用いて区間を表現してきたが\footnote{厳密に言えば区間は集合なので、不等号を用いて区間を表現するという言い方は適切ではない。}、もっと簡潔に$(\hspace{1mm}),[\hspace{1mm}]$を用いて表現する方法もある。
            この表現方法を使えば、$A,B,C$はそれぞれ$[-4,-1],(0,3),[-1,1]$と表せる。$(\hspace{1mm})$が等号を含まない、$[\hspace{1mm}]$が
            等号を含む、というわけである。

            では、値が無限に続く(例えば実数全体など)場合はどう表現すればよいのか。この場合は無限大の記号$\infty$を用いて$(-\infty,\infty)$などと表せばよい。\footnote{この方法を使えば、a以上の実数などの場合でも$[a,\infty)$と表せばよいことがわかる。}
            \clearpage
            
        \noindent
        \basicquestion 以下問に答えよ。

        \paragraph{問1}以下の主張のうち正しいものには〇を、間違っているものには×をつけよ。
            \begin{enumerate}
                \item $\sqrt{9}$は無理数である。
                \item 有理数は全て分数の形で表せる。
                \item $i$は複素数である。
                \item 有理数の集合は$\mathbb{Q}$として表し、無理数の集合は$\mathbb{N}$で表す。
                \item 自然数全体の集合(区間)は$(0,\infty]$である。
            \end{enumerate}
        \paragraph{問2}以下の区間について、数直線上に示せ。もし数直線上に記されていない数字が出てくる場合はそれも記載せよ。
            \begin{figure}[h]
                \centering
                \includegraphics[keepaspectratio,scale=0.6]{img/QuuNote/NumLine_1.png}
                \caption{数直線}
            \end{figure}

            $1.\quad [2,3]$\hspace{3mm}
            $2.\quad (3,5)$\hspace{3mm}
            $3.\quad [-5,\pi]$\hspace{3mm}
            $4.\quad (-2,0.5]$\hspace{3mm}
            $5.\quad [-1,0)$\hspace{3mm}
            $6.\quad (-\infty,0)$\hspace{3mm}
            $7.\quad [0,\infty)$\hspace{3mm}
        \clearpage
        
        \section{関数の性質}
            \subsection{偶関数・奇関数}
                一般の関数$f(x)$について、$f(-x)=f(x)$を満たすものを\textbf{偶関数}、$f(-x)=-f(x)$を満たすものを\textbf{奇関数}という。
                もちろん全ての関数が偶関数・奇関数のどちらかであるというわけではない。しかし、全ての関数は偶関数と奇関数の和で表せられること
                が知られている。\\

                関数が偶関数・奇関数である場合のグラフはどうなるだろうか。まずは偶関数から考えてみると、定義より$x>0$と$x<0$の点において$f$は
                同じ値を取るわけであるから、グラフはy軸に対して対象になるはずである。つぎに奇関数について考えてみよう。これも定義より$x>0$と$x<0$の
                点において、$f$はx軸に対してそれぞれ対象に点を取るはずである。つまり、グラフは原点に対して点対象になるはずである。\\

                偶関数の例となる関数は、$x^2,\cos x,a(\text{定数関数})$などがあげられる。奇関数の例となる関数は、$x,\sin x,\tan x$などがあげられる。
                各自でグラフソフトなどでグラフを見てみるとよい。
            \clearpage
            \subsection{べき関数}
                関数のなかでもっともなじみやすいのが、$f(x)=x^n$であろう。例えば$x$は一次関数、$x^2$は二次関数と呼ばれる。
                別に$n$は自然数に限らなくてもよい。$x^{\frac{1}{2}}=\sqrt{x}$は指数が自然数ではないが、これもべき関数の一つである。
                $n$が自然数のうちは、関数の定義域について特別意識をする必要はない。しかし$n=\frac{1}{2}$などのように指数が有理数であったり、
                $n=-1$のように指数が負の値を取る場合には定義域に十分注意する必要がある。このように、一般に$f(x)=x^n$で表される関数を\textbf{べき関数}という。

                次に関数のグラフについて、グラフ描画ソフトを用いて数式を入力すると以下のようになる。
                \begin{figure}[h]
                    \centering
                    \includegraphics[keepaspectratio,scale=0.5]{img/QuuNote/PowerFuncGraph.png}
                    \caption{べき関数グラフ}
                \end{figure}

                図からも$\sqrt{x}$が$x<0$で定義されないことがわかる。また、$x^2$と$\sqrt{x}$は$y=x$を軸にして線対象になっており、
                $x^2$と$\sqrt{x}$は互いに\textbf{逆関数}であることがわかる。
            \clearpage
            \subsection{三角関数}
                \textbf{三角関数}は三角比を一般角に拡張した関数である。定義からわかるように周期関数であり、周期は$2\pi$である。
                三角比の定義自体を忘れた人はいないだろうが一応説明しておく。
                \begin{figure}[h]
                    \centering
                    \includegraphics[keepaspectratio,scale=0.5]{img/QuuNote/triangleFunc.png}
                    \caption{三角比}
                \end{figure}

                上図\footnote{aの辺のことを対辺、bの辺のことを隣辺、cの辺のことを斜辺という。}において
                \begin{align}
                    \sin \theta &= \frac{a}{c}\\
                    \cos \theta &= \frac{b}{c}\\
                    \tan \theta &= \frac{a}{b}
                \end{align}
                また、定義より$\displaystyle\tan \theta = \frac{\sin \theta}{\cos \theta}$が成り立つ。

                三角関数には様々な公式がある。\footnote{公式集を眺めるとやたらと二乗がついていることがわかる。つまり三角関数は\underline{二乗に強い}のである。この性質は積分を解く際に重要である。}
                しかしそれらは単位円を書けばすぐに導けるので、一部を除いて割愛する。
                また、三角関数の角度の合成についても、全て\textbf{加法定理}より導けるのでここでは加法定理のみ紹介する。

                \begin{align}
                    \sin^2 x + \cos ^2 x &= 1 &\quad 1 + \tan^2 x &= \frac{1}{\cos^2 x}\\
                    \sin\left(\frac{\pi}{2}-x\right) &= \cos x &\quad \cos\left(\frac{\pi}{2}-x\right) &= \sin x\\
                    \tan\left(\frac{\pi}{2}-x\right) &= \frac{1}{\tan x} &&
                \end{align}
                \begin{align}
                    \sin(\alpha\pm\beta) &= \sin\alpha\cos\beta \pm \cos\alpha\sin\beta\\
                    \cos(\alpha\pm\beta) &= \cos\alpha\cos\beta \mp \sin\alpha\sin\beta\\
                    \tan(\alpha\pm\beta) &= \frac{\tan\alpha\pm\tan\beta}{1\mp\tan\alpha\tan\beta}
                \end{align}
            \clearpage
            \subsection{指数・対数関数}
                `指数的に増加する'という言葉をよく耳にする。これは、なにか爆発的な増加の様子を示している表現である。このように、\textbf{指数関数}は$x$の値が少し変わるだけで値が
                大きく増加・減少する関数である。その具体的な表式は$a^x$と表される。指数の部分が変数になっているのである。

                指数関数は、$a$の値によって性質が少し異なる。$0<a<1$の場合には単調減少関数となり、$1<a$の場合には単調増加関数になる。
                このとき$a$が負の値の場合は定義しない。\footnote{例として$(-2)^x$のグラフを書いてその理由を考えてみるといい。}\footnote{実際は定義することができるがその際には複素関数の知識が必要。なので今回は扱わない。}

                以下、指数法則について述べる。$a,b>0\quad x,y\in \mathbb{R}$とすると、
                \begin{align}
                    a^x\cdot a^y&=a^{x+y}\\
                    \frac{a^x}{a^y} &= a^{x-y}\\
                    (a^x)^y &= a^{xy}\\
                    (ab)^{x} &= a^x\cdot b^x
                \end{align}

                指数関数のグラフは、以下のようになる。
                \begin{figure}[h]
                    \centering
                    \includegraphics[keepaspectratio,scale=0.3]{img/QuuNote/ExpFuncGraph.png}
                    \caption{指数関数のグラフ}
                \end{figure}

                グラフから、$a$が1より大きくても小さくても$x=0$で$y=1$を取ることがわかる。

                なお、底が$e$の場合の指数関数は$e^x=\exp{x}$と書くこともある。\\

                では次に、指数関数の逆関数を考えてみよう。指数関数の逆関数は与えられた値に対して、
                底を何回掛けたらその値になるかの回数を表す関数である。つまり、指数関数の底ごとに
                逆関数が存在する。文章で見てもわかりずらいので数式で以下示す。

                \begin{equation}
                    f(x)=a^x \leftrightarrow x = f^{-1}(y) = \log_a{y}
                \end{equation}
                指数関数の逆関数は底が何かを示さないといけないので、逆関数$\log$に下付き文字で書く。
                しかし、底が$e$だった場合は省略して$\log y$と書いてもよい。この関数を\textbf{自然対数}という。\footnote{自然対数は$\ln x$と書くこともある。natural logarithmのことである。}
                底が$e$じゃない場合は単に対数と呼ぶ。\footnote{底が10の場合は常用対数という。}

                以下、対数の性質を述べる。必要ない限り底は省略して記載する。指数法則と見比べると理解が深まる。

                \begin{align}
                    \log(xy)&=\log x+\log y\\
                    \log\left(\frac{x}{y}\right)&=\log x - \log y\\
                    \log(a^b) &= b\log a \\
                    \frac{\log_c b}{\log_c a}&=\log_a b\qquad(\text{\textbf{底の変換公式}})
                \end{align}

                また、対数の定義より
                \begin{equation}
                    \log 1 = 0\quad \log_a a = 1
                \end{equation}
                が成り立つ。対数関数$\log x$の引数$x$のことを真数と呼び、これは$x>0$である。\footnote{真数が正であるという条件のことを真数条件という。}

                対数関数のグラフは以下のようになる。

                \begin{figure}[h]
                    \centering
                    \includegraphics[keepaspectratio,scale=0.5]{img/QuuNote/LogFuncGraph.png}
                    \caption{対数関数}
                \end{figure}
                
                グラフを見ればわかるように、底が1より大きいか小さいかで単調増加・減少かが変わる。

                対数関数は爆発的に増加・減少する指数関数とは対照的に、値の変化が($x<1$を除いて)緩やかである。
                そのため、値がとても大きい値でも対数を取ることで値のスケールを小さくすることができる。
                また、対数を取ることで\underline{掛け算を足し算にできる}。この性質は非常に重要である。
            \clearpage
            \subsection{逆三角関数}
                指数関数の逆関数である対数関数を考えたのと同じように、三角関数の逆関数も考えてみよう。
                三角関数は与えられた角度に対応するそれぞれの三角比を返す関数である。では三角関数の逆関数は
                与えられた三角比に対応する`角度'を返す関数であることがすぐに分かる。これらを次のように書くことにする。
                \begin{align}
                    \arcsin x &= \sin^{-1} x \\
                    \arccos x &= \cos^{-1} x \\
                    \arctan x &= \tan^{-1} x
                \end{align}
                左辺にちなんで左からそれぞれ「アークサイン」,「アークコサイン」,「アークタンジェント」と読む。これらをまとめて
                \textbf{逆三角関数}という。表記に左辺を用いるか右辺を用いるかは個人の好みによる。\footnote{だからといって、$\frac{1}{\sin x}$を$\sin^{-1}x$と書くことはまずない。}

                三角関数が周期関数であるため、逆三角関数は多価関数であることは容易に想像できる。
                逆三角関数を一価関数にするため、値域をそれぞれ$[-\frac{\pi}{2},\frac{\pi}{2}],[0,\pi],[-\frac{\pi}{2},\frac{\pi}{2}]$
                に制限して用いることがある。このことを\ruby{主枝}{しゅし}を取るという。またこの制限した値域を主枝という。
                主枝以外の値域を分枝と呼ぶ。

                逆三角関数が主枝を取っていることを明示するために
                \begin{align}
                    {\rm Arcsin} x &= {\rm Sin^{-1}} x \\
                    {\rm Arccos} x &= {\rm Cos^{-1}} x \\
                    {\rm Arctan} x &= {\rm Tan^{-1}} x
                \end{align}
                のように、先頭を大文字で書くこともある。しかし今回はこの記法は採用しない。

                逆三角関数のグラフは以下のようになる。
                \begin{figure}[h]
                    \begin{minipage}{5cm}
                        \centering
                        \includegraphics[keepaspectratio,scale=0.3]{img/QuuNote/ArcsinFuncGraph.png}
                        \caption{$\arcsin x$のグラフ}
                    \end{minipage}
                    \begin{minipage}{5cm}
                        \centering
                        \includegraphics[keepaspectratio,scale=0.3]{img/QuuNote/ArccosFuncGraph_ver2.png}
                        \caption{$\arccos x$のグラフ}
                    \end{minipage}
                    \begin{minipage}{5cm}
                        \centering
                        \includegraphics[keepaspectratio,scale=0.3]{img/QuuNote/ArctanFuncGraph.png}
                        \caption{$\arctan x$のグラフ}
                    \end{minipage}
                \end{figure}
                
                もちろん主枝を取らない場合は、それぞれと同じグラフが上や下につながっていく。
            \clearpage
            \subsection{双曲線関数}
                いきなりだが、次のように関数を定義する。$e$はネイピア数である。
                \begin{align}
                    \sinh x &= \frac{e^x - e^{-x}}{2}\\
                    \cosh x &= \frac{e^x + e^{-x}}{2}\\
                    \tanh x &= \frac{\sinh x}{\cosh x} = \frac{e^x - e^{-x}}{e^x + e^{-x}}
                \end{align}
                これらは\textbf{双曲線関数}と呼ばれる。読み方はそれぞれ「ハイパボリックサイン」、「ハイパボリックコサイン」、
                「ハイパボリックタンジェント」である。ただこれだと長ったらしいので「シンチ」、「コッシュ」、「タンチ」と呼ぶ
                場合もある。

                見た目が三角関数と酷使しているが、実は性質も似たものを持つ。例えば、$\cosh^2 x - \sinh^2 x = 1$など。
                また、加法定理も符号は若干異なるがほとんど同じ形をしている。

                さらに、オイラーの公式$e^{ix}=\cos x + i\sin x$\footnote{この公式自体はだいぶ後になって解説する。}を用いれば、
                $\sin ix = i\sinh x,\cos ix=\cosh x$が導ける。\footnote{むしろこの性質が成り立つように双曲線関数を定義するといったほうが正しいかもしれない。}\\

                双曲線関数のグラフは以下のようになる。
                \begin{figure}[h]
                    \begin{minipage}{5cm}
                        \centering
                        \includegraphics[keepaspectratio,scale=0.3]{img/QuuNote/SinhFuncGraph.png}
                        \caption{$\sinh x$のグラフ}
                    \end{minipage}
                    \begin{minipage}{5cm}
                        \centering
                        \includegraphics[keepaspectratio,scale=0.3]{img/QuuNote/CoshFuncGraph.png}
                        \caption{$\cosh x$のグラフ}
                    \end{minipage}
                    \begin{minipage}{5cm}
                        \centering
                        \includegraphics[keepaspectratio,scale=0.3]{img/QuuNote/TanhFuncGraph.png}
                        \caption{$\tanh x$のグラフ}
                    \end{minipage}
                \end{figure}

                $\cosh x$のグラフはカテナリーと呼ばれる。電柱などの垂れた線はこれにあたる。
            \clearpage
            \subsection{初等関数}

                

    \clearpage
    \part{微分}

    \clearpage
    \part{積分}

    \clearpage
    \part{無限級数}

\end{document}