\documentclass[a4j,dvipdfmx]{jsarticle}
\usepackage{amsmath,amssymb}
\usepackage{siunitx}

\renewcommand{\thesubsection}{\roman{subsection}}

\begin{document}
\section*{偏微分}
\subsection{二変数の関数}
いままでは、一つの変数$x$の関数$f(x)$を考えてきた。しかし、一つの変数では記述できない現象も多くある。例えば、理想気体の状態方程式を考えてみる。
\begin{equation*}
    pV=RT\quad(R:\text{気体定数})
\end{equation*}
この式中の$p$(圧力)がどのような値を取るのかは、$T$(温度)と$V$(体積)の両方を指定しないと決まらない。

一般に、二つの変数$x$と$y$があり、$x$と$y$の各々の値の組に対して$z$の値が決まるとき、$z$を$x$と$y$の関数といい、
\begin{equation*}
    z=f(x,y)
\end{equation*}
と表す。このとき、$x$と$y$を\textbf{独立変数}、$z$を\textbf{従属変数}という。
\subsubsection*{例題1}
$f(x,y)=x^2+2xy+y^2$であるとき、$x=1,y=-1$のときの$f(x,y)$を求めよ。

\subsubsection*{解答}
$f(x,y)=(x+y)^2=(1-1)^2=0$
\\\\
二変数の関数における極限について考えてみよう。点$P(x,y)$が点$A(a,b)$と一致することなく点Aに近づくとする。この時、その近づき方によらず、関数$f(x,y)$の値が同じ一つの値$c$
に近づくならば、$f(x,y)$には\textbf{極限}が存在して、その\textbf{極限値}は$c$であるといい、
\begin{equation*}
    f(x,y)\to c\quad(x\to a,y\to b)\qquad\lim_{x\to a,y\to b}f(x,y)=c\qquad\lim_{(x,y)\to(a,b)}f(x,y)=c
\end{equation*}
などと表す。関数$f(x,y)$が$c$に\textbf{収束する}ともいう。

この時、二つほど注意することがある。

(1)極限の定義において、点Pと点Aが一致することは除外している。一般に、点Aが関数$f(x,y)$の定義域に含まれているとは限らない。

(2)点Pが点Aに近づく仕方によって、$f(x,y)$が近づく値が異なるときには、極限は存在しない。


試しに以下の例題を解きながら考えてみよう。
\subsubsection*{例題2}
関数$\displaystyle f(x,y)=\frac{x^2}{x^2+y^2}$において、$x\to 0,y\to 0$の極限を調べ、極限値が存在するかどうか答えよ。


\subsubsection*{解答}
この関数の定義域は、全平面から原点Oを除外して得られる領域である。次の二つの路で、点$P(x,y)$が原点Oに近づくとしよう。

(a)点Pがx軸に沿って近づく場合。
(b)点Pがy軸に沿って近づく場合。\\
(a)の場合は$f(x,0)=1$より、$\displaystyle \lim_{x\to 0}f(x,0)=1$\\
(b)の場合は$f(0,y)=0$より、$\displaystyle \lim_{y\to 0}f(0,y)=0$\\
したがって、この関数は原点への近づき方によって異なる値を取り、極限値$\displaystyle\lim_{x\to 0,y\to 0}f(x,y)$は存在しない。


つづいて、関数の連続について考えてみよう。点$A(a,b)$の近くで定義されている関数$z=f(x,y)$について、次の三つの条件が成り立つ場合、$z=f(x,y)$は点$A(a,b)$において\textbf{連続}であるという。
\begin{enumerate}
    \item $f(a,b)$が定義されている。
    \item $\displaystyle\lim_{(x,y)\to(a,b)}f(x,y)$が存在する。
    \item $\displaystyle\lim_{(x,y)\to(a,b)}f(x,y)=f(a,b)$
\end{enumerate}
例題を解きながら考えてみよう。
\subsubsection*{例題3}
次の関数の原点での連続性についてしらべよ。
\begin{align*}
    &(1)f(x,y)= \left \{
        \begin{array}{l}
        \displaystyle\frac{x^3+y^3}{x^2+y^2} \quad((x,y)\neq(0,0))\\
        0 \qquad\qquad((x,y)=(0,0))
        \end{array}
        \right.\\
    &(2)f(x,y)= \left \{
        \begin{array}{l}
        \displaystyle\frac{xy}{x^2+y^2} \quad((x,y)\neq(0,0))\\
        0 \qquad\qquad((x,y)=(0,0))
        \end{array}
        \right.\\
    &(3)f(x,y)= \left \{
        \begin{array}{l}
        xy \hspace{2mm}\quad\qquad((x,y)\neq(0,0))\\
        2  \hspace{2mm}\quad\qquad\hspace{2mm}((x,y)=(0,0))
        \end{array}
        \right.
\end{align*}

\subsubsection*{解答}
(1)は、$f(0,0)$が存在し、
\begin{equation*}
    \lim_{(x,y)\to(0,0)}f(x)=\lim_{(x,y)\to(0,0)}\frac{x^3+y^3}{x^2+y^2}=0=f(0,0)
\end{equation*}
であるため、この関数は原点で連続である。

(2)$f(0,0)$は定義されている。ここで、原点$y=mx(m\neq 0)$に沿って原点に近づけると、
\begin{equation*}
    \lim_{x\to 0}f(x,mx)=\lim_{x\to 0}\frac{x\cdot mx}{x^2+(mx)^2}=
    \lim_{x\to 0}\frac{mx^2}{(m^2+1)x^2}=\lim_{x\to 0}\frac{m}{m^2+1}\neq f(0,0)
\end{equation*}
であるため、この関数は原点で連続ではない。別解として、直交座標表示から極座標表示に変換して求める方法もある。この場合だと原点に近づけるために必要なパラメータが$r$、つまり
原点からの距離だけとなるのでわかりやすいかもしれない。

(3)$\displaystyle\lim_{(x,y)\to(0,0)}f(x,y)=0\neq f(0,0)$であるため、この間数は連続ではない。もし、$f(0,0)=0$と定義しなおせば、この新しい関数は原点で連続になる。このように、
不連続点で関数を定義しなおすことによって。連続にできるとき、その不連続点は除きうる不連続であるという。
\subsection{偏微分}
関数$z=f(x,y)$において、独立変数$x,y$は、おのおの独立な変数である。どちらかを一定にして、もう片方の変数の値を変えることができる。もちろんどちらも同時に変えることができる。

では、$y$を一定にして$x$を変動させることを考えてみる。この時$f(x,y)$は$x$の関数だから、この$x$の関数の導関数
\begin{equation}
    \frac{\partial f(x,y) }{\partial x}=\lim_{\Delta x\to 0}\frac{f(x+\Delta x,y)-f(x,y)}{\Delta x}
\end{equation}
が存在すれば、\textbf{偏微分可能}であるという。また、$\displaystyle\frac{\partial f}{\partial x}$を$f(x,y)$の$x$に関する\textbf{偏導関数}という。同様に、$x$を一定として$y$を一定として変えたときの導関数を$f(x,y)$の$y$に関する偏導関数という。
一つの変数を定数とみなし、他方の変数で微分するのが、``偏"の意味であり、決して``変"な微分ではない。

偏導関数を表す記法はいろいろある。
\begin{equation*}
    \frac{\partial f(x,y)}{\partial x},f_x,f_x(x,y),\frac{\partial z}{\partial x},\left. \frac{\partial f(x,y)}{\partial x}\right|_y
\end{equation*}
上記の導関数はすべて$x$の偏導関数である。また、最後の記法はどの変数が一定かを強調している記法で、熱力学で多用されるらしい。
なお、$f(x,y)$の偏導関数を求めることを\textbf{偏微分する}という。
\subsubsection*{例題4}
次の関数を偏微分せよ。
\begin{equation*}
    (1)f(x,y)2x^3+5xy+2y^2\quad(2)p(T,V)=\frac{RT}{V}\quad(R:\text{定数})
\end{equation*}
\subsubsection*{解答}
(1)$y$を一定にして$x$で微分すると、$f_x=6x^2+5y$。反対に$x$を一定にして$y$で微分すると、
$f_y=5x+4y$。

(2)\begin{align*}
    &\frac{\partial p}{\partial T}=\frac{R}{V}\\
    &\frac{\partial p}{\partial V}=-\frac{RT}{V^2}
\end{align*}

多くの変数$x,y,z,\cdots ,t$の関数$u=f(x,y,x,\cdots,t)$を\textbf{多変数関数}という。これらも二変数関数と同様に偏導関数を求めることができる。

例えば、$\displaystyle r(x,y,z)=\sqrt{x^2+y^2+z^2}$の偏導関数を求めるとすると。
\begin{equation*}
    r_x=\frac{\partial}{\partial x}\sqrt{x^2+y^2+z^2}=\frac{1}{2}\cdot 2x(x^2+y^2+z^2)^{-\frac{1}{2}}=\frac{x}{r}
\end{equation*}
同様にして、
\begin{equation*}
    r_y=\frac{y}{r}\quad r_z=\frac{z}{r}
\end{equation*}
\end{document}
