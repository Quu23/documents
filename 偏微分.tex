\documentclass[a4j,dvipdfmx]{jsarticle}
\usepackage{amsmath,amssymb}
\usepackage{siunitx}
\usepackage{ascmac}

\renewcommand{\thesubsection}{\roman{subsection}}

\begin{document}
\section*{$\int$偏微分$\int$}
\subsection{二変数の関数}
いままでは、一つの変数$x$の関数$f(x)$を考えてきた。しかし、一つの変数では記述できない現象も多くある。例えば、理想気体の状態方程式を考えてみる。
\begin{equation*}
    pV=RT\quad(R:\text{気体定数})
\end{equation*}
この式中の$p$(圧力)がどのような値を取るのかは、$T$(温度)と$V$(体積)の両方を指定しないと決まらない。

一般に、二つの変数$x$と$y$があり、$x$と$y$の各々の値の組に対して$z$の値が決まるとき、$z$を$x$と$y$の関数といい、
\begin{equation*}
    z=f(x,y)
\end{equation*}
と表す。このとき、$x$と$y$を\textbf{独立変数}、$z$を\textbf{従属変数}という。
\subsubsection*{例題1}
$f(x,y)=x^2+2xy+y^2$であるとき、$x=1,y=-1$のときの$f(x,y)$を求めよ。

\subsubsection*{解答}
$f(x,y)=(x+y)^2=(1-1)^2=0$
\\\\
二変数の関数における極限について考えてみよう。点$P(x,y)$が点$A(a,b)$と一致することなく点Aに近づくとする。この時、その近づき方によらず、関数$f(x,y)$の値が同じ一つの値$c$
に近づくならば、$f(x,y)$には\textbf{極限}が存在して、その\textbf{極限値}は$c$であるといい、
\begin{equation*}
    f(x,y)\to c\quad(x\to a,y\to b)\qquad\lim_{x\to a,y\to b}f(x,y)=c\qquad\lim_{(x,y)\to(a,b)}f(x,y)=c
\end{equation*}
などと表す。関数$f(x,y)$が$c$に\textbf{収束する}ともいう。

この時、二つほど注意することがある。

(1)極限の定義において、点Pと点Aが一致することは除外している。一般に、点Aが関数$f(x,y)$の定義域に含まれているとは限らない。

(2)点Pが点Aに近づく仕方によって、$f(x,y)$が近づく値が異なるときには、極限は存在しない。


試しに以下の例題を解きながら考えてみよう。
\subsubsection*{例題2}
関数$\displaystyle f(x,y)=\frac{x^2}{x^2+y^2}$において、$x\to 0,y\to 0$の極限を調べ、極限値が存在するかどうか答えよ。


\subsubsection*{解答}
この関数の定義域は、全平面から原点Oを除外して得られる領域である。次の二つの路で、点$P(x,y)$が原点Oに近づくとしよう。

(a)点Pがx軸に沿って近づく場合。
(b)点Pがy軸に沿って近づく場合。\\
(a)の場合は$f(x,0)=1$より、$\displaystyle \lim_{x\to 0}f(x,0)=1$\\
(b)の場合は$f(0,y)=0$より、$\displaystyle \lim_{y\to 0}f(0,y)=0$\\
したがって、この関数は原点への近づき方によって異なる値を取り、極限値$\displaystyle\lim_{x\to 0,y\to 0}f(x,y)$は存在しない。


つづいて、関数の連続について考えてみよう。点$A(a,b)$の近くで定義されている関数$z=f(x,y)$について、次の三つの条件が成り立つ場合、$z=f(x,y)$は点$A(a,b)$において\textbf{連続}であるという。
\begin{enumerate}
    \item $f(a,b)$が定義されている。
    \item $\displaystyle\lim_{(x,y)\to(a,b)}f(x,y)$が存在する。
    \item $\displaystyle\lim_{(x,y)\to(a,b)}f(x,y)=f(a,b)$
\end{enumerate}
例題を解きながら考えてみよう。
\subsubsection*{例題3}
次の関数の原点での連続性についてしらべよ。
\begin{align*}
    &(1)f(x,y)= \left \{
        \begin{array}{l}
        \displaystyle\frac{x^3+y^3}{x^2+y^2} \quad((x,y)\neq(0,0))\\
        0 \qquad\qquad((x,y)=(0,0))
        \end{array}
        \right.\\
    &(2)f(x,y)= \left \{
        \begin{array}{l}
        \displaystyle\frac{xy}{x^2+y^2} \quad((x,y)\neq(0,0))\\
        0 \qquad\qquad((x,y)=(0,0))
        \end{array}
        \right.\\
    &(3)f(x,y)= \left \{
        \begin{array}{l}
        xy \hspace{2mm}\quad\qquad((x,y)\neq(0,0))\\
        2  \hspace{2mm}\quad\qquad\hspace{2mm}((x,y)=(0,0))
        \end{array}
        \right.
\end{align*}

\subsubsection*{解答}
(1)は、$f(0,0)$が存在し、
\begin{equation*}
    \lim_{(x,y)\to(0,0)}f(x)=\lim_{(x,y)\to(0,0)}\frac{x^3+y^3}{x^2+y^2}=0=f(0,0)
\end{equation*}
であるため、この関数は原点で連続である。

(2)$f(0,0)$は定義されている。ここで、原点$y=mx(m\neq 0)$に沿って原点に近づけると、
\begin{equation*}
    \lim_{x\to 0}f(x,mx)=\lim_{x\to 0}\frac{x\cdot mx}{x^2+(mx)^2}=
    \lim_{x\to 0}\frac{mx^2}{(m^2+1)x^2}=\lim_{x\to 0}\frac{m}{m^2+1}\neq f(0,0)
\end{equation*}
であるため、この関数は原点で連続ではない。別解として、直交座標表示から極座標表示に変換して求める方法もある。この場合だと原点に近づけるために必要なパラメータが$r$、つまり
原点からの距離だけとなるのでわかりやすいかもしれない。

(3)$\displaystyle\lim_{(x,y)\to(0,0)}f(x,y)=0\neq f(0,0)$であるため、この間数は連続ではない。もし、$f(0,0)=0$と定義しなおせば、この新しい関数は原点で連続になる。このように、
不連続点で関数を定義しなおすことによって。連続にできるとき、その不連続点は除きうる不連続であるという。
\subsection{偏微分}
関数$z=f(x,y)$において、独立変数$x,y$は、おのおの独立な変数である。どちらかを一定にして、もう片方の変数の値を変えることができる。もちろんどちらも同時に変えることができる。

では、$y$を一定にして$x$を変動させることを考えてみる。この時$f(x,y)$は$x$の関数だから、この$x$の関数の導関数
\begin{equation}
    \frac{\partial f(x,y) }{\partial x}=\lim_{\Delta x\to 0}\frac{f(x+\Delta x,y)-f(x,y)}{\Delta x}
\end{equation}
が存在すれば、\textbf{偏微分可能}であるという。また、$\displaystyle\frac{\partial f}{\partial x}$を$f(x,y)$の$x$に関する\textbf{偏導関数}という。同様に、$x$を一定として$y$を一定として変えたときの導関数を$f(x,y)$の$y$に関する偏導関数という。
一つの変数を定数とみなし、他方の変数で微分するのが、``偏"の意味であり、決して``変"な微分ではない。

偏導関数を表す記法はいろいろある。
\begin{equation*}
    \frac{\partial f(x,y)}{\partial x},f_x,f_x(x,y),\frac{\partial z}{\partial x},\left. \frac{\partial f(x,y)}{\partial x}\right|_y
\end{equation*}
上記の導関数はすべて$x$の偏導関数である。また、最後の記法はどの変数が一定かを強調している記法で、熱力学で多用されるらしい。
なお、$f(x,y)$の偏導関数を求めることを\textbf{偏微分する}という。
\subsubsection*{例題4}
次の関数を偏微分せよ。
\begin{equation*}
    (1)f(x,y)2x^3+5xy+2y^2\quad(2)p(T,V)=\frac{RT}{V}\quad(R:\text{定数})
\end{equation*}
\subsubsection*{解答}
(1)$y$を一定にして$x$で微分すると、$f_x=6x^2+5y$。反対に$x$を一定にして$y$で微分すると、
$f_y=5x+4y$。

(2)\begin{align*}
    &\frac{\partial p}{\partial T}=\frac{R}{V}\\
    &\frac{\partial p}{\partial V}=-\frac{RT}{V^2}
\end{align*}

多くの変数$x,y,z,\cdots ,t$の関数$u=f(x,y,x,\cdots,t)$を\textbf{多変数関数}という。これらも二変数関数と同様に偏導関数を求めることができる。

例えば、$\displaystyle r(x,y,z)=\sqrt{x^2+y^2+z^2}$の偏導関数を求めるとすると。
\begin{equation*}
    r_x=\frac{\partial}{\partial x}\sqrt{x^2+y^2+z^2}=\frac{1}{2}\cdot 2x(x^2+y^2+z^2)^{-\frac{1}{2}}=\frac{x}{r}
\end{equation*}
同様にして、
\begin{equation*}
    r_y=\frac{y}{r}\quad r_z=\frac{z}{r}
\end{equation*}

偏導関数$f_x,f_y$は$x$と$y$の関数であり、さらにそれらの偏導関数を考えることもできる。こうして得られる関数を元の関数の\textbf{二階偏導関数}という。

偏導関数$f_{xy}$は最初に$x$で、つぎの$y$で偏微分したものである。一方$f_{yx}$はその逆の順に偏微分したものである。
一般に、\underbar{$f_{xy}$と$f_{yx}$が連続ならば、}
\begin{equation}
    f_{xy}=f_{yx}
\end{equation}
が成り立つ。すなわち、偏微分の順序を好感しても偏導関数は変わらない。(2)式の拡張として、公開偏導関数においても、それらが連続であれば偏微分の順序を好感しても公開導関数は変わらない。例えば、三階偏導関数がすべて連続ならば、
$f_{xxy}=f_{xyx}=f_{yxx},f_{yyx}=f_{yxy}=f_{xyy}$が成り立つ。
\newpage
\subsection{全微分}
つぎに、関数$f(x,y)$が$x,y$でともに変化するときのことを考えよう。一変数の場合には、$\Delta x$を$x$の増分、$\Delta y$を$y$の増分としたときに、
$\Delta x$が十分小さければ、
\begin{equation}
    \Delta y=f'(x)\Delta x+\epsilon \Delta x
\end{equation}
の関係があったのであった。一方、$y=f(x)$の微分は
\begin{equation}
    dx=\Delta x\quad dy=f'(x)dx
\end{equation}
である。

さて、関数$z=f(x,y)$において、$x$の増分$\Delta x$と$y$の増分$\Delta y$に対する全増分を$\Delta z$とする。すなわち、
\begin{equation}
    \Delta z=f(x+\Delta x,y+\Delta y)-f(x,y)
\end{equation}
右辺から、同じものを引き、足して
\begin{equation}
    \Delta z=\{f(x+\Delta x,y+\Delta y)-f(x,y+\Delta y)\}+\{f(x,y+\Delta y-f(x,y)\}
\end{equation}
と書き直す。第一項では$x$のみが変化し、第二項では$y$だけが変化している。どちらも独立変数の一方だけが変化しているのだから、すでに登場した平均値の定理を用いて、
\begin{equation}
    \Delta z=f_x(x+\theta_1\Delta x,y+\Delta y)\Delta x+f_y(x,y+\theta_2\Delta y)\Delta y\quad(0<\theta_1<1,0<\theta_2<1)
\end{equation}
と書ける。偏導関数$f_x,f_y$が連続であれば、
\begin{align}
    &f_x(x+\theta_1\Delta x,y+\Delta y)=f_x(x,y)+\epsilon_1\\
    &f_y(x,y+\theta_2\Delta y)=f_y(x,y)+\epsilon_2
\end{align}
において、$\Delta x,\Delta y$が0に収束するとき、$\epsilon_1.\epsilon_2$はともに0に収束する。したがって、$\Delta x,\Delta y$が十分に小さければ、(7)~(9)の式より
\begin{equation}
    \Delta x=f_x(x,y)\Delta x+f_y(x,y)\Delta y+(\epsilon_1\Delta x+\epsilon_2\Delta y)
\end{equation}
を得る。この式は一変数関数に対する式(3)を二変数に拡張したものである。

$z=f(x,y)$の二変数関数において微分を定義する。独立変数$x,y$の微分、$dx,dy$は任意の増分$\Delta x,\Delta y$とする。
\begin{equation}
    dx=\Delta x\quad dy=\Delta y
\end{equation}
そして、関数$z=f(x,y)$の\textbf{全微分}(または単に\textbf{微分})を
\begin{equation}
    dz=f_x(x,y)dx+f_y(x,y)dy
\end{equation}
と定義する。(10)式と比べると、$\Delta x,\Delta y$が小さいならば、全微分$dz$は関数の全増分$\Delta z$のよい近似値を与えることがわかる。
\subsubsection*{例題5}
$z=xy$とするとき、$\Delta z$と$dz$はどのくらい異なるか答えよ。
\subsubsection*{解答}
$z_x=y,z_y=x$であるため、$dz=ydx+xdy$である。一方で、$x,y$の増分が$\Delta x=dx,\Delta y=dy$であるとき、全増分$\Delta x$は
$\Delta z=(x+\Delta x)(y+\Delta y)-xy=ydx+xdy+dxdy$である。よって$dz$と$\Delta z$は$\Delta x\Delta y=dxdy$だけ異なる。

変数が多くあるとき、$u=f(x,y,x,\cdots,t)$の全微分$du$は
\begin{equation}
    du=u_xdx+u_udy+u_zdz+\cdots+u_tdt
\end{equation}
で定義される。二変数のときと同様に$\Delta x=dx,\Delta y=dy,\Delta z=dz,\cdots,\Delta t=dt$が十分小さいとき、全微分$du$は全増分$\Delta u$のよい近似値になる。言葉で
説明するなら、「いくつもの小さな変化がもたらす変動全体は、各々の変化による変動を足し合わせ得たものとみなせる」ということを意味している。
\subsection{合成関数の微分}
関数$z=f(x,y)$において、$x,y$が変数$t$に依存している、すなわち、$x=x(t),y=y(t)$の場合の微分法を考えよう。増分と極限を用いて証明することもできるが、せっかくなのでここでは微分を用いて話をすす
めることにする。$z=f(x,y)\quad x=x(t),y=y(t)$より、
\begin{equation}
    dz=\frac{\partial f}{\partial x}dx+\frac{\partial f}{\partial y}dy\quad dx=\frac{dx}{dt}dt\quad dy=\frac{dy}{dt}dt
\end{equation}
後の二つの式を最初の式に代入して、
\begin{equation}
    dz=\frac{\partial f}{\partial x}\frac{dx}{dt}dt+\frac{\partial f}{\partial y}\frac{dy}{dt}dt
    =\left(\frac{\partial f}{\partial x}\frac{dx}{dt}+\frac{\partial f}{\partial y}\frac{dy}{dt}\right)dt
\end{equation}
一方で、$z$が$t$の関数とみなせるから。
\begin{equation}
    dz=\frac{dz}{dt}dt
\end{equation}
(15)式と(16)式を比べて、二変数関数における合成関数の微分法則
\begin{equation}
    \frac{dz}{dt}=\frac{\partial f}{\partial x}\frac{dx}{dt}+\frac{\partial f}{\partial y}\frac{dy}{dt}
    =\frac{\partial z}{\partial x}\frac{dx}{dt}+\frac{\partial z}{\partial y}\frac{dy}{dt}
\end{equation}
を得る。これは$z$が多変数関数の場合にも応用できる。

次に関数$z=f(x,y)$において、$x,y$は、変数$u,v$に依存して、
\begin{equation}
    x=g(u,v)\quad y=h(u,v)
\end{equation}
で与えられるとしよう。$z=f(x,y)\quad x=g(u,v)\quad y=h(u,v)$より、
\begin{equation}
    dz=\frac{\partial z}{\partial x}dx+\frac{\partial z}{\partial y}dy\quad dx=\frac{\partial x}{\partial u}du+\frac{\partial x}{\partial v}dv\quad dy=\frac{\partial y}{\partial u}du+\frac{\partial y}{\partial v}dv
\end{equation}
あとの二つの式を最初の式に代入して、
\begin{equation}
    dz=\left(\frac{\partial z}{\partial x}\frac{\partial x}{\partial u}+\frac{\partial z}{\partial y}\frac{\partial y}{\partial u}\right)du+\left(\frac{\partial z}{\partial x}\frac{\partial x}{\partial v}+\frac{\partial z}{\partial y}\frac{\partial y}{\partial v}\right)dv
\end{equation}
一方、$z$は$u,v$の関数とみなせるため、
\begin{equation}
    dz=\frac{\partial z}{\partial u}du+\frac{\partial z}{\partial v}dv
\end{equation}
(20)式と(21)式を見比べて、$du,dv$の係数をそれぞれ等しいと置き、
\begin{align}
    \frac{\partial z}{\partial u}=\frac{\partial z}{\partial x}\frac{\partial x}{\partial u}+\frac{\partial z}{\partial y}\frac{\partial y}{\partial u}\notag\\
    \frac{\partial z}{\partial v}=\frac{\partial z}{\partial x}\frac{\partial x}{\partial v}+\frac{\partial z}{\partial y}\frac{\partial y}{\partial v}
\end{align}
これは多変数関数にも応用できる。
\subsubsection*{例題6}
\begin{align*}
    u=f(x,y),\quad x=\rho\cos \phi,\quad y=\rho\sin \phi\quad\text{のとき}\\
    \left(\frac{\partial u}{\partial x}\right)^2+\left(\frac{\partial u}{\partial y}\right)^2=\left(\frac{\partial u}{\partial \rho}\right)^2+\frac{1}{\rho^2}\left(\frac{\partial u}{\partial \phi}\right)^2
\end{align*}
を示せ。

\subsubsection*{解答}
(22)式より、
\begin{align*}
    \frac{\partial u}{\partial \rho}&=\frac{\partial u}{\partial x}\cos\phi+\frac{\partial u}{\partial y}\sin \phi\\
    \frac{\partial u}{\partial \phi}&=\frac{\partial u}{\partial x}(-\rho\sin\phi)+\frac{\partial u}{\partial y}(\rho\cos\phi)
\end{align*}
よって、
\begin{equation*}
    \left(\frac{\partial u}{\partial \rho}\right)^2+\frac{1}{\rho^2}\left(\frac{\partial u}{\partial \phi}\right)^2=\left(\frac{\partial u}{\partial x}\cos\phi+\frac{\partial u}{\partial y}\sin \phi\right)^2+\left(\frac{\partial u}{\partial x}(-\sin\phi)+\frac{\partial u}{\partial y}(\cos\phi)\right)^2
    =\left(\frac{\partial u}{\partial x}\right)^2+\left(\frac{\partial u}{\partial y}\right)^2
\end{equation*}

\subsection{平均値の定理}
平均値の定理は、微分額の理論上きわめて重量な定理であるらしい。らしいというのは、私があまり微分学に精通していないからあくまで参考にしている本の受け売りとして書いたものである。それはさておき、一変数
の場合は前回に学んだ。この結果を用いて、二変数の関数$f(x,y)$に対して、平均値の定理を導く。

まず、$a,b,h,k$を定数として,関数
\begin{equation}
    F(t)=f(a+ht,b+kt)
\end{equation}
を定義する。一変数の場合の平均値の定理によって、
\begin{equation}
    F(1)=F(0)+F'(\theta)\quad(0<\theta<1)
\end{equation}
上の式の第二項$F'(\theta)$を、$f$の偏導関数で表す。$x=a+ht,y=b+kt$と置くと、合成関数の微分法則(17)式によって、
\begin{equation}
    \frac{dF(t)}{dt}=f_x\frac{dx}{dt}+f_y\frac{dy}{dt}=hf_x(a+ht,b+kt)+kf_y(a+ht,b+kt)
\end{equation}
よって、
\begin{equation}
    F'(\theta)=hf_x(a+h\theta,b+k\theta)+kf_y(a+h\theta,b+k\theta)
\end{equation}
この式と(23)式を(24)式に代入する。こうして、関数$f(x,y)$が偏微分可能ならば、
\begin{equation}
    f(a+h,b+k)=f(a,b)+hf_x(a+h\theta,b+k\theta)+kf_y(a+h\theta,b+k\theta)\quad(0<\theta<1)
\end{equation}
を得る。これを、\textbf{二変数関数の平均値の定理}という。

\subsection{偏導関数の応用}
偏微分法についての基本的な話は以上である。ここからは、知っていると便利な応用を述べることにする。
\subsubsection{陰関数の微分法}
$y=f(x)$のように、$x$の値に$y$の値を対応させる具体的な表式が示されているとき、$y$は$x$の\textbf{陽関数}であるという。
一方、
\begin{equation}
    F(x,y)=0
\end{equation}
のように、関係式として定められているだけであるときには、$y$は$x$の\textbf{陰関数}であるという。同様に$F(x,y,z)=0$のように多変数に対しても陰関数が定義できる。
余談だが、陰関数はもともと陰伏函数と呼ばれていたらしい。これはimplicit function(陰関数)のモジりである。なかなかに面白い。

さて、では陰関数に関する微分法をまとめていこう。
\paragraph{1}
$F(x,y)=0$のとき、$F_y\neq 0$ならば、
\begin{equation}
    \frac{dy}{dx}=-\frac{F_x(x,y)}{F_y(x,y)}
\end{equation}
\paragraph{2}
$F(x,y,z)=0$のとき、$F_z\neq 0$ならば、
\begin{equation}
    \frac{\partial z}{\partial x}=-\frac{F_x(x,y,z)}{F_z(x,y,z)},\quad\frac{\partial z}{\partial y}=-\frac{F_y(x,y,z)}{F_z(x,y,z)}
\end{equation}

\subsubsection*{例題7}
$x^2+3xy-2yz+xz+z^2=15$のとき、偏導関数$\displaystyle\frac{\partial z}{\partial x},\frac{\partial z}{\partial y}$を求めよ。

\subsubsection*{解答}
(30)式より
\begin{align*}
    \frac{\partial z}{\partial x}=-\frac{F_x(x,y,z)}{F_z(x,y,z)}=-\frac{2x+3y+z}{-2y+x+2z}\\
    \frac{\partial z}{\partial y}=-\frac{F_y(x,y,z)}{F_z(x,y,z)}=-\frac{3x-2z}{-2y+x+2z}
\end{align*}

\subsubsection{積分記号下での微分}
パラメータ$y(\alpha\leq y\leq\beta)$を含む定積分
\begin{equation}
    I(y)=\int_a^b f(x,y)dx
\end{equation}
を考えてみよう。積分上限、下限はともに$y$に依存しないとしよう。函数$f(x,y)$とその偏導関数$f_y(x,y)$が、
閉領域$a\leq x\leq b,\alpha\leq y\leq \beta$で連続ならば、平均値の定理(27)式より、
\begin{equation}
    f(x,y+\Delta y)-f(x,y)=\Delta y\frac{\partial f(x,y+\theta\Delta y)}{\partial y}\quad(0<\theta<1)
\end{equation}
が成り立つ。そして、関数$I(y)$の増分$\Delta I$は
\begin{equation}
    \Delta I=I(y*\Delta y)-I(y)=\int_a^b[f(x,y*\Delta y)-f(x,y)]dx=\Delta y\int_a^b\frac{\partial f(x,y+\theta\Delta y)}{\partial y}dx
\end{equation}
となる。上の式で極限$\Delta y\to 0$を取ると、$f_y(x,y)$は連続であるから
\begin{equation}
    \frac{dI}{dy}=\lim_{\Delta y\to 0}\frac{\Delta I}{\Delta y}=\int_a^b\frac{\partial f(x,y)}{\partial y}dx
\end{equation}
すなわち、定積分$I(y)$をパラメータ$y$で微分することは、積分記号下で$y$について、偏微分したものを積分することと同じになる!
この性質をうまく用いると、一見難しい定積分が簡単に求められることがある。
\subsubsection*{例題8}
(1)次の定積分を求めよ。
\begin{equation*}
    \int_0^\pi\frac{dx}{\alpha-\cos x}\quad(\alpha>1)
\end{equation*}

(2)次の定積分を求めよ。(ヒント:(1)の結果を用いよ。)
\begin{equation*}
    \int_0^\pi\frac{dx}{(3-\cos x)^2}dx
\end{equation*}

\subsubsection*{解答}
(1)普通に定積分の値を計算すればよい。
\begin{align*}
    \int_0^\pi\frac{dx}{\alpha-\cos x}&=2\int_0^\infty\frac{dt}{(\alpha-\frac{1-t^2}{1+t^2})\cdot(1+t^2)}=2\int_0^\infty\frac{dt}{(\alpha+1)t^2+(\alpha-1)}\\
    &=2\int_0^\infty\frac{dt}{(\alpha+1)t^2+(\alpha-1)}=\frac{2}{\alpha+1}\int_0^\infty\frac{dt}{t^2+\frac{\alpha-1}{\alpha+1}}=\frac{2}{\alpha+1}\left[\frac{1}{\sqrt{\frac{\alpha-1}{\alpha+1}}}\arctan \frac{t}{\sqrt{\frac{\alpha-1}{\alpha+1}}}\right]_0^\infty\\
    &=\frac{\pi}{(\alpha+1)\cdot\sqrt{\frac{\alpha-1}{\alpha+1}}}=\frac{\pi}{\sqrt{\alpha^2-1}}
\end{align*}
ここまでは去年習った内容なので、できる人も多いだろう。
(2) (1)の結果を用いる。
\begin{equation*}
    I(\alpha)=\int_0^\pi\frac{dx}{\alpha-\cos x}=\frac{\pi}{\sqrt{\alpha^2-1}}
\end{equation*}
と置く。(34)式より、
\begin{equation*}
    \frac{dI(\alpha)}{d\alpha}=\int_0^\pi\frac{\partial}{\partial \alpha}\left(\frac{1}{\alpha-\cos x}\right)dx=-\int_0^\pi\frac{dx}{(\alpha-\cos x)^2}
\end{equation*}
また、
\begin{equation*}
    \frac{d}{d\alpha}\left(\frac{\pi}{\sqrt{\alpha^2-1}}\right)=-\frac{\pi\alpha}{(\alpha^2-1)^{\frac{3}{2}}}
\end{equation*}
よって、
\begin{equation*}
    \int_0^\pi\frac{dx}{(\alpha-\cos x)^2}=\frac{\pi\alpha}{(\alpha^2-1)^{\frac{3}{2}}}
\end{equation*}
ここで、$\alpha=3$と置くと、
\begin{equation*}
    \int_0^\pi\frac{dx}{(3-\cos x)^2}dx=\frac{3\pi}{16\sqrt{2}}
\end{equation*}
なんと、定積分を微分だけで解いてしまったのだ。実際は(1)のように微分する前の関数の定積分の値を知っている必要があるが、それでもこの偏微分の有用性が伝わったと思う。

\subsection{終わりに}
今回は、偏微分について学んだ。二変数以上の関数でも微分することができるようになったことでまた一つできることの幅が広がったと思う。
次回は多重積分について学ぶ。その名の通り、多変数の定積分のことだ。これは理工学の幅広い分野で用いられ、むしろ常識となるらしい。確かに、
難しい本にはたいてい$\iiint$みたいな記号ばっかりな気がする。様々な積分が出てくると、頭が混乱してしまうかもしれない。そのため、あらかじめ
定積分を復習しておくとよいかもしれない。大切なのは、「定積分は積和の極限」であることだ。

以下に練習問題をのせておく。各自で参考にしてほしい。

\subsection{練習問題}
(1)関数$f(x,y)=x^3-2x^2y+3y^3$において,$f(0,1),f(1,1),f(2,0),f(1,-1)$の値を求めよ。

(2)次の関数の原点(0,0)での連続性を調べよ。
\begin{equation*}
    1.f(x,y)=x^2+y\qquad 
    2.f(x,y)=\left \{
        \begin{array}{l}
        \displaystyle\frac{\sin(x+y)}{x+y} \hspace{2mm}\quad((x,y)\neq(0,0))\\
        1   \qquad\qquad\quad\hspace{2mm}((x,y)=(0,0))
        \end{array}
        \right.
\end{equation*}

(3)次の関数の全微分を求めよ
\begin{equation*}
    1.z=x^4y+x^2y^2+xy^4\quad 2.\theta=\arctan(\frac{y}{x})
\end{equation*}

(4)導関数$\displaystyle\frac{du}{dt}$を求めよ。
\begin{align*}
    1.&u=x^3y^2,\quad x=t^2,\quad y=t^4\\
    2.&u=x\cos y-y\cos x,\quad x=\cos 2t,\quad y=\sin2t
\end{align*}

(5)与えられる陰関数の導関数$\displaystyle\frac{dy}{dx}$を求めよ。
\begin{equation*}
    1.x+y=e^{xy}\quad 2.x^2+y^2=1
\end{equation*}
(6)以下の定積分を解け。
\begin{equation*}
    \int_{-\infty}^{\infty}\frac{dx}{(4+x^2)^2}
\end{equation*}
ただし、
\begin{equation*}
    \int_{-\infty}^{\infty}\frac{dx}{\alpha+x^2}=\frac{\pi}{\sqrt{\alpha}}\quad(\alpha>0)
\end{equation*}
であるとする。
\end{document}
