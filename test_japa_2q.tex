\documentclass[a4paper,10pt]{article}
\usepackage{color}
\usepackage{longtable}
\usepackage[margin=12mm,nohead]{geometry}
\usepackage{hyperref}
\renewcommand{\familydefault}{\sfdefault}
\title{国語試験範囲:問題と答え一覧}
\date{}
\begin{document}
\maketitle
\vspace{-2cm}
\begin{longtable}{|p{0.45\textwidth}|p{0.45\textwidth}|}
\hline
\textbf{問題} & \textbf{答え} \\ \hline
    「イメージ」は「\textcolor{red}{潤滑油}」/食い違いがはなはだしくなると? & \textcolor{red}{潤滑油としての役目を喪失する} \\ \hline
    われわれの環境はますます多様になり,予測あるいは期待を下しながら,行動せざるをえなくなっている.つまり? & \textcolor{red}{イメージに頼りながら行動せざるをえなくなっている} \\ \hline
    イメージというものはだんだん…もとの現実と離れて独自の存在に化する.イメージが独り歩きしてしまうが,そういったイメージに対する名称 & \textcolor{red}{バケモノ} \\ \hline
    ササラ型社会とタコツボ型社会の違い & \textcolor{red}{「ササラ型」社会は異分野や異文化を横断する関係性があるが「タコツボ型」は集団ごとに閉じており,他集団との関係が希薄であるという違い.} \\ \hline
    社会が発達する…機能集団が多元的に分化(個別的な集団がタコツボ化)してくること自体は,世界的な傾向.日本の特殊性は?& \textcolor{red}{日本では教会あるいはサロンといったような役割をするものが乏しく,したがって民間の自主的なコミュニケーションのルートがはなはだ貧しい点} \\ \hline
    個別的な集団がタコツボ化し,社会がだんだん大社会になっていくと,被害者意識が増大する.何故? & \textcolor{red}{他のタコツボの人たちとコミュニケーションをとらないので,他のたこつぼであったり,他のコミュニティであったりに対して偏ったイメージを持ってしまう. そのため自分たちだけが不利を被っている,自分たちだけが良くない立場に置かれているという, コミュニケーション不足から来るマイノリティ意識から被害者意識が増大してしまう.} \\ \hline
    ポリコレ疲れ(欧米)とは?またそれはどこからくるのか? & \textcolor{red}{クレームじみた主張が縦横無尽に駆け巡り,国民が疲弊する問題.被害者意識が強まることで攻撃性というものが生まれる.} \\ \hline\hline
    ルース・ベネディクトの著作理由 & \textcolor{red}{日本人の行動様式・思考様式を解明するため} \\ \hline
    日本人は「\textcolor{red}{世間にお世話になっている}」と感覚が強い.これが[○○]になっている & \textcolor{red}{日々の決断や行動の出発点} \\ \hline
    「世間」と「社会」の違い(欧米との比較は後で別表にてまとめる.) & \textcolor{red}{社会は法律(ルール)や制度で定義づけられている世界であるが,世間は地域・親族・会社など,直接間接問わず,自分に何らかの関係のある人たちだけで形成される世界であるという違い.} \\ \hline
    ベネディクトが考える日本人にとっての「恩」とは? & \textcolor{red}{肩の荷,すなわち返すべき借りであり全力で果たすべき責任のこと./第二次世界大戦のあの極端な自己犠牲にもつながると指摘.} \\ \hline
    「恩に着る」,「恩着せがましい」という表現/恩は必ずしも受けてうれしいものではないが何故?& \textcolor{red}{日本人は偶然に人から恩を受け,したがって返礼の負い目を背負い込むことを好まないから.特に単なる知人や自分とほとんど対等な人間の場合に思う.(恩は目上の人から受けるもの.)} \\ \hline
    日本の幼児は,西欧人がおそらく想像すると思われる仕方とは,異なった仕方で育てられるがどう違う?それによって日本人の人生観,性格に何が生じる?? & \textcolor{red}{赤ん坊(と老人と)に最大の自由とわがままが許されている違い.これによって日本人の人生観や性格に二元性が生じる.} \\ \hline
    二元性とは & \textcolor{red}{幼少期に由来する奔放な側面と礼儀・恥・世間などの不文律に忠実に従う側面} \\ \hline
\end{longtable}

\end{document}
