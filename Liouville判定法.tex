\documentclass[a4j,dvipdfmx]{jsarticle}
\usepackage{amsmath,amssymb}
\usepackage{siunitx}
\usepackage{ascmac}

\begin{document}
\section*{$\frac{1}{\log x}$の原始関数が初等関数で表せないことの証明}
\subsection{Liouville判定法}
不定積分が初等関数で表せるかどうかを判定するために、Liouville判定法を用いる。
\begin{itembox}[c]{Liouville判定法}
    有理関数$f(x),g(x)$に対して、$\displaystyle f(x)e^{g(x)}$の不定積分が初等関数でかけるためには、
    ある有理関数$h(x)$が存在して、
    \begin{equation*}
        f(x)=h'(x)+h(x)g'(x)
    \end{equation*}
    と書けることが必要かつ十分である。
\end{itembox}
このことを利用して、$\frac{1}{\log x}$の不定積分が初等関数で表せないことを証明する。
\subsection{式の変形}
$\displaystyle \int \frac{dx}{\log x}$そのままの形ではLiouville判定法を使うことができないので変形する。$t=\log x$とすると、
$x=e^t$であるため、$\displaystyle dx=e^tdt$より、
\begin{equation*}
    \int \frac{dx}{\log x}=\int \frac{e^t}{t}dt
\end{equation*}
となる。よって、右辺の積分が初等関数で表せないことを示せばいい。\footnote{
    $\int f(x)dx$の$x=\phi(t)$における置換積分、$\int f(\phi(t))\phi'(t)dt$
    が存在しなければ、仮に$F'(x)=f(x)$とした場合に、$x=\phi(t)$とすると$(F(\phi(t)))'=F'(\phi(t))\phi'(t)$
    となり、両辺を$t$で積分したときに、$\int F(\phi(t))\phi'(t)dt=F(\phi(t))+C$となってしうので仮定に矛盾してしまう。
    よって、置換積分した不定積分が存在しなければ、置換する前の不定積分は存在しないことになる。
}
\subsection{証明}
$f(t)=\frac{1}{t},g(t)=t$とする。

$\displaystyle\frac{e^t}{t}$の不定積分が初等関数で書けるためには、
\begin{equation*}
    f(t)=h'(t)+h(t)g'(t)
\end{equation*}
となるような、有理関数$h(t)$が存在しなければならない。\\
ここで、$h(t)$が存在すると仮定すると、互いに素である二つの多項式、$p(t),q(t)$で
\begin{equation*}
    h(t)=\frac{p(t)}{q(t)}
\end{equation*}
と書ける。
判定法を満たす式より、
\begin{equation*}
    \frac{1}{t}=h'(t)+h(t)\to 1=h'(t)t+h(t)t
\end{equation*}
となる。これを計算すると、
\begin{align*}
    1&=\frac{p'(t)q(t)-p(t)q'(t)}{q^2(t)}\cdot t +\frac{p(t)}{q(t)}\cdot t\\
    q^2(t)&=(p'(t)q(t)-p(t)q'(t))t+p(t)q(t)t\\
    q^2(t)&=p'(t)q(t)t-p(t)q'(t)t+p(t)q(t)t\\
    p(t)q'(t)t&=q(t)(p'(t)t-q(t)+p(t)t)\tag{*}
\end{align*}
$p(t)$と$q(t)$は互いに素であるため、$q'(t)$もしくは$q'(t)t$が$q(t)$で割り切れることがわかる。

前者の場合は、$q(t)$が定数であることを意味し、$h(t)$が多項式となるため、判定式に矛盾が生じる。

後者の場合は、$q'(t)t$と$q(t)$が同じ次数であり、$q(t)$に定数項がないことを意味する。また、$q(t)$は単項式であることを意味する。なぜならば、
仮に$q(t)$が単項式ではないとすると、$a_n,a_{n-1},\cdots,a_1$を定数として、
\begin{equation*}
    q(t)=a_n t^n+a_{n-1}t^{n-1}+\cdots+a_1 t
\end{equation*}
と表せる。よって、$q'(t)t$は次のように表せる。
\begin{equation*}
    q'(t)t=a_n n t^n+a_{n-1}(n-1)t^{n-1}+\cdots+a_1 t
\end{equation*}
$q'(t)t$を$q(t)$で割ると、商は$n$、余りは
\begin{equation*}
    -a_{n-1}t^{n-1}-2a_{n-2}t^{n-2}-\cdots -(n-1)a_1t
\end{equation*}
となる。$q'(t)t$は$q(t)$で割り切れなければならないので、余りは$0$となる必要がある。よって$a_{n-1},\cdots,a_1$が
0になれば、余りが0となる。その場合、$q(t)$は次のように表せる。
\begin{equation*}
    q(t)=a_n t^n+0\cdot t^{n-1}+\cdots+0\cdot t=a_n t^n
\end{equation*}
よって仮定と矛盾するため、$q(t)$は単項式である。

ここで、(*)式に着目すると、、$a,n$を定数として、
\begin{equation*}
    p(t)\cdot ant^n=at^n(p'(t)t-at^n+p(t)t)
\end{equation*}
となっている。式を簡単にすると、
\begin{equation*}
    n\cdot p(t)=p'(t)t-at^n+p(t)t=t(p'(t)-at^{n-1}+p(t))
\end{equation*}
となり、$p(t)$は$t$で割り切れる。しかし、$p(t)$と$q(t)$が互いに素であることに矛盾である。

よって、判定式を満たす$h(t)$は存在しないので、$\displaystyle \int \frac{dx}{\log x}$は初等関数で表すことができない。
\end{document}